\documentclass[a4paper,fleqn]{article}

\usepackage[utf8]{inputenc}
\usepackage[austrian]{babel}
\usepackage[round]{natbib}
\usepackage{graphicx}
\usepackage{hyperref}

\def\email#1{{\tt#1}}
\newcommand{\pkg}[1]{{\normalfont\fontseries{b}\selectfont #1}}

\begin{document}

\title{F\"orderungsstipendium - Abschlussbericht}

\author{Stefan Theu"sl \\ Matrikelnummer: 0352689 \\ Email: \email{h0352689@wu-wien.ac.at}}

\maketitle              % typeset the title of the contribution

\section{Kurzbeschreibung der wissenschaftlichen Arbeit}

\begin{description}
\item[Titel der Diplomarbeit] Applied High Performance Computing using
  R (deutsch: Angewandtes Hochleistungsrechnen unter Verwendung von R)
\item[Betreuer] Prof. Dr. Kurt Hornik, Department of Statistics and
  Mathematics
\item[Sprache] Englisch

\end{description}

Wie der Titel schon andeutet geht es in dieser Diplomarbeit um
hochperformantes Rechnen unter Verwendung des Softwarepaketes
R \cite{Rcore}. Applikationen aus verschiedenen Bereichen der
Betriebswirtschaft (Finance) sowie aus Statistik und Mathematik,
sollen die M\"oglichkeiten, aber auch die Grenzen der parallelen
Programmierung aufzeigen.

Das Department f\"ur Statistik und
Mathematik bietet seit geraumer Zeit einen Rechencluster und somit
eine Plattform f\"ur rechenintensive Methoden an. Die neuen
M\"oglichkeiten, die die Wirtschaftsuniversit\"at nun im
computationalen Bereich bietet, sollen in dieser Arbeit
(\cite{theussl07}) 
getestet und
genutzt werden. Der Abstract der mittlerweile fertiggestellten Arbeit
lautet wie folgt:

\begin{abstract}
In den 1990er Jahren ebnete das Beowulf Projekt den Weg f\"ur massives
Parallelrechnen, weil dadurch der Zugang zu parallelen Rechenresourcen f\"ur
viele 
Forschungsinstitute und der Industrie erst leistbar wurde. Trotzdem
war der gro\ss{}e Durchbruch lange nicht erkennbar. Ausschlaggebend
daf\"ur waren zwei Aspekte:
Erstens gab es bisher keine g\"unstigen Parallelrechner und zweitens
waren auch keine einfachen Parallelprogrammiermodelle vorhanden. Die
Einf\"uhrung von Multicoreprozessoren f\"ur Desktopsysteme und
impliziter Programmiermodelle wie OpenMP f\"uhrte zu einem Umbruch,
der auch Auswirkungen auf die Softwareentwicklung hat. Der Trend geht
dabei immer mehr in Richtung Parallelrechnen.

Diese Diplomarbeitet bietet einen \"Uberblick \"uber das Feld des
Hochleistungsrechnens und im Speziellen des Parallelrechnens unter
Verwendung von R, einer Softwareumgebung f\"ur statistisches Rechnen
und Grafiken. Dar\"uber- hinaus beinhaltet diese Arbeit eine Einf\"uhrung %%SELBSTGEMACHTERUMBRUCH
in paralleles Rechnen unter Verwendung mehrerer Erweiterung zu R.

Der Kern dieser Arbeit ist die Entwicklung eines
Paketes namens  \pkg{paRc}. Diese Erweiterung beinhaltet eine
Schnittstelle zu OpenMP und stellt eine Benchmark Umgebung zur
Verf\"ugung, mit der verschiedene Paralleleprogrammiermodelle wie MPI oder
PVM sowie hochoptimierte Bibliotheken (BLAS) miteinander verglichen
werden k\"onnen. Als Aufgabe f\"ur diese Benchmarks wurde das
klassische Beispiel aus dem Bereich des Parallelrechnens, n\"amlich
die Matrix Multiplikation, gew\"ahlt.

Schlussendlich wird in dieser Arbeit eine Fallstudie aus der
computationalen Finanzwirtschaft pr\"asentiert. Im Mittelpunkt dieser
Fallstudie steht die Bewertung von Derivaten (Europ\"aische Call
Optionen) unter Verwendung paralleler Monte Carlo Simulation.
\end{abstract}

Im Zuge dieser
Diplomarbeit habe ich auch an der Konferenz EuroPVM/MPI
(\url{http://pvmmpi07.lri.fr}) teilgenommen. Der folgende Bericht
bezieht sich nun haupts"achlich auf diese Konferenz.  

\section{Bericht: Euro PVM/MPI 2007}

\subsection{MPI Tutorium}
Die Konferenz begann Sonntags mit einer kleinere Teilnehmeranzahl im
Rahmen des Tutoriums. Dieses war speziell fuer Anfaenger konzipiert
und hat erste Einblicke in die Arbeitsweise und Funktionalität von MPI
erläutert. Hierfür haben die beiden Entwickler des MPI 2.0 Standards
anhand eines einfachen Beispiels, nämlich dem ``Game of Life'',
erklärt wie man aus einer seriellen Implementation (ohne paralleles
Rechnen) schlussendlich eine parallele MPI 2.0 Applikation
entwickelt. Besonders auf die Aspekte des Ein- und Auslesens von Daten
sowie auch die Synchronisierung der parallelen Prozesse wurden
Schwerpunkte gelegt. Interessante Themen waren hier vor allem
Checkpointing sowie Performance Disk Libraries um hochperformant auf
die zugrundeliegenden Daten zuzugreifen.
Im zweiten Tutorial wurde ueber Advanced MPI Programming referiert
anhand des Beispiels der Matrix-Multiplikation und des PUMMA
Algorithmus.

\subsection{Konferenz}
Vom zweiten bis vierten Tag standen jeden Tag unterschiedliche
Themenschwerpunkte am Programm wobei die für die Diplomarbeit
relevanten Themen kurz erläutert werden.

Ein interessanter Vortrag von Rusty Lusk hatte die Vergangenheit
respektive Zukunft paralleler Programmiermodelle zum Thema. Als
vielversprechende Paradigmen wurden hier OpenMP in Verbindung mit MPI
hervorgehoben. Implizite Parallelprogrammiersprachen als Ergaenzung zu
expliziter Parallelisierung mit MPI oder PVM angesehen. Im Endeffekt
ist es schwierig im HPC Segment ueber MPI wegzuschauen (im Hinblick
auf Petascale Computing---mehrere Tausend Prozessoren sollen
angesprochen werden). Darueberhinaus gab es interessante Vortraege
ueber Tools fuer HPC (zur Performance Analysis, Performance Disk
Libraries, etc. ). Andere Vortr"age hatten die Input/Output
Problematik bei hochperformanten Anwendungen zum Thema. Hier wurden
spezielle Implementierungen empfohlen (z.B. ROMIO fuer Unix Systeme). 

Schlussendlich gab es noch Vortraege ueber Visualisierungstools,
Datenmodelle und MPI profiling die einen ersten Einblick in
professionelle MPI Parallelprogrammierung gaben.

Dazwischen sorgten auch ein Gala-Dinner im Senatshaus sowie ein von
Microsoft gesponserter Cocktail-Evening dafür, dass sich die Community
besser kennenlernen und vernetzen konnte.

\section{Res"umee}

Im Nachhinein betrachtet, hat sich die Konferenz in vielerlei Hinsicht
als gewinnbringend erwiesen. Viele neue Einblicke in die Welt des
parallelen Rechnens sowie viele Kontakte zu Forschern und Entwicklern
in der ganzen Welt werden das wissen


\begin{thebibliography}{2}
%
\bibitem {theussl07}
Theussl, S.:
Applied High Performance Computing using R.
Diploma Thesis (2007), see http://epub.wu-wien.ac.at/ in preparation

\bibitem {Rcore}
R Development Core Team:
R: A language and environment for statistical computing. R Foundation
for Statistical Computing, Vienna, Austria. (2006)
see http://www.R-project.org.

\end{thebibliography}


\end{document}
