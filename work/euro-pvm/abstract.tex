\documentclass{llncs}

\usepackage{makeidx}  % allows for indexgeneration
%\usepackage[utf8]{inputenc}
\usepackage[austrian]{babel}
%\usepackage[round]{natbib}
\usepackage{graphicx}
%\usepackage{hyperref}

\begin{document}

\title{Applied High Performance Computing in Statistics and Operations
Research}

\author{Reinhard Harter\inst{1} \and Stefan Theu"sl\inst{2}}

\institute{Vienna University of Economics and Business Administration,
  Augasse 2-6, 1090 Vienna, Austria\\
\email{reinhard.harter@sap.com} \and Vienna University of Economics and Business Administration,
  Augasse 2-6, 1090 Vienna, Austria\\
\email{stefan.theussl@wu-wien.ac.at}
}

\maketitle              % typeset the title of the contribution

%\begin{abstract}
%The abstract should summarize the contents of the paper
%using at least 70 and at most 150 words. It will be set in 9-point
%font size and be inset 1.0 cm from the right and left margins.
%There will be two blank lines before and after the Abstract. \dots
%\end{abstract}

\keywordname{ Parallel Computation, Statistics, Operation Research,
  R-project, COIN-OR }
\\
% Aufmacher
As a matter of fact parallel processing has become more and more
important in the last decade. Nowadays not only clusters of
workstations are capable of processing threads in parallel but also
mainstream consumer PCs, as technology has driven the cost of
multiprocessors down enough. Furthermore the computational complexity of
applications in Statistics and Operations Research have advanced
rapidly in the last years. Reducing the amount of time an application
needs to calculate a solution for a time-critical process has become a
major task not only for IT professionals but also for researches and
applicants developing software in Statistics or Operations Research.  

% Ziele der Arbeit

In the diploma thesis by \cite{theussl07} a comparison of parallel
processing techniques is being presented. Implementations using either
the Message Passing Interface (MPI), Parallel Virtual Machine (PVM)
and compiler driven parallelizer (OpenMP) of various
CPU intensive algorithms like heuristics for the travelling salesman
problem (TSP) or Monte Carlo Simulation of option prices are
compared to each other. 
As parallel programs are traditionally designed using low-level
message-passing (PVM or MPI) the aim is also to make use of
contributed packages to the R-project, which is a free software
environment for statistical
computing and graphics, to provide high level functions so that
enduser are able to make use of distributed or parallel computing
without having to look into the details of how to implement threads in
parallel using these message-passing libraries.

% Ergebnisse

Results show speedup lines for each implementation either in C or R. 

% verwendete Infrastruktur
The performance of the applications is evaluated using a cluster of 68
Linux workstations, running a Core 2 Duo 2.4 GHz???? and four workstations
running two Dual-Core Xeons at 3 GHz????. Message-passing is achieved using the MPI
implementation LAM as well as PVM. The programms are written in C and R. 


% Lobeshymnen auf R und OpenSource


\end{document}
