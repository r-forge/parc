\documentclass{llncs}

\usepackage{makeidx}  % allows for indexgeneration
%\usepackage[utf8]{inputenc}
\usepackage[austrian]{babel}
%\usepackage[round]{natbib}
\usepackage{graphicx}
%\usepackage{hyperref}

\begin{document}

\title{Applied High Performance Computing in Statistics and Operations
Research}

\author{Reinhard Harter\inst{1} \and Stefan Theu"sl\inst{2}}

\institute{Vienna University of Economics and Business Administration,
  Augasse 2-6, 1090 Vienna, Austria\\
\email{reinhard.harter@sap.com} \and Vienna University of Economics and Business Administration,
  Augasse 2-6, 1090 Vienna, Austria\\
\email{stefan.theussl@wu-wien.ac.at}
}

\maketitle              % typeset the title of the contribution

%\begin{abstract}
%The abstract should summarize the contents of the paper
%using at least 70 and at most 150 words. It will be set in 9-point
%font size and be inset 1.0 cm from the right and left margins.
%There will be two blank lines before and after the Abstract. \dots
%\end{abstract}

\keywordname{ Parallel Computation, Statistics, Operation Research,
  R-project, COIN-OR }
\\
% Aufmacher gleich fuer Stef + Reinhard

As a matter of fact parallel processing has become more and more
important in the last decade. Nowadays not only clusters of
workstations are capable of processing threads in parallel but also
mainstream consumer PCs, as technology has driven the cost of
multiprocessors down enough. Furthermore the computational complexity of
applications in Statistics and Operations Research has advanced
rapidly in the last years. Reducing the amount of time for calculating
a solution in a time-critical process has become a
major task not only for IT professionals but also for researchers and
users developing software in Statistics or Operations Research.   

% Ziele der Arbeit

In our work we present a comparison of parallel
processing techniques \cite{theussl07}. Implementations using either
the Message Passing Interface (MPI), Parallel Virtual Machine (PVM)
and compiler driven parallelizer (OpenMP) of various
CPU intensive algorithms are
compared to each other. There is a special focus on the Travelling Salesman
Problem (TSP) and Monte Carlo Simulation of option prices. 
As parallel programs are traditionally designed using low-level
message-passing (PVM or MPI) the aim is also to make use of
contributed packages to the R-project for Statistical Computing
\cite{Rcore} to provide high level functions.
End-users should be able to make use of distributed or parallel computing
without having to look into the details of how to implement threads in
parallel using these message-passing libraries.  

% Ergebnisse Stefan

Results show speedup lines (CPU counts up to 150) for each application
either compiled in C or running in R. Furthermore, for the Travelling
Salesman Problem performance is
compared to other heuristics and the branch and cut solver
Concorde. This is done with R by using the contributed package
TSP. Last but not least economical interpretation of the findings are
given.


% Ziele Reinhard
Solving mixed-integer optimization problems has hardly ever been a trivial
task. In the second part of our presentation we focus on possibilities
to solve these problems using 
available open-source alternatives like
COIN-OR\cite{rharter07}.
The Computational Infrastructure for Operations Research
project (COIN-OR) is an initiative to spur the development of open-source
software for the Operations Research community \cite{lougeeheimer2003coi}.
It is a constantly growing repository of source code, models, data and
examples available under open-source licenses. 
A specific implementation using COIN-OR
and how to use SYMPHONY as part of COIN-OR for solving
optimization problems is shown. A special focus on high performance computing
(HPC) on multiple processors is made in our presentation.
An introduction to the modeling language AMPL and it�s open-source
alternative using the GLPK package is provided in order to have all the
necessary tools for implementation available.
Furthermore, practical examples and solutions
are shown, which demonstrate the capabilities of the used techniques in
an HPC computing environment.

% Ergebnisse Reinhard
In addition an introduction into mixed-integer linear programming on multiple
processors are given in \cite{rharter07}. Starting with theory and
necessary modeling languages (GLPK, AMPL) an overview
of different high performance computing techniques and their usage will be
provided. PVM, OpenMP as well as Cluster OpenMP by Intel will be
technologies of choice. 
Used programming languages in the COIN-OR package are C (used in SYMPHONY) and
C++ (as used in BCP, Branch Cute Price).
Furthermore we are exploring the possibilities of connecting COIN-OR
projects to the widely used language R.


% verwendete Infrastruktur
The computational infrastructure is provided by the Department of
Statistics and Mathematics of the Vienna University of Economics and
Business Administration. The performance of the applications is
evaluated using a cluster of 68 
Linux workstations, running an Intel Core 2 Duo 6600 at 2.4 GHz and
four workstations 
running two Intel Dual-Core Xeons 5140 at 2.33 GHz. Message-passing is achieved
using the MPI implementation LAM as well as PVM. The programs are
written in C and R. C programs are compiled using either the GNU GCC
compiler or the Intel compiler.

% Lobeshymnen auf R und OpenSource

% ---- Bibliography ----
%
\begin{thebibliography}{2}
%
\bibitem {theussl07}
Theussl, S.:
Applied High Performance Computing using R.
Diploma Thesis (2007), see http://epub.wu-wien.ac.at/ in preparation

\bibitem {rharter07}
Harter, R.:
High Performance Computing using COIN-OR.
Diploma Thesis (2007), see http://epub.wu-wien.ac.at/ in preparation

\bibitem {lougeeheimer2003coi}
Lougee-Heimer, R.:
The common optimization INterface for operations research: promoting
open-source software in the operations research community.
IBM Journal of Research and Development {\bf 47} (2003) 57--66


\bibitem {Rcore}
R Development Core Team:
R: A language and environment for statistical computing. R Foundation
for Statistical Computing, Vienna, Austria. (2006)
see http://www.R-project.org.

\end{thebibliography}


\end{document}
