\documentclass[a4paper,fleqn]{article}

\usepackage[austrian]{babel}
\usepackage[latin1]{inputenc}
\usepackage[round]{natbib}
\usepackage{graphicx, url}
\usepackage{hyperref}

\def\email#1{{\tt#1}}

\begin{document}

\title{Antrag F\"orderungsstipendium - Beschreibung der noch nicht
  abgeschlossenen Arbeit}

\author{Reinhard Harter \\ Matrikelnummer: 0353051 \\ Email:\email{h0353051@wu-wien.ac.at}}

\maketitle              % typeset the title of the contribution

\section{Kurzbeschreibung}

\begin{description}
\item[Titel der Diplomarbeit] High Performance Computing using COIN-OR
\item[Betreuer] Prof. Dr. Kurt Hornik, Department of Statistics and
  Mathematics
\item[Sprache] Englisch

\end{description}

Diese Diplomarbeit besch\"aftigt sich mit dem L\"osen komplexer nicht-linearer
Optimierungsprobleme, mit einem speziellen Fokus auf verteiltem Rechnen.
Verteiltes Rechnen erm\"oglicht die immensen M\"oglichkeiten eines Cluster-Rechners, wie
wir ihn am Institut f\"ur Mathematik und Statistik vorfinden, auszun\"{u}tzen und hochperformantes Rechnen durchzuf\"uhren.
Aufgaben die fr\"uher Jahre gebraucht h\"atten, k\"onnen nun in wenigen Stunden und Tagen
durchgef\"uhrt werden.
Es wird gezeigt, wie diese in der Praxis sehr relevanten Problemstellungen mithilfe von Open Source L\"osungen (z.B. COIN-OR \cite{lougeeheimer2003coi}) gel\"ost werden k\"onnen.
Durch die Unterst\"utzung von verteiltem Rechnen und deren Techniken wie PVM, OpenMP und Cluster OpenMP sind Open Source Frameworks auch wirtschaftlich interessant und kommerziell einsetzbar. 

Die Teilnahme an der Konferenz EuroPVM/MPI (Detaillierte Informationen finden sich unter \url{http://pvmmpi07.lri.fr}) wird weitere Aufschl\"usse geben und Zukunftsperspektiven er\"offnen, welche in diese Diplomarbeit einfliessen werden und einen gro�en Nutzen f\"ur die zuk\"unftige Verwendung des Cluster-Rechners mit sich bringen wird.

Es wurde auch fristgerecht folgender Abstract bei der Konferenz \linebreak EuroPVM/MPI  eingereicht, um die Zwischenergebnisse im Rahmen der Erstellung der Diplomarbeit auch dort vorzutragen und zu diskutieren:

\begin{abstract}
Solving mixed-integer optimization problems has hardly ever been a trivial
task. The diploma thesis by \cite{rharter07} focuses on possibilities to solve these problems using
available open-source alternatives with a special focus on COIN-OR.
The Computational Infrastructure for Operations Research (COIN) project is an initiative to spur the development of open-source software for the
operations research community.
It is a constantly growing repository of source code, models, data and
examples available under open-source licenses.
The thesis gives you an introduction into linear and integer programming in
general. 
Afterwards we focus on the specific implementation using COINOR
and show how to use SYMPHONY as part of COIN-OR solving for
optimization problems with a special focus on high performance computing
(HPC) on multiple processors.
An introduction to the modeling language AMPL and it's open source
alternative using the GLPK package is provided in order to have all the
necessary tools for implementation .
The thesis then moves on to a few practical examples and solutions
are being shown, which demonstrate the capabilities of the used techniques in
an HPC computing environment.
 
The diploma thesis will give you a introduction into
mixed-integer linear programming on multiple processors.
Starting with theory and necessary modeling languages (GLPK, AMPL) an overview
of different high performance computing techniques and their usage will be
provided. PVM, OpenMP as well as ClusterOpenMP by Intel will be the technologies of our choice.
Used programming languages in the COIN-OR package are C (used in SYMPHONY) and
C++ (as used in BCP, Branch Cute Price).
Furthermore we're exploring the possibilities of connecting COIN-OR projects to
the widely used language R.

% Ziele der Arbeit
The computational infrastructure is provided by the Department of
Statistics and Mathematics of the Vienna University of Economics and
Business Administration. The performance of the applications is
evaluated using a cluster of 68 
Linux workstations, running an Intel Core 2 Duo 6600 at 2.4 GHz and
four workstations 
running two Intel Dual-Core Xeons 5140 at 2.33 GHz. Message-passing is achieved
using the MPI implementation LAM as well as PVM. The programs are
written in C, C++ and R. C programs are compiled using either the GNU GCC
compiler or the Intel compiler.

\end{abstract}


\begin{thebibliography}{2}
%
\bibitem {rharter07}
Harter, R.:
High Performance Computing using COIN-OR
Diploma Thesis (2007)

\bibitem{lougeeheimer2003coi}
Lougee-Heimer, R.: The common optimization INterface for operations research: promoting open-source software in the operations research community.
IBM Journal of Research and Development
volume 47, number 1, pages 57-66, year 2003

\end{thebibliography}

\end{document}
