\documentclass[a4paper,fleqn]{article}

\usepackage[utf8]{inputenc}
\usepackage[austrian]{babel}
\usepackage[round]{natbib}
\usepackage{graphicx}
\usepackage{hyperref}

\def\email#1{{\tt#1}}

\begin{document}

\title{Antrag F\"orderungsstipendium - Beschreibung der noch nicht
  abgeschlossenen Arbeit}

\author{Stefan Theu"sl \\ Matrikelnummer: 0352689 \\ Email:\email{h0352689@wu-wien.ac.at}}

\maketitle              % typeset the title of the contribution

\section{Kurzbeschreibung}

\begin{description}
\item[Titel der Diplomarbeit] Applied High Performance Computing using
  R
\item[Betreuer] Prof. Dr. Kurt Hornik, Department of Statistics and
  Mathematics
\item[Sprache] Englisch

\end{description}

Wie der Titel schon suggeriert geht es in dieser Diplomarbeit um
hochperformantes Rechnen unter Verwendung des Softwarepaketes
R. Applikationen aus verschiedenen Bereichen der
Betriebswirtschaft wie Finance und Marketing sowie aus der Statistik
sollen die M\"oglichkeiten aber auch die Grenzen der parallelen
Programmierung aufzeigen. Der folgende Abstract wurde bei der
Konferenz EuroPVM/MPI eingereicht: 

\begin{abstract}
As a matter of fact parallel processing has become more and more
important in the last decade. Nowadays not only clusters of
workstations are capable of processing threads in parallel but also
mainstream consumer PCs, as technology has driven the cost of
multiprocessors down enough. Furthermore the computational complexity of
applications in Statistics and Operations Research have advanced
rapidly in the last years. Reducing the amount of time, an application
needs to calculate a solution for a time-critical process, has become a
major task not only for IT professionals but also for researches and
applicants developing software in Statistics or Operations Research.  
In my diploma thesis a comparison of parallel
processing techniques is being presented. Implementations using either
the Message Passing Interface (MPI), Parallel Virtual Machine (PVM)
and compiler driven parallelizer (OpenMP) of various
CPU intensive algorithms like heuristics for the travelling salesman
problem (TSP) or Monte Carlo Simulation of option prices are
compared to each other. 
As parallel programs are traditionally designed using low-level
message-passing (PVM or MPI) the aim is also to make use of
contributed packages to the R-project, which is a free software
environment for statistical
computing and graphics, to provide high level functions so that
enduser are able to make use of distributed or parallel computing
without having to look into the details of how to implement threads in
parallel using these message-passing libraries.
Results show speedup lines (cpu counts up to 150) for each application
either compiled in C or running in R. Furthermore, for the Travelling
Salesman Problem performance is
compared to other heuristics and the branch and cut solver
Concorde. This is done with R by using the contributed package
TSP. Last but not least economical interpretation of the findings are
given.
The performance of the applications is evaluated using a cluster of 68
Linux workstations, running a Core 2 Duo 2.4 GHz???? and four workstations
running two Dual-Core Xeons at 3 GHz????. Message-passing is achieved
using the MPI implementation LAM as well as PVM. The programms are
written in C and R.

\end{abstract}

\end{document}
