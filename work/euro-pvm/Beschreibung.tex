\documentclass[a4paper,fleqn]{article}

\usepackage[utf8]{inputenc}
\usepackage[austrian]{babel}
\usepackage[round]{natbib}
\usepackage{graphicx}
\usepackage{hyperref}

\def\email#1{{\tt#1}}

\begin{document}

\title{Antrag F\"orderungsstipendium - Beschreibung der noch nicht
  abgeschlossenen Arbeit}

\author{Stefan Theu"sl \\ Matrikelnummer: 0352689 \\ Email: \email{h0352689@wu-wien.ac.at}}

\maketitle              % typeset the title of the contribution

\section{Kurzbeschreibung}

\begin{description}
\item[Titel der Diplomarbeit] Applied High Performance Computing using
  R
\item[Betreuer] Prof. Dr. Kurt Hornik, Department of Statistics and
  Mathematics
\item[Sprache] Englisch

\end{description}

Wie der Titel schon andeutet geht es in dieser Diplomarbeit um
hochperformantes Rechnen unter Verwendung des Softwarepaketes
R \cite{Rcore}. Applikationen aus verschiedenen Bereichen der
Betriebswirtschaft, wie Finance und Marketing sowie aus der Statistik,
sollen die M\"oglichkeiten, aber auch die Grenzen der parallelen
Programmierung aufzeigen.

Das Department f\"ur Statistik und
Mathematik bietet seit geraumer Zeit einen Rechencluster und somit
eine Plattform f\"ur rechenintensive Methoden an. Die neuen
M\"oglichkeiten, die die Wirtschaftsuniversit\"at nun im
computationalen Bereich bietet, sollen in dieser Arbeit getestet und
genutzt werden.

Im Zuge dieser
Diplomarbeit ist auch die Teilnahme an der Konferenz EuroPVM/MPI
(\url{http://pvmmpi07.lri.fr}) angedacht. Der folgende Abstract wurde
dort eingereicht, um im Rahmen dieser Veranstaltungen
Ergebnisse mit anderen Teilnehmern zu diskutieren, bzw. internationale
Erfahrung zu sammeln: 

\begin{abstract}
As a matter of fact parallel processing has become more and more
important in the last decade. Nowadays not only clusters of
workstations are capable of processing threads in parallel but also
mainstream consumer PCs, as technology has driven the cost of
multiprocessors down enough. Furthermore the computational complexity of
applications in Statistics and Operations Research has advanced
rapidly in the last years. Reducing the amount of time for calculating
a solution in a time-critical process has become a
major task not only for IT professionals but also for researchers and
users developing software in Statistics or Operations Research.   

% Ziele der Arbeit

In our work we present a comparison of parallel
processing techniques \cite{theussl07}. Implementations using either
the Message Passing Interface (MPI), Parallel Virtual Machine (PVM)
and compiler driven parallelizer (OpenMP) of various
CPU intensive algorithms are
compared to each other. There is a special focus on the Travelling Salesman
Problem (TSP) and Monte Carlo Simulation of option prices. 
As parallel programs are traditionally designed using low-level
message-passing (PVM or MPI) the aim is also to make use of
contributed packages to the R-project for Statistical Computing
\cite{Rcore} to provide high level functions.
End-users should be able to make use of distributed or parallel computing
without having to look into the details of how to implement threads in
parallel using these message-passing libraries.  

% Ergebnisse Stefan

Results show speedup lines (CPU counts up to 150) for each application
either compiled in C or running in R. Furthermore, for the Travelling
Salesman Problem performance is
compared to other heuristics and the branch and cut solver
Concorde. This is done with R by using the contributed package
TSP. Last but not least economical interpretation of the findings are
given.\\
\ldots \\
The computational infrastructure is provided by the Department of
Statistics and Mathematics of the Vienna University of Economics and
Business Administration. The performance of the applications is
evaluated using a cluster of 68 
Linux workstations, running an Intel Core 2 Duo 6600 at 2.4 GHz and
four workstations 
running two Intel Dual-Core Xeons 5140 at 2.33 GHz. Message-passing is achieved
using the MPI implementation LAM as well as PVM. The programs are
written in C and R. C programs are compiled using either the GNU GCC
compiler or the Intel compiler.

\end{abstract}


\begin{thebibliography}{2}
%
\bibitem {theussl07}
Theussl, S.:
Applied High Performance Computing using R.
Diploma Thesis (2007), see http://epub.wu-wien.ac.at/ in preparation

\bibitem {Rcore}
R Development Core Team:
R: A language and environment for statistical computing. R Foundation
for Statistical Computing, Vienna, Austria. (2006)
see http://www.R-project.org.

\end{thebibliography}


\end{document}
