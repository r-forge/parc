\documentclass[a4paper,fleqn]{article}

\usepackage[utf8]{inputenc}
\usepackage[austrian]{babel}
\usepackage[round]{natbib}
\usepackage{graphicx}
\usepackage{hyperref}


\def\email#1{{\tt#1}}

\title{F\"orderungsstipendium - Effektive Kostenaufwendung}
\author{Stefan Theu"sl\\Matrikelnummer: 0352689\\Email: \email{h0352689@wu-wien.ac.at}}

\begin{document}

\author{Stefan Theu"sl \\Matrikelnummer: 0352689\\ \email{h0352689@wu-wien.ac.at}}


\maketitle

\section{Kostenaufwand}

Folgend eine detaillierte Aufschl\"usselung der besonderen Kosten, die
im Zusammenhang mit der Erstellung der wissenschaftlichen Arbeit
entstanden sind. Im Besonderen geht es um den Besuch der Konferenz
EuroPVM/MPI 2007 in Paris im Zeitraum vom 30.09.2007 bis 03.10.2007
(Details: \url{http://pvmmpi07.lri.fr/}) sowie um Spezialliteratur.

\begin{table}[ht]
  \centering
  \caption{Kostenaufstellung}
  \label{table:Kostenaufstellung}
  \begin{tabular}{|l|l|}
    \hline
    Kostenart                              & Kosten in Euro \\
    \hline
    Spezialliteratur                                    & \\
    ~~~Handbook of Computational Statistics (\$$188.86^*$)  & \\
    ~~~Algorithmics for Hard Problems       (\$$46.30^*$ )   & \\
    ~~~Zoll                                 (EUR 6.50)  & \\
    ~~~Linear Algebra                       (EUR 66.99) & \\
    ~~~SUMME:                                       & 258.95\\ 
    Teilnahmegeb\"uhr f\"ur Konferenz EuroPVM/MPI07 & 250  \\
    Teilnahme an einem Tutorium                     &  75  \\
    Flugkosten nach Paris                           & 147.52  \\
    Hotelkosten f\"ur viert\"agigen Aufenthalt      & 650  \\
    \hline
    Gesamt                                          & 1381.47  \\  
    \hline
  \end{tabular}
\end{table}

\textit{* \ldots{} Aufgrund des g\"unstigen Wechselkurses und der Tatsache,
dass diese B"ucher in den USA um 25\% g"unstiger waren, war der Import
die "okonomisch beste L"osung. Der Preis in Euro betr"agt f"ur beide
B"ucher EUR 185.46 (siehe Kreditkartenabrechnung).}

\newpage
\section{Gegen\"uberstellung Kostenaufwand - erhaltenes Stipendium}

Das aus erhaltene  F"ordersipendium wird in
Tabelle~\ref{table:Finanzierungsplan} dem realen Aufwand
gegen"ubergestellt. Die Differenz von EUR 171.47 wurde aus Eigenmittel
finanziert.

\begin{table}[ht]
  \centering
  \caption{Gegen"uberstellung Kosten/Stipendium}
  \label{table:Finanzierungsplan}
  \begin{tabular}{|l|l|}
    \hline
    Gegen\"uberstellung                           & Euro \\
    \hline
    Erhaltenes Stipendium                         & 1240    \\
    Summe Kostenaufwand                           &-1381.47 \\
    \hline
    \textbf{Differenz:}                    &\textbf{-171.47}\\
    \hline
  \end{tabular}
\end{table}

\end{document}
