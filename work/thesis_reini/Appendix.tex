%%%%%%%%%%%%%%%%%%%%%%%%%%%%%%%%%%%%%%%%%%%%%%%%%%%%%%%%%%%%%%%%%%%
%                                                                 %
%                            APPENDICES                           %
%                                                                 %
%%%%%%%%%%%%%%%%%%%%%%%%%%%%%%%%%%%%%%%%%%%%%%%%%%%%%%%%%%%%%%%%%%%
 
\appendix
\addcontentsline{toc}{chapter}{Appendices}
\chapter{Cluster experiences}

To test our cluster we first have to install the necessary software, starting with the Cluster openMP intel compiler.

For a first test the evaluation version downloaded from www.intel.com will be sufficient. After the registration you'll receive a serial key valid for 30 days.

VV7M-PNSJ6FG8 

Latest download instruction for cluster openMP evaluation version
Installing as root (using sudo if you have that privilege) is recommended, as that will update the system RPM database. 

\section{Setting Up the Compiler Environment}
\label{sec:SettingUpTheCompilerEnvironment}

The programs in the Intel C++ Compiler 9.1 for Linux product rely on the environment variables \verb*|PATH| and \verb*|LD_LIBRARY_PATH|.
The installation script (install.sh) creates compiler environment script files (iccvars.sh/idbvars.sh) that set these variables. It is strongly recommended that you add those script files into your login script (.login file). 
Once the variables are set in the .login file there is no need to run the script files for each session.

source the script to setup the compiler environment:

    * > source <install-dir>/bin/iccvars.sh(.csh)
      to use icc
    * > source <install-dir>/bin/idbvars.sh(.csh)
      to use idb

\begin{verbatim}
wget http://registrationcenter-download.intel.com/irc_nas/659/l_cc_c_9.1.047.tar.gz
tar -xzvf l_cc_c_9.1.047.tar.gz
%
cd l_cc_c_9.1.047/
./install.sh
%
The installation program was not able to detect the IA32 version of the following libraries installed   :
libstdc++
libgcc
glibc


Link to cluster openMP installation guide:

http://registrationcenter-download.intel.com/irc_nas/659/l_cc_c_9.1.047_InstallationGuide.htm

 ./configure CC=icc CXX=icc LD=link
 
 ./configure --with-gmpl --with-glpk-incdir=/home/rharter/usr/gnu/include 						--with-glpk-lib="-L/home/rharter/usr/gnu/lib -lglpk"	--with-openmp CC=icc CXX=icc LD=link
 
 besseres format: 
 ./configure --with-gmpl --with-glpk-incdir=/home/rharter/usr/gnu/include --with-glpk-lib="-L/home/rharter/usr/gnu/lib -lglpk" --with-openmp CC=icc CXX=icc LD=link

das mit CC=icc und CXX=icc scheint nicht wirklich zu funktionieren.


GLPK:

VERSION 4.15 is not working, so use 4.13!!!!!!!!!

wget http://gd.tuwien.ac.at/gnu/gnusrc/glpk/glpk-4.13.tar.gz
tar -xzvf glpk-4.13.tar.gz

./configure CC=icc CXX=icc LD=link
make prefix=~/usr/gnu
make prefix=~/usr/gnu install


Bignode:

#$ -N mein_job_name
#$ -q bignode.q@bignode001

echo "dieser job laueft auf der bignode queue mit xeons"

echo "finished"

Damit rechnet ers auf bignode001

#$ -q bignode.q@bignode001

Normaler MPI job:

#$ -N mpi-pachecoEx2
#$ -pe lam 20
#$ -q node.q

mpirun -np 20  ./pachecoEx2-mpi.x

20 prozessoren aktivieren und mit lam rechnen

#$ -pe lam 20

Stefans Aufruf von symphony:

 #$ -N openMP-symphony
#$ -pe pvm 20
#$ -q node.q

./symphony model.dat

\end{verbatim}
