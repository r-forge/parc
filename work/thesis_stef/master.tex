\documentclass[12pt,a4paper]{report}
% zur Kontrolle des Umbruchs Klassenoption draft verwenden
%\usepackage[ngerman]{babel}
\usepackage[latin1]{inputenc}
%\usepackage[T1]{fontenc}
\usepackage{graphicx}
\usepackage{latexsym}
\usepackage{amsmath,amssymb,amsfonts}
\usepackage{ifpdf}
\usepackage{url}
%% \usepackage{/usr/share/R/share/texmf/Sweave} see below new environment
\usepackage[round]{natbib}
%% it is essential that hyperref is before the algorithm pkg and float
%% before hyperref
\usepackage{float}
\usepackage{hyperref}
\usepackage{algorithm}
\usepackage{algorithmic}


% Seitenlayout
% DIV# gibt den Divisor f�r die Layoutberechnung an.
% Vergr��ern des Divisors vergr��ert den Textbereich.
% BCOR#cm gibt die Breite des Bundstegs an.
%\usepackage[DIV11,BCOR2cm]{typearea}

% Abstand obere Blattkante zur Kopfzeile ist 2.54cm - 15mm
\setlength{\topmargin}{-5mm}


%% commands, environments, etc.
\let\code=\texttt
\newcommand{\pkg}[1]{{\normalfont\fontseries{b}\selectfont #1}}
\newcommand{\proglang}[1]{\textsf{#1}}
\newcommand{\class}[1]{`\code{#1}'}


\usepackage{theorem}
\theoremheaderfont{\normalfont\bfseries}
\theorembodyfont{\normalfont}
\newtheorem{Example}{Example}[chapter]

\newcommand{\ackname}{Acknowledgments}
\newenvironment{acknowledgments}%
   {\section{\ackname}}
   {\vfil}


%% example for newenvironment
%% \newenvironment{Proof}{\noindent{Proof:}}{%
%%    \hspace*{\fill}$\Box$\par\vskip2ex}

\RequirePackage{fancyvrb}
\RequirePackage{color}

%\newenvironment{Schunk}{\par\begin{minipage}{\textwidth}}{\end{minipage}}
\newenvironment{Schunk}{}{}

\definecolor{Sinput}{rgb}{0,0,0.56}
\definecolor{Scode}{rgb}{0,0,0.56}
\definecolor{Soutput}{rgb}{0.56,0,0}

\DefineVerbatimEnvironment{Sinput}{Verbatim}
{formatcom={\Zinput},fontsize=\small}
\DefineVerbatimEnvironment{Soutput}{Verbatim}
{formatcom={\Zoutput},fontsize=\small}
\DefineVerbatimEnvironment{Scode}{Verbatim}
{formatcom={\Zcode},fontsize=\small}

\newcommand{\Zoutput}{\vspace*{-0.55cm} \color{Soutput}}
\newcommand{\Zinput}{\vspace*{-0.3cm} \color{Sinput}}
\newcommand{\Zcode}{\vspace*{-0.2cm} \color{Scode}}
%% Title commands
\newcommand{\titel}[1]{\def\@titel{#1}\newcommand{\Titel}{#1}}
\renewcommand{\title}[1]{\def\@title{#1}\newcommand{\Title}{#1}}

%% Title
\title{Applied High Performance Computing \\Using R}
\titel{Angewandtes Hochleistungsrechnen unter Verwendung von R}


\begin{document}
  \pagestyle{empty}

  % Title Page
  \begin{titlepage}
  \begin{center}\large
    \vspace*{1cm}
    
    {\LARGE Diploma Thesis}\\
    \vspace*{2cm}

    {\huge Applied High Performance Computing using R}
    \vspace*{2cm}

    Stefan Theu\ss{}l
    \vspace*{.5cm}

    when its done \dots 
    \vspace*{2cm}

    \includegraphics[scale=0.07]{wu_logo_schwarz.jpg}
    \vspace*{2cm}

    Supervisor: o. Prof. Dipl. Ing. Dr. Kurt Hornik\\[2ex]

    Department of Statistics and Mathematics\\
    Vienna University of Economics and Business Administration
    Augasse 2-6\\
    A-1090 Vienna\\[2ex]
    
  \end{center}
\end{titlepage}
\vspace*{\fill}

% R�ckseite mit Information zum Titelbild. Dieses sollte der Arbeit 
% entnommen sein. Bedeutung des Bildes kurz erleutern und einen
% Verweis auf dessen Auftreten innerhalb der Arbeit angeben.

% Das Titelbild zeigt \dots (siehe auch Seite \dots).



  \pagenumbering{roman}
  \pagestyle{plain}

  %% Abstract and Kurzfassung
  \documentclass{llncs}

\usepackage{makeidx}  % allows for indexgeneration
%\usepackage[utf8]{inputenc}
\usepackage[austrian]{babel}
%\usepackage[round]{natbib}
\usepackage{graphicx}
%\usepackage{hyperref}

\begin{document}

\title{Applied High Performance Computing in Statistics and Operations
Research}

\author{Reinhard Harter\inst{1} \and Stefan Theu"sl\inst{2}}

\institute{Vienna University of Economics and Business Administration,
  Augasse 2-6, 1090 Vienna, Austria\\
\email{reinhard.harter@sap.com} \and Vienna University of Economics and Business Administration,
  Augasse 2-6, 1090 Vienna, Austria\\
\email{stefan.theussl@wu-wien.ac.at}
}

\maketitle              % typeset the title of the contribution

%\begin{abstract}
%The abstract should summarize the contents of the paper
%using at least 70 and at most 150 words. It will be set in 9-point
%font size and be inset 1.0 cm from the right and left margins.
%There will be two blank lines before and after the Abstract. \dots
%\end{abstract}

\keywordname{ Parallel Computation, Statistics, Operation Research,
  R-project, COIN-OR }
\\
% Aufmacher gleich fuer Stef + Reinhard

As a matter of fact parallel processing has become more and more
important in the last decade. Nowadays not only clusters of
workstations are capable of processing threads in parallel but also
mainstream consumer PCs, as technology has driven the cost of
multiprocessors down enough. Furthermore the computational complexity of
applications in Statistics and Operations Research has advanced
rapidly in the last years. Reducing the amount of time for calculating
a solution in a time-critical process has become a
major task not only for IT professionals but also for researchers and
users developing software in Statistics or Operations Research.   

% Ziele der Arbeit

In our work we present a comparison of parallel
processing techniques \cite{theussl07}. Implementations using either
the Message Passing Interface (MPI), Parallel Virtual Machine (PVM)
and compiler driven parallelizer (OpenMP) of various
CPU intensive algorithms are
compared to each other. There is a special focus on the Travelling Salesman
Problem (TSP) and Monte Carlo Simulation of option prices. 
As parallel programs are traditionally designed using low-level
message-passing (PVM or MPI) the aim is also to make use of
contributed packages to the R-project for Statistical Computing
\cite{Rcore} to provide high level functions.
End-users should be able to make use of distributed or parallel computing
without having to look into the details of how to implement threads in
parallel using these message-passing libraries.  

% Ergebnisse Stefan

Results show speedup lines (CPU counts up to 150) for each application
either compiled in C or running in R. Furthermore, for the Travelling
Salesman Problem performance is
compared to other heuristics and the branch and cut solver
Concorde. This is done with R by using the contributed package
TSP. Last but not least economical interpretation of the findings are
given.


% Ziele Reinhard
Solving mixed-integer optimization problems has hardly ever been a trivial
task. In the second part of our presentation we focus on possibilities
to solve these problems using 
available open-source alternatives like
COIN-OR\cite{rharter07}.
The Computational Infrastructure for Operations Research
project (COIN-OR) is an initiative to spur the development of open-source
software for the Operations Research community \cite{lougeeheimer2003coi}.
It is a constantly growing repository of source code, models, data and
examples available under open-source licenses. 
A specific implementation using COIN-OR
and how to use SYMPHONY as part of COIN-OR for solving
optimization problems is shown. A special focus on high performance computing
(HPC) on multiple processors is made in our presentation.
An introduction to the modeling language AMPL and it�s open-source
alternative using the GLPK package is provided in order to have all the
necessary tools for implementation available.
Furthermore, practical examples and solutions
are shown, which demonstrate the capabilities of the used techniques in
an HPC computing environment.

% Ergebnisse Reinhard
In addition an introduction into mixed-integer linear programming on multiple
processors are given in \cite{rharter07}. Starting with theory and
necessary modeling languages (GLPK, AMPL) an overview
of different high performance computing techniques and their usage will be
provided. PVM, OpenMP as well as Cluster OpenMP by Intel will be
technologies of choice. 
Used programming languages in the COIN-OR package are C (used in SYMPHONY) and
C++ (as used in BCP, Branch Cute Price).
Furthermore we are exploring the possibilities of connecting COIN-OR
projects to the widely used language R.


% verwendete Infrastruktur
The computational infrastructure is provided by the Department of
Statistics and Mathematics of the Vienna University of Economics and
Business Administration. The performance of the applications is
evaluated using a cluster of 68 
Linux workstations, running an Intel Core 2 Duo 6600 at 2.4 GHz and
four workstations 
running two Intel Dual-Core Xeons 5140 at 2.33 GHz. Message-passing is achieved
using the MPI implementation LAM as well as PVM. The programs are
written in C and R. C programs are compiled using either the GNU GCC
compiler or the Intel compiler.

% Lobeshymnen auf R und OpenSource

% ---- Bibliography ----
%
\begin{thebibliography}{2}
%
\bibitem {theussl07}
Theussl, S.:
Applied High Performance Computing using R.
Diploma Thesis (2007), see http://epub.wu-wien.ac.at/ in preparation

\bibitem {rharter07}
Harter, R.:
High Performance Computing using COIN-OR.
Diploma Thesis (2007), see http://epub.wu-wien.ac.at/ in preparation

\bibitem {lougeeheimer2003coi}
Lougee-Heimer, R.:
The common optimization INterface for operations research: promoting
open-source software in the operations research community.
IBM Journal of Research and Development {\bf 47} (2003) 57--66


\bibitem {Rcore}
R Development Core Team:
R: A language and environment for statistical computing. R Foundation
for Statistical Computing, Vienna, Austria. (2006)
see http://www.R-project.org.

\end{thebibliography}


\end{document}


  %% Table of contents
  \tableofcontents
  % show page headings now
  
    %% Chapter Introduction to Parallel Computing 
    \chapter{Introduction}
\section{Motivation}
\section{Goal of the Thesis}
\section{Organization of the Thesis}

The Thesis is organized as follows. First I briefly summarize the
fundamentals of parallel processing in Chapter
\ref{chap:parallelcomputing}. Computer architectures and programming
models as well as the principles of performance analysis are described
in the course of this chapter.

After that, possibilities of how to
use these paradigms in R are explained in Chapter
\ref{chap:Rhpc}. This part of the thesis provides an overview of the
available extensions supplying parallel computing functionality to R.

In the remaining chapters selected applications for High Performance
Computing are going to be
described. Each of these chapters has the following structure: First,
a description of the topic is given in a comprehensive way. Then
details of the implementations are going to be presented. At the end of
each chapter results of the comparison between each implementation are
shown.
The selected topics are
\begin{itemize}
\item Parallel Matrix Multiplication in Chapter \ref{chap:matrix},
\item Parallel Greedy Randomized Adaptive Search procedure in Chapter
  \ref{chap:pgrasp},
\item and Option Pricing using Parallel Monte Carlo Simulation in Chapter
  \ref{chap:options}.
\end{itemize}
Last but not least Chapter \ref{chap:conclusion} summarizes
the findings and give economical interpretations as well as an outlook
to future work.


    %% Chapter High Performance Computing - overview
    \chapter{Parallel Computing}
\section{Introduction}

In 1965 a director in the Fairchild Semiconductor division of
Fairchild Camera and Instrument Corporation predicted: "The complexity
(of integrated circuits, annotation of the author)
for minimum costs has increased at a rate of roughly two per year
[\ldots] over the short term this rate can be expected to continue, if
not increase." \cite{moore} This soon became known as Moore's Law. And
indeed for nearly fourty years the advance predicted by Moore has
taken place (the overall growth of the world's most powerful computers
has approximately doubled each 18 months).

Furthermore greater speed implicates greater memory, otherwise you have
the situation that a computer capable of performing trillions of
operations in each second only has access to a small memory which is
rather useless for solving data driven calculations.

Now, more than 40 years later "desktop" systems with two to four
processors and lots of memory are available, which means that they can
be compared with high performance workstations of just a few years
ago.

Factors which constitute this trend are (\cite{architecture}):
\begin{itemize}
\item a slowdown in uniprocessor performance arising from  diminishing
  returns in exploiting instruction-level-parallelism
\item a growth in servers and server performance
\item a growth in data-intensive applications
\item the insight that increasing performance on the desktop is less
  important
\item an improved understanding of how to use multiprocessors
  effectively, especially in server environments where there is
  significant thread-level parallelism
\item the advantages of leveraging a design investment by replication
  rather than unique design--all multiprocessor designs provide such leverage
\end{itemize}

With the need to solve large problems and the availability of adequate
workstations, large-scale parallel computing has become more and more
important. Some examples to illustrate this need in science and
research include:
\begin{itemize}
\item data mining for large data sets
\item algorithms for solving NP-complete problems
\item climate prediction
\item computations in biology, chemistry and physics
\item cryptography
\item astronomy
\end{itemize}

The R project for Statistical Computing and Graphics is an open-source
software environment available for different platforms \cite{R}. With the
development mentioned above there have some extensions (called
packages) for High Performance Computing come up.

In this chapter the fundamentals of parallel computing are
presented. It provides a general overview of the field of High
Performance Computing and the possibilities to use R in this area.


\section{Computer Architecture}
In this section computer architectures are briefly described to
understand how performance can be maximized when using parallel
computing. First one need to know how one can make an application fast
on a single processor and second how can data efficiently supplied to
these processors. The beginning of this chapter deals with a taxonomy of
the design alternatives for multiprocessors. After that a brief look
into processor design and memory hierarchies is given. The section closes with a description of high performance computing
servers.

\subsection{Beyond Flynn's Taxonomy}

Providing a high-level standard for programming HPC applications is a
major challenge nowadays. There are a variety of architectures for
large-scale computing. They all have specific features and 
therefore there should be a taxonomy in which such architectures can
be classified. About fourty years ago, \cite{flynn72sco} classified
architectures on the
presence of single or multiple either instructions or data streams
known as Flynn's taxonomy:
\begin{description}
\item[Single Instruction Single Data (SISD)]
  This type of architecture of CPUs (called uniprocessors),
  developed by the 
  mathematician John von Neumann, was the standard for a long
  time. These Computers are also known as serial computers. 
\item[Multiple Instruction Single Data (MISD)] the theoritical possibility
  of applying multiple instructions on a single datum is generally
  impractical.
\item[Single Instruction Multiple Data (SIMD)] 
  a single instruction is applied by multiple processors to different
  data in parallel (data-level parallelism). 
\item[Multiple Instruction Multiple Data (MIMD)] processors apply
  different instructions on different data (thread-level
  parallelism). See section \ref{sec:mimd_computers} for details. 
\end{description}

But these distinctions are insufficient for classifying modern
computers according to \cite{duncan90survey}. For example there are
pipelined vector processors capable of concurrent arithmetic execution and
manipulating hundreds of vector elements in parallel.

Therefore \cite{duncan90survey} defines that a parallel architecture
provides an explicit, high-level framework for 
the development of parallel programming solutions by providing
multiple processors that cooperate to solve problems through
concurrent execution:

\begin{description}
\item[Synchronos architectures] coordinate concurrent operations in
  lockstep through global clocks, central control unit, or vector unit
  controllers. These architectures involve pipelined vector processors
  (characterized by multiple, pipelined functional units, which
  implement arithmetic and Boolean operations), SIMD architectures
  (typically a control unit broadcasting a single instruction to
  all processors executing the instruction on local data) and Systolic
  architectures (pipelined multiprocessors in which data flows from
  memory through a network of processors back to memory synchronized
  by a global clock. 
\item[MIMD architectures] consist of multiple processors applying
  different instructions on local (different) data. The MIMD models
  are asynchronous computers although they may be synchronized by
  messages passing through an interconnection network (or by accessing
  data in a shared memory). The advantage is the possibility of
  executing largely independent subcalculations. These architectures
  can further be classified in distributed or shared memory systems
  whereas the first achieves interprocess communication via an
  interconnection network and the last via a global memory each
  processor can address.
\item[MIMD-based architectural paradigms] involve MIMD/SIMD hybrids,
  dataflow architectures, reduction machines, and wavefront arrays. A
  MIMD architecture can be called a MIMD/SIMD hybrid if parts of the
  MIMD architecture are controlled in SIMD fashion (some sort of
  master/slave relation). In dataflow architectures instructions are
  enabled for execution as soon as all of their operands become
  available whereas in reduction architectures an instruction is
  enabled for execution when its results are required as operands for
  another instruction already enabled for execution. Wavefront array
  processors are a mixture of systolic data piplining and asynchronous
  dataflow execution.
\end{description}

\subsection{Processor and memory}



\subsection{High performance computing server}

Traditionally computers can be categorized as follows:
\begin{itemize}
\item personal computers
\item workstations
\item mini computers
\item mainframe computers
\item High performance computing servers (super computers)
\end{itemize}

This has changed rapidly in the last years, as the requirements of
people went through an interesting process. Microprocessor-based
computers dominate the market. PCs and workstations have emerged as
major products in the computer industry. Servers based on
microprocessors have replaced minicomputers and mainframes have almost
been replaced with networks of multiprocessor workstations.

We are now interested in the last category namely high performance
computing servers, because these computers have the highest
computation power and they are mainly used
for scientific or technical applications. The number of processors
such a computer can have reaches from 100 to several thousand (mainly
64 bit processors).
You can divide this category of computers in ``vector computers'' or ``parallel
computers''. The amount of memory they have varies according to the type of
these servers or application, which is running on them. The maximum
amount of memory can absolutely be beyond several tera bytes.  

This development started in the 1990's where parallel programming has
become more and more important. With the increasing need of powerful
hardware computer vendors started to build supercomputers capable of
executing more and more instructions in parallel. In the beginning
vector supercomputing system using a single shared memory were widely
used to run large-scale applications. Due to the fact that vector
systems were rather expensive in both purchasing and operation and
bus-based shared-memory parallel systems were technologically bounded
to smaller CPU counts the first distributed-memory parallel platforms (DMPs)
came up. The big advantage was that they were inexpensive and could be
built in many different sizes. DMPs could be rather small like a
connection of workstations connected via a network to form a cluster of
workstations (COW) or arbitrarily large. In fact individual
organizations could buy clusters of a size that suited their
budgets. The big disadvantage DMPs have is the higher transportation
time of messages passing through a network in comparison to a shared
memory system. Although the technology has improved (Infiniband) there
is still space for improvements. 

Current HPC systems are still formed of vector supercomputers
(providing highest level of performance, but only for certain
applications), SMPs with two to over 100 hundred CPUs and DMPs. As
more than one CPU has become commonplace in desktop machines clusters
can nowadays provide both small SMPs and big DMPs. When it comes to a
decition which architecture to buy companies, laboratories or
universities often choose to use a cluster of workstations because the
cost of networking has decreased as I mentioned before and performance
has increased since systems of this kind already dominate the Top500
list. Furthermore, COWs are so successful because they can be
configured from commodity chips and network and built the open-source
Linux operating system and therefore offer high performance for a
really good price.

New developments show that there are interesting alternatives. Some
HPC platforms offer a global addressing scheme and therefore enables
CPUs to directly access the entire system's memory. This can be done
with ccNUMA (cache coherent nonuniform memory access) systems (cite hp
sgi paper and sgi.pdf-luthien) or
global arrays (see section \ref{globalArrays})
They have become an important improvement to distributed memory
platforms because amongst others it enables compiler driven
parallelizing techniques like OpenMP (see section \ref{OpenMP})to be
used on these systems. 

\section{MIMD computers}
\label{sec:mimd_computers}


\textbf{TODO: distributed and shared memeory systems}

heterogeneous network computing see geist et al p.2

\section{Parallel Programming Models}
\label{sec:programming_models}

Programmers who like to create an application which should run in
parallel have to face a challenging task. They have to distribute all
the computation involved to a large number of processors. It is
important that this is done in a way so that each of these computation
nodes performs roughly the same work. Furthermore, developers have to
ensure that the data required for the computation is available to the
processor with as little delay as possible. Therefore some sort of
coordination is required for the locality of data in the memory.
One might think that this could be hard work, and indeed it is. But
there are programming models for HPC already available which make life
easier.

When parallel computing came up there have been a lot of programming
languages developed. Only few of them experienced any real use. There
have also been many implementations of computer vendors for their own
machines. For a long time there has been no standard in sight as no
agreement between hardware vendors and developers emerged. It was
common that application developers have to write seperate HPC
applications for each architecture.

In the 1990's broad commonity efforts produced the first defacto
standards for parallel computing. Commonalities were identified
between all the implementations available and the field of parallel
computing had been understood better. Since then libraries and certain
implementations as well as compiler driven parallelizer have been
developed.

The programming models presented in this section are common and widely
used. Starting with the first generation the Message Passing Interface
(MPI - see \cite{forum94:MPI}) and Parallel Virtual Machine (PVM - see
\cite{geist94pvm})
are going to be described. The second generation of HPC programming
models involve OpenMP (see \cite{openMP05}) and Global Arrays (GA - see
\cite{nieplocha96gan}). These programming models are not the only ones
but are commonly used. 

\subsection{The Message Passing Interface (MPI)}
\label{sec:MPI}

In 1994 the Message Passing Interface Forum (MPIF) has defined a set
of library interface standards for message passing. Over 40
organizations participated in the discussion which started in
1993. The aim was to develop a widely standard for writing
message-passing programs. The standard was called the Message Passing
Interface (see \cite{forum94:MPI}).

As mentioned in section \ref{sec:programming_models} before there
were a lot of different programming models and languages for high
performance computing available. With MPI a standard was established
that should be practical, portable, efficient and flexible. After
releasing Version 1.0 in 1994 the MPIF continued to correct errors and
made clarifications in the MPI document so that in June 1995 Version
1.1 of MPI had been released.

Further corrections and clarifications have been made Version 1.2 of
MPI and with Version 2 completely new types of functionality has been
added (see \cite{forum94:MPI-2} for details).

MPI is now a library which can be called from various programming
languages like C or Fortran. It contains all the infrastructure for
inter process communication.

\textbf{TODO: Advantages/Disadvantages}

\subsubsection{Why using message-passing?}
\label{sec:why_m-p}

If one uses a distributed memory machine (DMP) as platform for
carrying out computations in parallel, data hs to be transmitted from
one computation node to the other. With MPI there are higher level
routines and abstractions available which build upon lower level
message-passing routines.
  
A message-passing function is a function which explicitly transmits
data from one process to another. With these
functions, creating parallel programs can be extremely
efficient. Developers don't have to care about low level
message-passing anymore.

\subsubsection{Implementations of the MPI standard}

\textbf{TODO: introduction to implementations}

\begin{description}
\item[LAM/MPI] is an open-source implementation of MPI. It runs on a
  network of UNIX machines connected via a local area network or via
  the Internet. On each node runs one UNIX daemon which is a
  micro-kernel plus a dozen system processes responsible for network
  communication among other things. The micro-kernel is
  responsible for the communication between local processes (see
  \cite{burns94lam}). The LAM/MPI implementation includes all of the
  MPI-1.2 but not everything of the MPI-2 standard. LAM/MPI is
  available at \url{http://www.lam-mpi.org/}.
\item[MPICH] is a freely available complete implementation of the MPI
  specification (MPICH1 includes all MPI-1.2 and MPICH2 all MPI-2
  specifications). The main goal was to deliver high performance with
  respect to portability. ``CH'' in MPICH stands for ``Chameleon''
  which means adaptability to one's environment and high performance
  as chameleons are fast (see \cite{gropp96mpich}). MPICH had been
  developed during the discussion process of the MPIF and had immediately been
  available to the community after the standard had been
  confirmed. There are many other implementaions of MPI based on MPICH
  available because of its high and easy portability. Sources and
  Windows binaries are available from
  \url{http://www-unix.mcs.anl.gov/mpi/mpich/}.
\item[Open MPI] has been build upon three MPI Implementations like
  LAM/MPI among the others. It is a new open-source MPI-2
  Implementation. The goal of Open MPI is to implement the full
  MPI-1.2 and MPI-2 specifications focusing on production-quality and
  performance (see \cite{gabriel04:_open_mpi}). Open MPI can be
  downloaded from  \url{http://www.open-mpi.org/}.
\end{description}

for setting up these implementations in Debian GNU Linux and cluster
jobscripts see Appendix \ref{app:mpi_imp}. 

\subsection{Parallel Virtual Machine (PVM)}
\label{sec:PVM}

The PVM system is another but own implementation of a functionally
complete message-passing model. It is designed to link computing
resources for example in a heterogenous network and to provide
developers with a parallel platform. As the message-passing model
seemed to be the paradigm of choice in the late 80's the PVM project
was created in 1989 to allow developers to exploit distributed
computing across a wide variety of computer types. A key concept in
PVM is that it makes a collection of computers appear as one large
\textit{virtual} machine, hence its name \cite{geist94pvm}. The PVM
system can be found on \url{http://www.csm.ornl.gov/pvm/}. In Appendix
\ref{app:mpi_imp} one can find on how to setup this system on a linux
machine and how to start a PVM job on a cluster.

\textbf{TODO: Advantages/Disadvantages}


\subsection{OpenMP}
\label{sec:OpenMP}

A completely other approach in the parallel programming model context
is the use of shared memory. Message-passing is built upon this shared
memory model that means every processor has direct access to the
memory of every other processor in the system. This class of
multiprocessor is also called Scalable Shared Memory Multiprocessor
(SSMP). Like the development of the MPI specifications the development
of OpenMP started for one simple reason: portability. Prior the
introduction of OpenMP as an industry standard every vendor of shared
memory systems has created its own proprietary extension for
developers. Therefore a lot of people interested in parallel
programming used portabel message-passing models like MPI (see
Section \ref{sec:MPI}) or PVM (see Section \ref{sec:PVM}). As there
was an increasing demand for a simple and scalable programming model and
the desire to begin parallelizing existing code without having to
completely rewrite it, the OpenMP Architecture Review Board released a
set of compiler directives and callable runtime library routines
called the OpenMP API (see \cite{openMP05}). This directives and routines extend the
programming languages Fortran and C or C++ respectively to express
shared memory parallelism.
Furthermore this programming model is based on the fork/join execution
model which makes it easy to get parallelism out of a sequential
program. Therefore unlike in message-passing the program or algorithm
need not to be completely decomposed for parallel execution. Given a
working sequential program it is not difficult to incrementally
parallelize individual loops and thereby realize a performance gain in
a multiprocessor system (\cite{dagum1997opi}). The specification and
further information can be found on the following website:
\url{http://www.openmp.org/}.

\textbf{TODO: Advantages/Disadvantages}

\subsection{Global Arrays}
\label{globalArrays}


\section{Performance Analysis}
\label{sec:perf_analysis}

\section{Implementation of parallel applications in R}
\label{sec:parallel_R}

\subsection{Example: Matrix multiplication}
\subsection{The Rmpi package}
\subsection{The Rpvm package}
\subsection{SNOW package}
\subsection{Building an OpenMP shared library}
\subsection{Comparison}

\section{Hardware used}

\textbf{TODO: own section? short description of machines, details in
  appendix}

The Hardware used for the applications presented in this thesis is
going to be described in this section.
At the Department of Statistics and Mathematics of the Vienna
University of Economics and Business Administration a high performance
computing server, a cluster of workstations, is
available. Cluster@WU is a shared Platform among several departments
of the university which are represented through a research institute
for computational methods. 
\begin{table}

\end{table}
R-Forge specs

2 x Dual Core AMD Opteron 2.4 GHz
12 GB RAM
1.2 TB RAID 5

Cluster specs (node.q, bignode, bignode.q)

4 Nodes (bignodes):
2 x Dual Core Intel XEON CPU 5140 @ 2.33 GHz
16 GB RAM

68 Nodes:
1 x Intel Core 2 Duo CPU 6600 @ 2.4 GHz
4 GB RAM


\section{Conclusion}

TODO: performance in a closer sense

      benchmarks

      LINPACK benchmark (implement LINPACK Java in R
      http://www.top500.org/about/linpack

      performance in a wider sense


    %% Chapter High Performance Computing and R
    \chapter{High Performance Computing and R}
\label{chap:Rhpc}
\section{Introduction}

This chapter provides an overview of the capabilities of R in the area
of high performance computing. I start with a short description of the
software package R. Subsequently extensions to the base environment
(packages) which provide high performance computation functionality to
R are going to be explained. Among these extensions there is the
package called ``parc'', which has been developed in the course of
this thesis.

\section{The Rmpi Package}
\label{sec:Rmpi}
The Message Passing Interface (MPI) is a set of library interface
standards for message passing and there are many implementations using
these standards (see also Section~\ref{sec:MPI}).
\pkg{Rmpi} is an interface to MPI (\cite{yu02Rmpi} and
\cite{yu06Rmpi}). As 
of the time of this writing \pkg{Rmpi} uses
the LAM implementation of MPI. For process spawning the standard
MPI-1.2 is required which is available in the LAM/MPI implementation 
as LAM/MPI (version 7.1.3) supports a large portions of the MPI-2
standard. This is necessary if one likes to use interactive spawning
of R processes. With MPI versions prior to MPI-1.2 separate R
processes have to be created by hand using \code{mpirun} (part of many
MPI implementations) for example.

\pkg{Rmpi} contains a lot of low level interface functions to the MPI
C-library. 
Furthermore, a handful of high level functions are supplied. A
selection of routines is going to be presented in this section arranged
into the following topics:

\begin{itemize}
\item Initialization and Status Queries
\item Process Spawning and Communication
\item Built-in High Level Functions
\item Other Important Functions
\end{itemize}  

A windows implementation of this package (which uses MPICH2)
is available from~\url{http://www.stats.uwo.ca/faculty/yu/Rmpi}.

\subsection{Initialization and Status Queries}

The LAM/MPI environment has to be booted prior to using any
message passing library functions. One possibility is to use the
command line, the other is to load the \pkg{Rmpi} package. It automatically
sets up a (small---1 host) LAM/MPI environment (if the executables are
in the search path). 

When using the Sun Grid Engine (SGE) or other queueing systems to boot
the LAM/MPI parallel environment the developer is not engaged with
setting up and booting the environment anymore (see
appendix \ref{app:gridengine} on how to do this). On a cluster of
workstations this is the method of choice. 

\subsubsection{Management and Query Functions}

\begin{description}
\item[\code{lamhosts()}] finds the hostname associated with its node
  number.
\item[\code{mpi.universe.size()}] returns the total number of CPUs
  available to the MPI environment (ie. in a cluster or in a parallel
  environment started by the grid engine).
\item[\code{mpi.is.master()}] returns TRUE if the process is the
  master process or FALSE otherwise. 
\item[\code{mpi.get.processor.name()}] returns the hostname where the
  process is executed.
\item[\code{mpi.finalize()}] cleans all MPI states (this is done when
  calling \code{mpi.exit} or \code{mpi.quit}.
\item[\code{mpi.exit()}] terminates the mpi communication
  environment and detaches the \pkg{Rmpi} package which makes reloading of
  the package \pkg{Rmpi} in the same session impossible.  
\item[\code{mpi.quit()}] terminates the mpi communication
  environment and quits R.  
\end{description}

Example~\ref{ex:Rmpi-init} shows how the configuration of the
parallel environment can be obtained. First it returns the hosts
connected to the parallel environment and then prints the number of
CPUs available in it. After a query if this process is the master
process, the hostname the current process runs on is
returned. 

\begin{Example} Queries to the MPI communication environment
\label{ex:Rmpi-init}
\begin{Schunk}
\begin{Sinput}
> library("Rmpi")
> lamhosts()
\end{Sinput}
\begin{Soutput}
node065 node065 node045 node045 node025 node025 node047 node047 
      0       1       2       3       4       5       6       7 
\end{Soutput}
\begin{Sinput}
> mpi.universe.size()
\end{Sinput}
\begin{Soutput}
[1] 8
\end{Soutput}
\begin{Sinput}
> mpi.is.master()
\end{Sinput}
\begin{Soutput}
[1] TRUE
\end{Soutput}
\begin{Sinput}
> mpi.get.processor.name()
\end{Sinput}
\begin{Soutput}
[1] "node065"
\end{Soutput}
\end{Schunk}
\end{Example}

\subsection{Process Spawning and Communication}

In \pkg{Rmpi} it is easy to spawn R slaves and use them as
workhorses. The 
communication between all the involved processes is carried out in a
so called communicator (comm). All processes within the same
communicator are able to send or receive messages from other
processes. The processes are identified through their commrank (see
also the fundamentals of message passing in
Section~\ref{sec:messagepassing}). The big advantage of \pkg{Rmpi}
slaves is, that they can be used interactively when using the default
R slave script.

\subsubsection{Process Management  Functions}
\begin{description}
\item[\code{mpi.spawn.Rslaves(Rscript =
    system.file(nslaves =
    mpi.universe.size(), ...)}] spawns \code{nslaves} number of R
  workhorses to those hosts automatically chosen by MPI. For other
  arguments represented by \ldots to this function we refer to
  \cite{yu06Rmpi}.
\item[\code{mpi.close.Rslaves(dellog = TRUE, comm = 1)}] closes
  previously spawned R slaves and returns 1 if succesful.
\item[\code{mpi.comm.size()}] returns the total number of members in
  a communicator.
\item[\code{mpi.comm.rank()}] returns the rank (identifier) of the
  process in a communicator.
\item[\code{mpi.remote.exec(cmd, ..., comm = 1, ret = TRUE)}]
  executes a command \code{cmd} on R slaves with \ldots arguments to
  \code{cmd} and returns executed results if \code{ret} is
  \code{TRUE}.
\end{description}

In Example~\ref{ex:Rmpi2} as many slaves are spawned as are available
in the parallel environment. The size of the communicator is returned
(1 master plus the spawned slaves) and a remote query of the commrank
is carried out. Before the slaves are closed the commrank of the
master is printed.

\begin{Example} Process management and communication 

\begin{Schunk}
\begin{Sinput}
> mpi.spawn.Rslaves(nslaves = mpi.universe.size())
\end{Sinput}
\begin{Soutput}
	8 slaves are spawned successfully. 0 failed.
master (rank 0, comm 1) of size 9 is running on: node065 
slave1 (rank 1, comm 1) of size 9 is running on: node065 
slave2 (rank 2, comm 1) of size 9 is running on: node065 
slave3 (rank 3, comm 1) of size 9 is running on: node045 
slave4 (rank 4, comm 1) of size 9 is running on: node045 
slave5 (rank 5, comm 1) of size 9 is running on: node025 
slave6 (rank 6, comm 1) of size 9 is running on: node025 
slave7 (rank 7, comm 1) of size 9 is running on: node047 
slave8 (rank 8, comm 1) of size 9 is running on: node047 
\end{Soutput}
\begin{Sinput}
> mpi.comm.size()
\end{Sinput}
\begin{Soutput}
[1] 9
\end{Soutput}
\begin{Sinput}
> mpi.remote.exec(mpi.comm.rank())
\end{Sinput}
\begin{Soutput}
  X1 X2 X3 X4 X5 X6 X7 X8
1  1  2  3  4  5  6  7  8
\end{Soutput}
\begin{Sinput}
> mpi.comm.rank()
\end{Sinput}
\begin{Soutput}
[1] 0
\end{Soutput}
\begin{Sinput}
> mpi.close.Rslaves()
\end{Sinput}
\begin{Soutput}
[1] 1
\end{Soutput}
\end{Schunk}
\label{ex:Rmpi2}
\end{Example}
 
\subsection{Built-in High Level Functions}

\pkg{Rmpi} provides many high level functions. We selected a few
of them which we think are commonly used. Most of them have been
utilized to build parallel programs presented in the subsequent
chapters.

\subsubsection{High Level Functions}
\begin{description}
\item[\code{mpi.apply(x, fun, ..., comm = 1)}] applies a function
  \code{fun} with additional arguments \ldots to a specific part of
  a vector \code{x}. The return value is of type list with the same
  length as of \code{x}. The length of
  \code{x} must not exceed the 
  number of R slaves spawned as each element of the vector is used
  exactly by one slave. To achieve some sort of load balancing please
  use the corresponding apply functions below.
\item[\code{mpi.applyLB(x, fun, ..., comm = 1)}] applies a function
  \code{fun} with additional arguments \ldots to a specific part of
  a vector \code{x}. There are a few more variants explained in
  \cite{yu06Rmpi}.
\item[\code{mpi.bcast.cmd(cmd = NULL, rank = 0, comm = 1)}]
  broadcasts a command \code{cmd} from the sender \code{rank} to
  all R slaves and evaluates it.
\item[\code{mpi.bcast.Robj(obj, rank = 0,comm = 1)}]
  broadcasts an R object \code{obj} from process rank \code{rank}
  to all other processes (master and slaves).
\item[\code{mpi.bcast.Robj2slave(obj, comm = 1)}] broadcasts an R
  object \code{obj} to all R slaves from the master process. 
\item[\code{mpi.parSim( ... )}] carries out a Monte Carlo simulation
  in parallel. For details on this function see the package manual
  (\cite{yu06Rmpi}) and on Monte Carlo simulation
  the applications in chapter \ref{chap:options}.
\end{description}

How to use the high level function \code{mpi.apply()} is shown in
Example~\ref{ex:Rmpi3}. A vector of $n$ random numbers is generated on
each of the $n$ slaves and are returned to the master as a
list (each list element representing one row). Finally a $n  \times n$
matrix is formed and printed. The output of the matrix shows for each
row the same random numbers. This is because of the fact, that each
slave has the same seed. This problem is more specific treated in
Chapter~\ref{chap:options}. For more information about parallel random
number generators see the descriptions of the packages \pkg{rsprng}
and \pkg{rlecuyer} in Section~\ref{sec:otherpackages}.

\begin{Example} Using mpi.apply
\label{ex:Rmpi3}
\begin{Schunk}
\begin{Sinput}
> n <- 8
> mpi.spawn.Rslaves(nslaves = n)
\end{Sinput}
\begin{Soutput}
	8 slaves are spawned successfully. 0 failed.
master (rank 0, comm 1) of size 9 is running on: node065 
slave1 (rank 1, comm 1) of size 9 is running on: node065 
slave2 (rank 2, comm 1) of size 9 is running on: node065 
slave3 (rank 3, comm 1) of size 9 is running on: node045 
slave4 (rank 4, comm 1) of size 9 is running on: node045 
slave5 (rank 5, comm 1) of size 9 is running on: node025 
slave6 (rank 6, comm 1) of size 9 is running on: node025 
slave7 (rank 7, comm 1) of size 9 is running on: node047 
slave8 (rank 8, comm 1) of size 9 is running on: node047 
\end{Soutput}
\begin{Sinput}
> x <- rep(n, n)
> rows <- mpi.apply(x, runif)
> X <- matrix(unlist(rows), ncol = n, byrow = TRUE)
> round(X, 3)
\end{Sinput}
\begin{Soutput}
     [,1]  [,2]  [,3]  [,4]  [,5]  [,6]  [,7]  [,8]
[1,]  0.5 0.976 0.347 0.266 0.471 0.444 0.069 0.955
[2,]  0.5 0.976 0.347 0.266 0.471 0.444 0.069 0.955
[3,]  0.5 0.976 0.347 0.266 0.471 0.444 0.069 0.955
[4,]  0.5 0.976 0.347 0.266 0.471 0.444 0.069 0.955
[5,]  0.5 0.976 0.347 0.266 0.471 0.444 0.069 0.955
[6,]  0.5 0.976 0.347 0.266 0.471 0.444 0.069 0.955
[7,]  0.5 0.976 0.347 0.266 0.471 0.444 0.069 0.955
[8,]  0.5 0.976 0.347 0.266 0.471 0.444 0.069 0.955
\end{Soutput}
\begin{Sinput}
> mpi.close.Rslaves()
\end{Sinput}
\begin{Soutput}
[1] 1
\end{Soutput}
\end{Schunk}
\end{Example}

\subsection{Other Important Functions}

To complete the set of important functions supplied by the \pkg{Rmpi}
package the following functions have to be explained.

\subsubsection{Collective Communication Routines}
\begin{description}
\item[\code{.mpi.gather(x, type, rdata, root = 0, comm = 1)}] gathers
  data distributed on the nodes (\code{x}) to a 
  specific process (mostly the master) into a single array or list
  (depending on the length of the data) of type \code{type} which can
  be integer, double or character. It
  performs  
  a send of messages from each member in a comm. A
  specific process (\code{root}) accumulates this messages into a 
  single array or list prepared with the \code{rdata} command.
\item[\code{.mpi.scatter(x, type, rdata, root = 0, comm = 1)}]
  sends to each member of a comm a partition of  a vector \code{x},
  type \code{type} which can be integer, double or character,
  from a specified member of the group (mostly the master). Each
  member of the comm receive its part of \code{x} after preparing the
  receive buffer with the argument \code{rdata}.
\end{description}

\subsection{Conclusion}

The package \pkg{Rmpi} implements many of the routines available in
MPI-2. But there are some that have to be ommitted or are not included
because they are not needed for use in R (e.g., data management
routines are not necessary as R has its own tools for data handling).
A really interesting aspect of the \pkg{Rmpi} package is the
possibilty to spawn interactive R slaves. That enables the user to
interactively 
define functions which can be executed remotely on the slaves in
parallel. An example how one can do this is shown in
Chapter~\ref{chap:matrix}, where the implementation of matrix
multiplication is shown. A major disadvantage is that MPI in its
current implementations lack in fault tolerance. This results in a
rather instable execution of MPI applications. Moreover, debugging
is really difficult as there is no support for it in \pkg{Rmpi}.

All in all this package is a good start in creating parallel programs
as this can easy be achieved entirely in R.

For further interface functions supplied by the \pkg{Rmpi} package, a
more detailed description and further examples please consult the
package description \cite{yu06Rmpi}.


\section{The rpvm Package}
\label{sec:rpvm}
Parallel virtual machine uses the message passing model and makes a
collection of computers appear as a single virtual machine (see Section
\ref{sec:PVM} for details).
The package \pkg{rpvm} (\cite{nali07rpvm}) provides an interface to
low level PVM functions and
a few high level parallel functions to R. It uses most of the
facilities provided by the PVM system which makes \pkg{rpvm} ideal for
prototyping parallel statistical applications in R.

Generally, parallel applications can be either written in compiled
languages like C or 
FORTRAN or can be called as R processes. The latter method is used in
this thesis and therefore a good selection of \pkg{rpvm}
functions are explained in this section to show how PVM and R can be
used together. 
%Provided functions are categorized as follows:

%\begin{itemize}
%\item Initialization and Status Queries
%\item Process Spawning and Communication
%\item Built-in High Level Functions
%\item Other Important Functions
%\end{itemize}  

\subsection{Initialization and Status Queries}

At first for using the PVM system \textit{pvmd3} has to be
booted. This can be done via the command line using the \code{pvm}
command (see pages 22 and 23 in \cite{geist94pvm}) or directly within
R after loading the \pkg{rpvm} package using \code{.PVM.start.pvmd()}
explained in this section.

\subsubsection{Functions for Managing the Virtual Machine}
\begin{description}
\item[\code{.PVM.start.pvmd()}] boots the \textit{pvmd3} daemon. The
  currently running R session becomes the master process. 
\item[\code{.PVM.add.hosts(hosts)}] takes a vector of hostnames to
  be added to the current virtual machine. The syntax of the
  hostnames is similar to the lines of a pvmd hostfile (for details
  see the man page of \textit{pvmd3}). 
\item[\code{.PVM.del.hosts()}] simply deletes the given hosts from
  the virtual machine configuration.
\item[\code{.PVM.config()}] returns information about the present
  virtual machine.
\item[\code{.PVM.exit()}] tells the PVM daemon that this process
  leaves the parallel environment.
\item[\code{.PVM.halt()}] shuts down the entire PVM system and exits
  the current R session.
\end{description}
  
When using a job queueing system like the Sun Grid Engine (SGE) to boot
the PVM parallel environment the developer is not engaged with
setting up and booting the environment anymore (see
appendix \ref{app:gridengine} on how to do this).

Example~\ref{ex:rpvm-init} shows how the configuration of the parallel
environment can be obtained. \code{.PVM.config()} returns the hosts
connected to 
the parallel virtual machine. After that the
parallel environment is stopped. 
\begin{Example} Query status of PVM 
\label{ex:rpvm-init}
\begin{Schunk}
\begin{Sinput}
> library("rpvm")
> set.seed(1782)
> .PVM.config()
\end{Sinput}
\begin{Soutput}
  host.id    name    arch speed
1  262144 node066 LINUX64  1000
2  524288 node020 LINUX64  1000
3  786432 node036 LINUX64  1000
4 1048576 node016 LINUX64  1000
\end{Soutput}
\begin{Sinput}
> .PVM.exit()
\end{Sinput}
\end{Schunk}
\end{Example}

\subsection{Process Spawning and Communication}

The package \pkg{rpvm} uses the master-slave paradigm meaning that one
process is the master task and the others are slave tasks. \pkg{rpvm}
provides a routine to spawn R slaves but these slaves cannot be used
interactively like the slaves in \pkg{Rmpi}. The spawned R slaves
source an R script which contains all the necessary function calls to
set up communication and carry out the computation and after
processing terminate.
PVM uses task IDs (\code{tid}---a positive integer for identifying a task)
and tags for communication (see
also the fundamentals of message passing in
Section~\ref{sec:messagepassing}).

\subsubsection{Process Management  Functions}
\begin{description}
\item[\code{.PVM.spawnR(slave, ntask = 1, ...)}] spawns \code{ntask}
  copies of an \code{slave} R process. \code{slave} is a character
  specifying the source file for the R slaves located in the package's
  demo directory (the default). There are more
  parameters indicated by the \ldots (we refer to
  \cite{nali07rpvm}). The \code{tids} of the successfully spawned R
  slaves are returned.
\item[\code{.PVM.mytid()}] returns the \code{tid} of the calling
  process.
\item[\code{.PVM.parent()}] returns the \code{tid} of the parent
  process that spawned the calling process.
\item[\code{.PVM.siblings()}] returns the \code{tid} of the processes
  that were spawned in a single spawn call.
\item[\code{.PVM.pstats(tids)}] returns the status of the PVM
  process(es) with task ID(s) \code{tids}.
\end{description}

\subsection{Built-in High Level Functions}

\pkg{rpvm} provides only two high level functions. One of them is a
function to get or set values in the virtual machine settings. This is
certainly because each new high level functions needs separate source
files for the slaves and this is not what developers do intuitively. 

\subsubsection{High Level Functions}
\begin{description}
\item[\code{PVM.rapply(X, FUN = mean, NTASK = 1))}] Apply a function
  \code{FUN} to the rows of a matrix \code{X} in
  parallel using \code{NTASK} tasks.
\item[\code{PVM.options(option, value)}] Get or set values of libpvm
  options (for details see \cite{nali07rpvm} and \cite{geist94pvm}).
\end{description}

Example~\ref{ex:rpvm-rapply} shows how the rows of a matrix \code{X}
can be summed up in parallel via \code{PVM.rapply()}.

\begin{Example} Using PVM.rapply
\begin{Schunk}
\begin{Sinput}
> n <- 8
> X <- matrix(rnorm(n * n), nrow = n)
> round(X, 3)
\end{Sinput}
\begin{Soutput}
       [,1]   [,2]   [,3]   [,4]   [,5]   [,6]   [,7]   [,8]
[1,] -0.200 -0.183  0.560  1.286  0.468  0.502  0.874 -0.778
[2,] -1.371  0.484 -0.498  1.788  0.534 -0.566  0.152 -1.307
[3,]  1.041  0.484  0.399  0.580  0.586 -0.660  1.833 -1.405
[4,] -1.117 -0.893  0.408 -1.612  0.486  0.644  0.422 -1.639
[5,]  1.397 -0.237 -1.287 -0.122 -1.076  0.225 -0.047  0.020
[6,] -0.046  0.537 -1.287 -0.089  0.564  2.671 -0.715 -0.901
[7,]  1.085  0.706 -0.034  0.929  0.057 -2.402 -1.233  1.135
[8,]  0.605 -0.076 -0.554  1.385 -0.436  0.249  0.338  1.369
\end{Soutput}
\begin{Sinput}
> PVM.rapply(X, sum, 3)
\end{Sinput}
\begin{Soutput}
Try to spawn tasks...
Work sent to  524289 
Work sent to  786433 
Work sent to  1048577 
[1]  2.5291297 -0.7840539  2.8572559 -3.3002539 -1.1275728
[6]  0.7329710  0.2431164  2.8806456
\end{Soutput}
\end{Schunk}
\label{ex:rpvm-rapply}
\end{Example}

Before we explain how \code{PVM.rapply} works the following \pkg{rpvm}
functions have to be explained:

\subsubsection{Other Important functions}
\begin{description}
\item[\code{.PVM.initsend()}] clears the default send buffer and
  prepares it for packing a new message.
\item[\code{.PVM.pkstr(data = "")}] and \code{.PVM.pkint(data =
    0, stride = 1)} are low level correspondents of the PVM packing
    routines (see \cite{geist94pvm} for more information on packing
    data). 
\item[\code{.PVM.pkdblmat(data)}] packs a double matrix including
  the dimension information. There are more packing routines
  available. They are explained in \cite{nali07rpvm}.
\item[\code{.PVM.send(tid, msgtag)}] sends the message stored in the
  active buffer to the PVM process identified by \texttt{tid}. The
  content is labeled by the identifier \texttt{msgtag}.
\item[\code{.PVM.recv(tid = -1, msgtag = -1)}] blocks the process
  until a message with label \texttt{msgtag} has arrived from
  \texttt{tid}. $-1$ means any. The receive buffer is cleared
  and the received message is placed there instead.
\item[\code{.PVM.upkstr(), .PVM.upkint(), .PVM.upkdblvec()}] and
  others are the corresponding unpack functions to the
  pack functions explained before.
\item[\texttt{.PVM.gather(x, count = length(x), msgtag, group,
    rootginst = 0)}] gathers data distributed on the nodes (\code{x})
  to a 
  specific process (mostly the root) into a single array. It performs
  a send of messages from each member of a group of processes. A
  specific process (the root) accumulates this messages into a single
  vector. 
\item[\texttt{.PVM.scatter(x, count, msgtag, group, rootqinst = 0)}]
  sends to each member of a group a partition of  a vector \code{x}
  from a 
  specified member of the group (mostly the root) where \texttt{count}
  is an integer specifying the number of elements to be sent to each
  member. 
\end{description}

Let us now examine the function to see how parallel programs using
\pkg{rpvm} can be written.

\begin{Example} PVM.rapply master routine
\begin{Scode}

PVM.rapply <- function (X, FUN = mean, NTASK = 1)
{
    WORKTAG <- 22
    RESULTAG <- 33
    if (!is.matrix(X)) {
        stop("X must be a matrix!")
    }
    if (NTASK == 0) {
        return(apply(X, 1, FUN))
    }
    end <- nrow(X)
    chunk <- end%/%NTASK + 1
    start <- 1
    mytid <- .PVM.mytid()
    children <- .PVM.spawnR(ntask = NTASK, slave = "slapply")
    if (all(children < 0)) {
        cat("Failed to spawn any task: ", children, "\n")
        .PVM.exit()
    }
    else if (any(children < 0)) {
        cat("Failed to spawn some tasks.  Successfully spawned ",
            sum(children > 0), "tasks\n")
        children <- children[children > 0]
    }
    for (id in 1:length(children)) {
        .PVM.initsend()
        range <- c(start, ifelse((start + chunk - 1) > end, end,
            start + chunk - 1))
        work <- X[(range[1]):(range[2]), , drop = FALSE]
        start <- start + chunk
        .PVM.pkstr(deparse(substitute(FUN)))
        .PVM.pkint(id)
        .PVM.pkdblmat(work)
        .PVM.send(children[id], WORKTAG)
        cat("Work sent to ", children[id], "\n")
    }
    partial.results <- list()
    for (child in children) {
        .PVM.recv(-1, RESULTAG)
        order <- .PVM.upkint()
        partial.results[[order]] <- .PVM.upkdblvec()
    }
    .PVM.exit()
    return(unlist(partial.results))
}
\end{Scode} 
\label{ex:rpvm-rapplymaster}
\end{Example}
The corresponding slave R script (slapply.R) looks as follows:

\begin{Example} PVM.rapply slave routine
\begin{Scode}

library (rpvm)
WORKTAG <- 22
RESULTAG <- 33
mytid  <- .PVM.mytid ()
myparent  <- .PVM.parent ()
## Receive work from parent (a matrix)
buf <- .PVM.recv (myparent, WORKTAG)
## Function to apply
func  <- .PVM.upkstr ()
cat ("Function to apply: ", func, "\n")
## Order
order <- .PVM.upkint ()
partial.work <- .PVM.upkdblmat ()
print (partial.work)
## actually work, take the mean of the rows
partial.result <- apply (partial.work, 1, func)
print (partial.result)
## Send result back
.PVM.initsend ()
.PVM.pkint (order)
.PVM.pkdblvec (partial.result)
.PVM.send (myparent, RESULTAG)
## Exit PVM
.PVM.exit ()
## Exit R
q (save="no")
\end{Scode}
\label{ex:rpvm-rapplyslave}
\end{Example}


Example~\ref{ex:rpvm-rapplymaster} shows the
master routine of \code{PVM.rapply()}. This function takes a matrix,
the function which is going to be applied and the number of processors
to use as arguments. At first 
the message tags are specified. These tags are necessary to uniquely
identify messages sent in a message passing environment. After
input validation \code{NTASK} child processes are spawned using the
\code{.PVM.spawnR()} command. After initializing the send buffer the
partitioned data (packed in the buffer using the \code{.PVM.pk*}
commands) is send to the corresponding child processes represented by
their task IDs using
\texttt{.PVM.send()}. PVM uses these task identifiers (tid) to
address pvmds, tasks, and groups of tasks within a virtual
machine.

Meanwhile the spawned slave processes (see
Example~\ref{ex:rpvm-rapplyslave}) have been idle because they wait
for input (\code{.PVM.receive()} is a blocking command). After
receiving data from the parent the data gets unpacked. Now the slaves
apply the given function to their part of the matrix. Finally another
send is initialized to provide the results to the parent process and
the slaves are detached from the virtual machine by calling a
\texttt{.PVM.exit()}. 

\subsection{Conclusion}

The \code{PVM.rapply()} example shown in this section followed the
Single Program Multiple Data (SPMD) paradigm. Data is split into
different parts which are sent to different processes. I/O is handled
solely by a master
process. When loading \pkg{rpvm} in an R session this session becomes the
master process. Slaves can easily be spawned provided that there are
working slave scripts available.

We encountered no problems when using the routines in \pkg{rpvm}. This
package seem to be rather stable in contrast to \pkg{Rmpi}, where for
unknown reasons the MPI environment sometimes crashed.

A major disadvantage is that the \pkg{rpvm}
package only has two higher level function. One of them can be used
for calculations. That means when using this package for parallel
computing one has 
to deal with low level message passing but which in turn may provide
higher flexibility. New parallel functions can be constructed on the
basis of the provided interface. The \code{PVM.rapply} code
can be taken as a template for further routines. 

Another disadvantage is the missing support for interactive R
slaves. Parallel tasks have to be created on the basis of separate
slave source files which are sourced on the creation of the slaves.

For further interface functions supplied by the \pkg{rpvm} package, a more
detailed 
description and further examples please consult the package description
\cite{nali07rpvm}.


\section{The snow Package}
\label{sec:snow}
The aim of simple network of workstations
(\pkg{snow}---\cite{rossini03snow}, \cite{tierney07snow}) is to
provide a simple parallel computing environment in R. To make a
collection of computers to appear as a virtual cluster in R \pkg{snow}
three different message passing environments can be used:

\begin{itemize}
\item PVM via R package rpvm (see section \ref{sec:rpvm})
\item MPI via R package Rmpi (see section \ref{sec:Rmpi})
\item SOCK via TCP sockets
\end{itemize}

The details of which mechanism is used and how it is used are hidden
from the high level user.
After setting up this virtual cluster developing parallel R functions
can be achieved via an standardized interface to the computation
nodes.
Moreover, when using \pkg{snow} one can rely on a good handful of
high level functions. This makes it rather easy to use the underlying
parallel computational engine.
Indeed \pkg{snow} uses existing interfaces to R namely \pkg{Rmpi} when
using MPI (see Section~\ref{sec:Rmpi}), \pkg{rpvm} when using PVM (see
Section~\ref{sec:rpvm}) and a new possibilty of message passing namely
TCP sockets, which is a rather simple way of achieving communication
between nodes (in most application this is not the optimal way).
What follows is a description of high level functions supplied by the
\pkg{snow} package. They are assigned to one of the topics:

\begin{itemize}
\item Initialization
\item Built-in High Level Functions
\item Fault Tolerance
\end{itemize}

\subsection{Initialization}

Initializing a \pkg{snow} cluster is rather easy if the system is
prepared accordingly. When using MPI (achieved through \pkg{Rmpi}) a
LAM/MPI environment has to be booted prior starting the virtual
cluster (see section \ref{sec:Rmpi}). Is PVM the method of choice the
\pkg{rpvm} package must be available and an appropriate PVM has to be
started (see section \ref{sec:rpvm}). For both MPI and PVM the
parallel environment 
can be configured through a grid engine (see appendix
\ref{app:gridengine}). TCP sockets can be set up directly using the
package. MPI or PVM have the possibility to query the status of the
parallel environment. This can be done using the functions supplied
from the corresponding package.


\pkg{snow} management functions:

\begin{description}
\item[\code{makecluster(spec, type = getClusterOption(``type''))}]
  starts a cluster of type \code{type} with \code{spec} numbers of
  slaves. If the cluster is of connection type SOCK then \code{spec}
  must be a charactor vector containing the hostnames of the
  slavenodes to join the cluster. The return value is a list
  containing the cluster specifications. This object is necessary in
  further function calls.
\item[\code{stopCluster(cl)}] stops a cluster specified in \code{cl}.
\end{description}


\begin{Example} Start/stop cluster in \pkg{snow} \newline
running on cluster@WU using the node.q -- the parallel environment was
started with the SGE using 8 nodes.

\begin{Schunk}
\begin{Sinput}
> library("snow")
> set.seed(1782)
> n <- 8
> cl <- makeCluster(n, type = "MPI")
\end{Sinput}
\begin{Soutput}
	8 slaves are spawned successfully. 0 failed.
\end{Soutput}
\begin{Sinput}
> stopCluster(cl)
\end{Sinput}
\begin{Soutput}
[1] 1
\end{Soutput}
\end{Schunk}
\label{ex:snowstartstop}
\end{Example}

\subsection{Built-in High Level Functions}

\pkg{snow} provides a good handful of high-level functions: 


%% TODO
\begin{description}
\item[\code{clusterEvalQ(cl, expr)}] evaluates an R expression
  \code{expr} on
  each cluster node provided by \code{cl}. 
\item[\code{clusterCall(cl, fun, ...)}] calls a function
  \code{fun} with arguments \ldots on each node found in \code{cl}
  and returns a list of the results.
\item[\code{clusterApply(cl, x, fun, ...)}] applies a function
  \code{fun} with additional arguments \ldots to a specific part of
  a vector \code{x}. The return value is of type list with the same
  length as of \code{x}. The length of
  \code{x} must not exceed the 
  number of R slaves spawned as each element of the vector is used
  exactly by one slave. To achieve some sort of load balancing please
  use the corresponding apply functions below.
\item[\code{clusterApplyLB(cl, x, fun, ...)}] is a load balancing
  version of \code{clusterApply()} which applies a function
  \code{fun} with additional arguments \ldots to a specific part of
  a vector \code{x} with the difference that the length of
  \code{x} can exceed the number of cluster nodes. If a node
  finished with the computation the next job is placed on the
  available node. This is repeated until all jobs have completed.
\item[\code{clusterExport(cl, list)}] broadcasts a list of global
  variables on the master (\code{list}) to all slaves.
\item[\code{parApply(cl, x, fun, ...)}] is one of the parallel
  versions of the \code{apply} functions available in R. We refer to
  the package documentation (\cite{tierney07snow}) for further details.
\item[\code{parMM(cl, A,B)}] is a simple parallel implementation of
  matrix multiplication. 
\end{description}


\begin{Example} Using high-level functions of snow\newline
running on cluster@WU using the node.q -- the parallel environment was
started with SGE using 8 nodes

\begin{Schunk}
\begin{Sinput}
> n <- 8
> cl <- makeCluster(n, type = "MPI")
\end{Sinput}
\begin{Soutput}
	8 slaves are spawned successfully. 0 failed.
\end{Soutput}
\begin{Sinput}
> x <- rep(n, n)
> rows <- clusterApply(cl, x, runif)
> X <- matrix(unlist(rows), ncol = n, byrow = TRUE)
> X
\end{Sinput}
\begin{Soutput}
           [,1]      [,2]     [,3]      [,4]      [,5]      [,6]       [,7]
[1,] 0.03582783 0.3065854 0.228593 0.2512860 0.1055006 0.1938628 0.05882335
[2,] 0.03582783 0.3065854 0.228593 0.2512860 0.1055006 0.1938628 0.05882335
[3,] 0.03582783 0.3065854 0.228593 0.2512860 0.1055006 0.1938628 0.05882335
[4,] 0.03582783 0.3065854 0.228593 0.2512860 0.1055006 0.1938628 0.05882335
[5,] 0.03582783 0.3065854 0.228593 0.2512860 0.1055006 0.1938628 0.05882335
[6,] 0.03582783 0.3065854 0.228593 0.2512860 0.1055006 0.1938628 0.05882335
[7,] 0.03582783 0.3065854 0.228593 0.2512860 0.1055006 0.1938628 0.05882335
[8,] 0.03582783 0.3065854 0.228593 0.2512860 0.1055006 0.1938628 0.05882335
          [,8]
[1,] 0.3884308
[2,] 0.3884308
[3,] 0.3884308
[4,] 0.3884308
[5,] 0.3884308
[6,] 0.3884308
[7,] 0.3884308
[8,] 0.3884308
\end{Soutput}
\end{Schunk}
\label{ex:snowapply}
\end{Example}

\subsection{Fault Tolerance}



\subsection{Conclusion}

The routines available in package \pkg{snow} are easy to understand
and use, provided that there is a corresponding communication
environment set up. Generally, the user need not know the underlying
parallel infrastructure, she just ports her sequential code so that it
uses the functions supplied by \pkg{snow}. All in all as the title
suggests simple network of workstations is simple to get started with
and is simple with respect to the possibilities of parallel
computations. 

For further interface functions supplied by the \pkg{snow} package, a
more detailed 
description and further examples please consult the package description
\cite{tierney07snow}.


\section{paRc -- PARallel Computations in R}

\section{Conclusion}



    %% Chapter Parallel Matrix Multiplication 
    \chapter{Parallel Matrix Multiplication}
\label{chap:matrix}
\section{Introduction}

If we think of applications in parallel computing matrix
multiplication comes into mind. Because of its nature it is prime
example for data parallelism. There are many algorithms for
parallelizing matrix multiplication available.


In this chapter a short introduction to matrix multiplication is
given. Subsequently implementations of a selection of parallel
algorithms are presented. Eventually results of the comparison are
presented.


\section{Notation}

$ \mathbb{R} $ denotes the set of real numbers and $ \mathbb{R}^{m
  \times n} $ the vector space of all m-by-n real matrices.

$$ A \in \mathbb{R}^{m \times n} \Longleftrightarrow A = (a_{ij}) = 
\left( \begin{array}{ccc}
a_{11} & \ldots & a_{1n} \\
\vdots &        & \vdots \\
a_{m1} & \ldots & a_{mn}
\end{array} \right)
a_{ij} \in \mathbb{R}
 $$

The lower case letter of the letter which denotes the matrix with
subsripts $ij$ refers to the entry in the matrix. 

\subsubsection{Column and Row Partitioning}

A matrix $A$ can be accessed through its rows as it is a stack of row
vectors:
$$ A \in \mathbb{R}^{m \times n} \Longleftrightarrow A = 
\left( \begin{array}{c}
a_{1}^T \\
\vdots \\
a_{m}^T 
\end{array} \right)
a_{k} \in \mathbb{R}^n 
 $$

This is called a \textit{row partition} of $A$.


If $A \in \mathbb{R}^{m \times n}$ the $k$th row of $A$ can be notated
as $A(k,)$ (according to the row access in R). I.e.,

$$ A(k,) = \left( \begin{array}{ccc}
a_{k1}, & \ldots, & a_{kn}
\end{array} \right)
$$


The other alternative is to see a matrix as a collection of column
vectors:

$$ A \in \mathbb{R}^{m \times n} \Longleftrightarrow A = 
\left( \begin{array}{ccc}
a_{1}, & \ldots, & a_{n}
\end{array} \right)
a_{k} \in \mathbb{R}^m 
 $$

This is called a \textit{column partition} of $A$.


If $A \in \mathbb{R}^{m \times n}$ the $k$th column of $A$ can be notated
as $A(,k)$ (according to the row access in R). I.e.,

$$ A(,k) = \left( \begin{array}{c}
a_{1k} \\
\vdots \\
a_{mk} 
\end{array} \right)
 $$

\subsubsection{Block Notation}

Block matrices are central in many algorithms. They have become very
important in high performance computing because it enables
easy distributing of data. 

In general a $m$ by $n$ matrix $A$ can be partitioned to obtain

$$ 
A = \left( \begin{array}{ccc}
A_{11} & \ldots & A_{1q} \\
\vdots &        & \vdots \\
A_{p1} & \ldots & A_{pq}
\end{array} \right)
$$

$A_{ij}$ is the $(i,j)$ block or submatrix with dimensions $m_i$ by
$n_j$ of $A$. $\sum_{i=1}^p m_i = m$ and $\sum_{j=1}^q n_j = n$. 
We can say that $A = A_{ij}$ is a $p$ by $q$ block
matrix.   

Now we can see that column and row partitionings are special cases of
block matrices.

\section{Naive Parallel Algorithm}

\subsection{Basic Multiplication Algorithm}

In the usual matrix multiplication procedure the array $C$ is computed
through dot products one at a time from left to right and top to
bottom order.

 ($\mathbb{R}^{m \times r} \times
\mathbb{R}^{r \times n} \to \mathbb{R}^{m \times n}$)

$$ C = AB \Longrightarrow c_{ij} = \sum_{k=1}^r a_{ik}b_{kj} $$


%\subsubsection{Gaxpy Matrix Multiply}

%\subsubsection{Outer Product MAtrix Multiply}

\section{Implementation}

\subsection{Naive Parallel Algorithm}




\subsubsection{MPI}

\begin{Scode}
mm.Rmpi <- function(X, Y, n_cpu = 1, spawnRslaves=FALSE) {
  dx <- dim(X) ## dimensions of matrix A
  dy <- dim(Y) ## dimensions of matrix B
  ## Input validation
  matrix.mult.validate(X, Y, dx, dy)
  
  if( n_cpu == 1 )
    return(X%*%Y)
  if( spawnRslaves == TRUE )
    mpi.spawn.Rslaves(nslaves = n_cpu)

  ## broadcast data and functions necessary on slaves
  mpi.bcast.Robj2slave(Y) 
  mpi.bcast.Robj2slave(X) 
  mpi.bcast.Robj2slave(n_cpu)
  
  nrows_on_slaves <- ceiling(dx[1]/n_cpu)
  nrows_on_last <- dx[1] - (n_cpu - 1)*nrows_on_slaves
  mpi.bcast.Robj2slave(nrows_on_slaves)
  mpi.bcast.Robj2slave(nrows_on_last)
  mpi.bcast.Robj2slave(mm.Rmpi.slave)

  ## start partial matrix multiplication on slaves
  mpi.bcast.cmd(mm.Rmpi.slave())

  ## gather partial results from slaves
  local_mm <- NULL
  mm <- mpi.gather.Robj(local_mm, root=0, comm=1)
  out <- NULL

  ## Rmpi returns a list when the vectors have different length (local_mm = NULL)
  for(i in 1:n_cpu)
    out <- rbind(out,mm[[i+1]])
  
  if( spawnRslaves == TRUE )
    mpi.close.Rslaves()
  out

\end{Scode}

\begin{Scode}
mm.Rmpi.slave <- function(){
  commrank <- mpi.comm.rank() -1
  if(commrank==(n_cpu - 1))
    local_mm <- X[(nrows_on_slaves*commrank + 1):(nrows_on_slaves*commrank + nrows_on_last),]%*%Y
  else
    local_mm <- X[(nrows_on_slaves*commrank + 1):(nrows_on_slaves*commrank + nrows_on_slaves),]%*%Y
  mpi.gather.Robj(local_mm,root=0,comm=1)    
}
\end{Scode}

\subsubsection{PVM}
\subsubsection{OpenMP}

\begin{Scode}
void OMP_matrix_mult( double *x, int *nrx, int *ncx,
		      double *y, int *nry, int *ncy,
		      double *z) {
  int i, j, k;
  double tmp, sum;

#pragma omp parallel for private(sum) shared(x, y, z, j, k, nrx, nry, ncy, ncx)
  for(i = 0; i < *nrx; i++)
    for(j = 0; j < *ncy; j++){
      sum = 0.0;
      for(k = 0; k < *ncx; k++) 
	sum += x[i + k**nrx]*y[k + j**nry];
      z[i + j**nrx] = sum;
    }
}
\end{Scode}

\subsection{Fox or other}

\subsubsection{MPI}
\subsubsection{PVM}
\subsubsection{OpenMP}

%% Results
\input{section_mm_results.tex}


    %% Chapter GRASP: The Travelling Salesman Problem
   % \chapter{Parallel Greedy Randomized Adaptive Search Procedure}
\section{Introduction}
\subsection{The Travelling Salesman Problem}
\section{Implementation}
\section{Results}



    %% Chapter Option Pricing using Monte Carlo Simulation: 
    \chapter{Option Pricing using Parallel Monte Carlo Simulation}
\section{Introduction}
\label{sec:optionintro}
Derivatives are important in todays financial markets. Futures and
options have been increasingly traded since the last 20 years. Now,
there are many different types of derivatives.
\begin{description}
\item[A derivative] is an financial instrument whose value depends on
  the value of an other variable (or the values of other
  variables). This variable is called \textit{Underlying}.
\end{description}
Derivatives are traded on exchange-traded markets (people trade
standardized contracts defined by the exchange) or over-the-counter
markets (trades are done in a computer linked network or over the
phone - no physical contact).

There are two main types of derivatives, namely forwards (or futures)
and options.

\begin{description}
\item[A Forward] is a contract in which one party buys (long position)
  or sells (short postion) an
  asset at a certain time in the future for a certain price. Unlike
  forward contracts, \textit{futures} are standardized and therefore
  are normally traded on an exchange.
\item[A call Option] is a contract, which gives the holder the right
  to buy an asset at a certain time for a certain price. 
\item[A put Option] is a contract, which gives the holder the right
  to sell an asset at a certain time for a certain price. 
\end{description}

With this information given we can write down their payoff functions:\newline
for the call option
\begin{equation}\label{eq:call}
C_T = \left\{ \begin{array}{lcl} 0 & \textrm{when}& S_T \leq K \\
                         S_T - K & \textrm{when} & S_T > K, \end{array}\right.
\end{equation}
and
\begin{equation}\label{eq:put}
P_T = \left\{ \begin{array}{lcl} 0 & \textrm{when}& K \leq S_T \\
                         K - S_T & \textrm{when} & K > S_T \end{array}\right.
\end{equation}
for the put option.

Figure \ref{fig:payoffs} shows the payoff functions of a call and
a put option and the position of the investor (left:~long position,
right:~short position).


%% TODO: Figure fig:payoffs produced from .rnw

Forward Prices (or Future Prices) can be determined in a simple way
and therefore it is not computational interesting for this chapter.

\subsection{Random Walk}\label{sec:randomwalk}
Given a probability space ($\Omega,{\cal F},P$) we can differentiate
between stochastic processes in discrete time (sequences of random
variables $(X_n)_{n=0}^N$, $N \in\mathbf{N}$) and stochastic processes
in continous time $(X_t)_{0 \leq t \leq T}$.

A process $(X_n)_{n=0}^N$ with independent and identical distributed
increments~($Z_i ~ Z_1$ for $n=1,\ldots,N$)is called random walk.
$$ X_n = X_0 + Z_1 + Z_2 + \ldots + Z_n $$
Figure~\ref{fig:randomwalk} shows a random walk starting from $X_0 =
10$ and $Z_i ~ Z_1 ~ N(0,1)$.

%% TODO: Figure produced from RNW


\subsection{Binomial process}

A random walk, which can only have 1 of 2 values $u$ and $d$ as increments, is
called a binomial process.

There is a $0 \leq p \leq 1$ where the probability $P(Z_n = u)$ equals
$p$ and the probability $P(Z_n = d)$ equals $1 - p$.

Figure~\ref{fig:binomial} shows a binomial tree (relevant for the
pricing of the option) and a path of a binomial process. 


%% TODO: Figure produced from RNW

\subsection{Wiener process}

A process $(W_t)_{0 \leq < \infty}$ is a Wiener process if
\begin{enumerate}
\item $W_0 = 0$,
\item the paths are continous,
\item all increments $W_{t_1}, W_{t_2} - W_{t_1}, \ldots W_{t_n} - W_{t_{n-1}}$
  are independent and normal distributed for all $0 < t_1 < t_2 < \ldots < t_n$ 
\end{enumerate}


%% TODO: Figure produced from RNW

In figure \ref{fig:wienerpath} one can see an example of a Wiender
path.

\subsection{Black-Scholes model}

There is a market which consists of
\begin{itemize}
\item a bank account process $B_t = e^{rt}$
\item and a share $S_t = S_0 e^{(\mu - \frac{\sigma^2}{2}) t + \sigma W_t}$. 
\end{itemize}

Equation~\ref{eq:blackscholes} gives the price of a European call
option of time $t$ with maturity $T$ and payoff function $h(S_T)$.
\begin{equation}\label{eq:blackscholes}
 F(t,x) = \int e^{-r(T-t)}h(xe^{(r - \frac{\sigma^2}{2})(T-t)+\sigma \sqrt{T-tz}})\phi(z)dz 
\end{equation}

\section{Implementation}
\section{Results}



    %% Chapter Parallel Support Vector Machines: Classification with
    %% Support Vector Machines
    %% \chapter{Parallel Implementation using Support Vector Machines}
\section{Introduction}
\section{Implementation}
\section{Results}


  
    %% Conclusions and Future Work
    \chapter{Conclusion and Future Work}
\label{chap:conclusion}
\section{Summary}

The overview of the field of high performance computing presented in
the first chapters is a good start to become familiar with the state
of the art in high performance computing and in particular in parallel
computing. Furthermore we presented in this thesis all relevant
packages which extent the base R environment with parallel computing
facilities. All in all they are a good start to implement parallel
routines although it is hard work to get routines running with high
stability. Nevertheless as algorithms become more time comsuming and
problems in computational statistics become more data intensive we
propose to have at least a look on the possiblities which are offered
from parallel programming models.

With the introduction of OpenMP parallel programming has become easier
than ever. In this thesis we presented how one can only with a few
statements parallelize existing serial code to achieve a significant
reduction of the runtime. With R coming the restriction to low level
languages like C or FORTRAN as a disadvantage. Therefore we propose
implementing high level interface functions to efficient parallel
OpenMP low level routines.

Using the benchmark environment provided by the new package \pkg{paRc}
shows that existing high level  functions for general use in parallel
computing (like the routines in \pkg{snow})
help the user to easy achieve parallelism. Nevertheless, creating
specialized function for a certain task will give better performance.

The second contribution of this thesis is the parallel implementation
of Monte Carlo simulation with a special focus on derivative
pricing. Especially in finance, with the increasing importance of
algorithmic trading, minimizing time is of major interest. The need for
cutting edge hardware of investment companies and banks to minimize
round tripping underlines this trend. Using high performance computing
is becoming increasingly important in this field as parallel computers
become mainstream and it will be a major contribution in this
development.


\section{Economical Coherences}

%When talking about optimization in business administration one needs
%operating figures.
The economic principle has ever been maximizing
profits. This means either raising of revenue, increasing the return
or augmenting assets. In today's ever faster society time has become
an important figure. And indeed, minimizing time to complete a task
can be seen as some sort of profit. In view of finance saving time for
calculating a trading strategy means an advance in knowledge and
therefore can raise ones (monetary) profits.

The objective of parallel
computing is increasing performance which means reducing computational
time. In an economic which depends more and more on information
technology many companies will have to do the step from the serial to
the parallel world. This means that a major rethinking will take place. 

Already a long time ago one of the most important economist \cite{smith:iin}
proposed the division of labor to increase welfare and now it is time
to divide tasks such that time is minimized or in other words profits
are increased.

\section{Outlook}

%%benchmark environment.


Further research could clearly be done in the computational finance
segment. The recent advances in algorithmic trading and the increasing
need for computational power clearly indicate a  trend to high
performance computing.

A first step would be to improve parallel pseudo random number
generation. Only few research has been done in this area in the last
years 

%%computational statistics 

%%algorithmic trading~\cite{domowitz:mai}
%%Improvements and further work 
%%\cite{kontoghiorghes06handbookpcstat}


%%Further research could also be done


%%Going one setp further


Besides the topics mentioned above, many more could be thought of and
implemented, thus there are more than enough fields for further
research.


\begin{acknowledgments}
I was extremely lucky to carry out my diploma thesis at the Department
of Statistics and Mathematics. My position as a studies assistant gave
me the opportunity to discuss many topics of this thesis with
the staff members---my collegues. Especially I would like to thank my
supervisor Kurt Hornik for his guidance and mentorship and of course for
the many reading.

I would like to thank David Meyer as well as
Achim Zeileis who helped me with statistical, graphical and
information technology issues.

Furthermore I would like to thank my Josef Leydold for his hints
regarding random number generation.

Lastly, I would like to thank my girlfriend Selma for proofreading and
of course for her patience and her support.

\end{acknowledgments}
  
    %% appendix
    \appendix
    
    %% Appendix A: PVM/MPI Implementation command line handling
    
\section{Setting up MPI implementations on Debian}
\label{app:mpi_imp}

lamboot lamwipe etc.


    %% Appendix B: Sun Grid Engine
    
\section{Sun Grid Engine}
\label{app:gridengine}

\subsection{LAM/MPI parallel environment}

To use the LAM/MPI pe with the grid engine the following has to be
inserted into the user's .bashrc:

\begin{verbatim}

export LD_LIBRARY_PATH=/home/stheussl/lib/lam-gcc/lib:\$LD_LIBRARY_PATH
export LAMHOME=/home/stheussl/lib/lam-gcc

\end{verbatim}

A job script for starting a LAM/MPI application looks like this:

\begin{verbatim}

#\$ -N Rmpi-example
# use parallel environment lam with 12 nodes
#\$ -pe lam 12
# run job on queue node.q to have a homogeneous processing environment 
#\$ -q node.q

R --vanilla < /path/to/my/Rmpi-example.R

\end{verbatim}


\subsection{PVM parallel environment}

To use the PVM pe with the grid engine the following has to be
inserted into the user's .bashrc:

\begin{verbatim}

# PVM>
# you may wish to use this for your own programs (edit the last
# part to point to a different directory f.e. ~/bin/_\$PVM_ARCH.
#
if [ -z \$PVM_ROOT ]; then
    if [ -d /home/stheussl/lib/pvm3 ]; then
        export PVM_ROOT=/home/stheussl/lib/pvm3
    else
        echo "Warning - PVM_ROOT not defined"
        echo "To use PVM, define PVM_ROOT and rerun your .bashrc"
    fi
fi

if [ -n \$PVM_ROOT ]; then
    export PVM_ARCH=`\$PVM_ROOT/lib/pvmgetarch`

# uncomment one of the following lines if you want the PVM commands
# directory to be added to your shell path.
#
       export PATH=\$PATH:\$PVM_ROOT/lib            # generic
       export PATH=\$PATH:\$PVM_ROOT/lib/\$PVM_ARCH  # arch-specific
#
# uncomment the following line if you want the PVM executable directory
# to be added to your shell path.
#
       export PATH=\$PATH:\$PVM_ROOT/bin/\$PVM_ARCH
fi

\end{verbatim}

A job script for starting a PVM application looks like this:

\begin{verbatim}

#\$ -N rpvm-test
# use parallel environment pvm with 8 nodes
#\$ -pe pvm 8 
# run job on queue node.q to have a homogeneous processing environment 
#\$ -q node.q

## if shared library not available in standard paths add the following
export LD_LIBRARY_PATH=/home/stheussl/lib/pvm3/lib/LINUX64/:\$LD_LIBRARY_PATH

R --vanilla < /path/to/my/rpvm-example.R

\end{verbatim}
    %% Appendix C: Benchmark definitions
    \chapter{Benchmark Definitions}
\label{app:benchmark}
In this appendix the benchmark source files and job scripts are shown.

\section{Matrix Multiplication}

\subsubsection{Uniformly Distributed Matrices}

In general we used the uniform distribution ($[-5,5]$) to run the
benchmark in this thesis. The code for running a benchmark on a
distributed memory machine and a detailed description can
be found in Section~\ref{sec:benchmarkdescription}. The following code
describes a benchmark for a shared memory machine.

\begin{Scode}
## load required libraries
library("Rmpi")
library("paRc")
library("snow")

## definition of benchmark
max_cpu <- 4
task <- "matrix multiplication"
taskID <- "mm"
paradigm <- "shared"
types <- c("OpenMP", "MPI","snow-MPI", "MPI-wB")
complexity <- c(1000, 2500)
runs <- 250
## data description
bmdata <- list()
bmdata[[1]] <- bmdata[[2]] <- 1000
bmdata[[3]] <- function(x){
  runif(x,-5,5)
}

## create benchmark object
bm <- create_benchmark(task=task, data=bmdata,
                       type=types[1], parallel=TRUE,
                       cpu_range=1:max_cpu, runs=runs)
set.seed(1782)
## for each complexity and type run a number of benchmarks
for(n in complexity){
  bmdata[[1]] <- bmdata[[2]] <- n
  benchmark_data(bm) <- bmdata
  for(type in types){
    benchmark_type(bm) <- type
    writeLines(paste("Starting",type,"benchmark with complexity",n,
                     "..."))
    results <- run_benchmark(bm)
    save(results,file=sprintf("%s-%s-%s-%d.Rda", paradigm, taskID,
                               type, n))
  }
}
\end{Scode}

and the corresponding cluster job script looks as follows:

\begin{verbatim}
#$ -N MPI-distr-MM
#$ -pe lam 20
#$ -q node.q

/path/to/R-binary --vanilla < \
  /path/to/MPI-benchmark-examples/distributed-mm.R
\end{verbatim}

\subsubsection{Normally Distributed Matrices}

When using normally distributed entries in the matrices we used the
following benchmark definition to run on a distributed memory machine:

\begin{Scode}
## load required libraries
library("rpvm")
library("paRc")
library("snow")

## definition of benchmark run
max_cpu <- 20
task <- "matrix multiplication"
taskID <- "mm-norm"
paradigm <- "distributed"
types <- c("PVM","snow-PVM")
complexity <- c(1000, 2500, 5000)
runs <- 250
bmdata <- list()
bmdata[[1]] <- bmdata[[2]] <- 1000
bmdata[[3]] <- function(x){
  rnorm(x)
}

bm <- create_benchmark(task=task, data=bmdata,
                       type=types[1], parallel=TRUE,
                       cpu_range=1:max_cpu, runs=runs)
set.seed(1782)
for(n in complexity){
  bmdata[[1]] <- bmdata[[2]] <- n
  benchmark_data(bm) <- bmdata
  for(type in types){
    benchmark_type(bm) <- type
    writeLines(paste("Starting",type,"benchmark with complexity",n,
                     "..."))
    results <- run_benchmark(bm)
    save(results,file=sprintf("%s-%s-%s-%d.Rda", paradigm, taskID,
                               type, n))
  }
}
\end{Scode}

and the corresponding job script for the grid engine looks as follows:

\begin{verbatim}
#$ -N PVM-norm-distr-MM
#$ -pe pvm 20
#$ -q node.q

/path/to/R-binary --vanilla < \
  /path/to/PVM-benchmark-examples/distributed-mm.R
\end{verbatim}

\section{Option Pricing}

To benchmark parallel Monte Carlo simulations the following code was
used:

\begin{Scode}
## load required libraries
library("Rmpi")
library("paRc")

## definition of benchmark run
max_cpu <- mpi.universe.size()
task <- "Monte Carlo simulation"
taskID <- "mcs"
paradigm <- "distributed"
types <- c("normal", "MPI")
runs <- 10

bm <- create_benchmark(task=task, data=list(),
                       type=types[1], paralle=TRUE,
                       cpu_range=1:max_cpu, runs=runs)
## define option
opt <- define_option(c(0.1,0.4,100),100,1/12)

bmdata <- list()
bmdata[[1]] <- opt
bmdata[[2]] <- 0.1 ## yield
bmdata[[3]] <- 30  ## path length
bmdata[[4]] <- 5000 ## number of paths
bmdata[[5]] <- 50  ## number of simulations
bmdata[[6]] <- TRUE ## use antithetic
benchmark_data(bm) <- bmdata

for(type in types){
  benchmark_type(bm) <- type
  writeLines(paste("Starting",type,"benchmark..."))
  results <- run_benchmark(bm)
  save(results,file=sprintf("%s-%s-%s-%d.Rda", paradigm, taskID,
                               type, n))
}

\end{Scode}
the corresponding job script looks as follows:

\begin{verbatim}
#$ -N MPI-distr-MCS
#$ -pe lam 10
#$ -q node.q

/path/to/R-binary --vanilla < \
  /path/to/MPI-benchmark-examples/distributed-mcs.R
\end{verbatim}

   
  % Bibliography (begins on an odd page)
  \cleardoublepage
  
  \listoffigures
\listoftables
  \bibliographystyle{plainnat}
  \bibliography{hpc}
  
\end{document}
