\chapter{Conclusion and Future Work}
\label{chap:conclusion}
\section{Summary}

The overview of the field of high performance computing presented in
the first chapters is a good start to become familiar with the state
of the art in high performance computing and in particular in parallel
computing. Furthermore we presented in this thesis all relevant
packages which extent the base R environment with parallel computing
facilities. All in all they are a good start to implement parallel
routines although it is hard work to get routines running with high
stability. Nevertheless as algorithms become more time comsuming and
problems in computational statistics become more data intensive we
propose to have at least a look on the possiblities which are offered
from parallel programming models.

With the introduction of OpenMP parallel programming has become easier
than ever. In this thesis we presented how one can only with a few
statements parallelize existing serial code to achieve a significant
reduction of the runtime. With R coming the restriction to low level
languages like C or FORTRAN as a disadvantage. Therefore we propose
implementing high level interface functions to efficient parallel
OpenMP low level routines.

Using the benchmark environment provided by the new package \pkg{paRc}
shows that existing high level  functions for general use in parallel
computing (like the routines in \pkg{snow})
help the user to easy achieve parallelism. Nevertheless, creating
specialized function for a certain task will give better performance.

The second contribution of this thesis is the parallel implementation
of Monte Carlo simulation with a special focus on derivative
pricing. Especially in finance, with the increasing importance of
algorithmic trading, minimizing time is of major interest. The need for
cutting edge hardware of investment companies and banks to minimize
round tripping underlines this trend. Using high performance computing
is becoming increasingly important in this field as parallel computers
become mainstream and it will be a major contribution in this
development.


\section{Economical Coherences}

%When talking about optimization in business administration one needs
%operating figures.
The economic principle has ever been maximizing
profits. This means either raising of revenue, increasing the return
or augmenting assets. In today's ever faster society time has become
an important figure. And indeed, minimizing time to complete a task
can be seen as some sort of profit. In view of finance saving time for
calculating a trading strategy means an advance in knowledge and
therefore can raise ones (monetary) profits.

The objective of parallel
computing is increasing performance which means reducing computational
time. In an economic which depends more and more on information
technology many companies will have to do the step from the serial to
the parallel world. This means that a major rethinking will take place. 

Already a long time ago one of the most important economist \cite{smith:iin}
proposed the division of labor to increase welfare and now it is time
to divide tasks such that time is minimized or in other words profits
are increased.

\section{Outlook}

%%benchmark environment.


Further research could clearly be done in the computational finance
segment. The recent advances in algorithmic trading and the increasing
need for computational power clearly indicate a  trend to high
performance computing.

A first step would be to improve parallel pseudo random number
generation. Only few research has been done in this area in the last
years 

%%computational statistics 

%%algorithmic trading~\cite{domowitz:mai}
%%Improvements and further work 
%%\cite{kontoghiorghes06handbookpcstat}


%%Further research could also be done


%%Going one setp further


Besides the topics mentioned above, many more could be thought of and
implemented, thus there are more than enough fields for further
research.


\begin{acknowledgments}
I was extremely lucky to carry out my diploma thesis at the Department
of Statistics and Mathematics. My position as a studies assistant gave
me the opportunity to discuss many topics of this thesis with
the staff members---my collegues. Especially I would like to thank my
supervisor Kurt Hornik for his guidance and mentorship and of course for
the many reading.

I would like to thank David Meyer as well as
Achim Zeileis who helped me with statistical, graphical and
information technology issues.

Furthermore I would like to thank my Josef Leydold for his hints
regarding random number generation.

Lastly, I would like to thank my girlfriend Selma for proofreading and
of course for her patience and her support.

\end{acknowledgments}