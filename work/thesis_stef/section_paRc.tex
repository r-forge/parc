\section{paRc---PARallel Computations in R}
\label{sec:paRc}

In the course of this thesis a package called \pkg{paRc}
(\cite{theussl07paRc}) has been developed with the aim
to evaluate performance of parallel computations and to show how
interfacing high performance applications written in C can be done
when using OpenMP ~(see Section~\ref{sec:OpenMP}.

The package \pkg{paRc} contains interface functions to the OpenMP library
and provides high level function to a few C implementations of
parallel applications using OpenMP.
Furthermore, it supplies a benchmark environment for performance
evaluation of parallel programs.

\pkg{paRc} can be obtained from \url{R-Forge.R-project.org}---the
R-project community service. To install this package directly within R
call \code{install.packages("paRc", repos = "R-Forge.r-project.org")}.
To properly install this package you need either the Intel compiler
with version 9.1 or newer (the Linux compiler is free for
non-commercial use) or the GNU~C compiler with version 4.2 or
newer. They are known to support OpenMP.

\subsection{OpenMP interface functions}

\begin{description}
\item[\texttt{omp.get.num.procs()}] returns the number of threads
  available to the program.
\item[\texttt{omp.set.num.threads()}] sets the number of threads to be
  used in subsequent parallel executions.
\item[\texttt{omp.get.max.threads()}] gets the number of threads to be
  used in subsequent parallel executions.
\end{description}


\subsubsection{OpenMP specific environment variables}

Certain environment variables affect the runtime behaviour of OpenMP
programs. These environment variables are~(\cite{openMP05}):

\begin{description}
\item[\code{OMP\_NUM\_THREADS}] sets the number of threads to use in
  parallel regions of OpenMP programs. 
\item[\code{OMP\_SCHEDULE}] sets the runtime schedule type and
  chunk size.
\item[\code{OMP\_DYNAMIC}] defines wether dynamic adjustments of threads
  should be used in parallel regions.
\item[\code{OMP\_NESTED}] enables or disables nested parallelism.
\end{description}



\begin{Example} OpenMP function calls using \pkg{paRc} \newline
running on cluster@WU using a bignode which provides 4 cores.

\begin{Schunk}
\begin{Sinput}
> library("paRc")
> omp.get.num.procs()
\end{Sinput}
\begin{Soutput}
[1] 4
\end{Soutput}
\begin{Sinput}
> omp.set.num.threads(2)
> omp.get.max.threads()
\end{Sinput}
\begin{Soutput}
[1] 2
\end{Soutput}
\end{Schunk}
\label{ex:paRcOMP}
\end{Example}


\subsection{Built-in high level functions}

\pkg{paRc} provides the following high level OpenMP functions: 

\begin{description}
\item[\texttt{omp.matrix.mult(X, Y, ncpu = 1)}] multiplies the matrix
  \code{X} with matrix \code{Y} using \code{n\_cpu} numbers of
  processors.
\end{description}


\subsection{Benchmark}

handle benchmark objects

\begin{description}
\item[\code{bm.cpu.range}]
\item[\code{bm.data}]
\item[\code{bm.functions.to.apply}]
\item[\code{bm.is.parallel}]
\item[\code{bm.task}]
\item[\code{run.benchmark}]
\end{description}

S3 methods

\begin{description}
\item[\code{print.benchmark}]
\item[\code{plot.benchresults}]
\end{description}

\subsection{Other important functions}

To complete the set of important functions supplied by package
\pkg{paRc} the following functions have to be explained:

\subsection{Conclusion}

%The \texttt{PVM.rapply()} example shown in this section followed the Single Program
%Multiple Data (SPMD) paradigm. Data is splitted into different parts
%which are sent to different processes. I/O is handled by a master
%process. When loading rpvm in an R session this session becomes the
%master process. Slaves can easily be spawned provided that there are
%working slave scripts available. A major disadvantage is that the rpvm
%package only has two higher-level function. One of them can be used
%for calculations. That means when using this package for HPC one has
%to deal with low-level message-passing which in turn provides high
%flexibility. New parallel functions can be constructed on the basis of
%the provided interface.

%For further interface functions supplied by the \pkg{paRc} package, a more detailed
%description and further examples please consult the package description
%\cite{yu06Rmpi}.
