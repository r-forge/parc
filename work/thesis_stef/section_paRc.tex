
:q\section{paRc---PARallel Computations in R}
\label{sec:paRc}

In the course of this thesis a package called \pkg{paRc}
(\cite{theussl07paRc}) has been developed with the aim
to evaluate performance of parallel applications and to show how
interfacing high performance applications written in C can be done
using OpenMP~(see Section~\ref{sec:OpenMP}).

The package \pkg{paRc} contains interface functions to the OpenMP library
and provides high level function to a few C implementations of
parallel applications using OpenMP (e.g., matrix multiplication---see
Chapter~\ref{chap:matrix}).
Furthermore, it supplies a benchmark environment for performance
evaluation of parallel programs and a framework for pricing options
with parallel Monte Carlo simulation (see
Chapter~\ref{chap:options}).

\pkg{paRc} can be obtained from \url{R-Forge.R-project.org}---the
R-project community service. To install this package directly within R
call \code{install.packages("paRc", repos = "R-Forge.r-project.org")}.
To properly install this package you need either the Intel compiler
with version 9.1 or newer (the Linux compiler is free for
non-commercial use) or the GNU~C compiler with version 4.2 or
newer. They are known to support OpenMP.

Examples in this section are produced on a bignode of
cluster@WU. Bignodes provide a shared memory platform with up to 4
CPUs. Shared memory platforms are necessary for running parallel
OpenMP applications.

\subsection{OpenMP Interface Functions}

The user is provided with a few interface functions to the OpenMP
library. They are used to query the internal variables of the compiled
parallel environment or to change them.

\subsubsection{OpenMP Routines}
\begin{description}
\item[\texttt{omp.get.num.procs()}] returns the number of threads
  available to the program.
\item[\texttt{omp.set.num.threads()}] sets the number of threads to be
  used in subsequent parallel executions.
\item[\texttt{omp.get.max.threads()}] gets the number of threads to be
  used in subsequent parallel executions.
\end{description}

\subsubsection{OpenMP Specific Environment Variables}

Moreover, environment variables can affect the runtime behaviour of
OpenMP programs. These environment variables are~(\cite{openMP05}):

\begin{description}
\item[\texttt{OMP\_NUM\_THREADS}] sets the number of threads to use in
  parallel regions of OpenMP programs. 
\item[\texttt{OMP\_SCHEDULE}] sets the runtime schedule type and
  chunk size.
\item[\texttt{OMP\_DYNAMIC}] defines wether dynamic adjustments of threads
  should be used in parallel regions.
\item[\texttt{OMP\_NESTED}] enables or disables nested parallelism.
\end{description}

Example~\ref{ex:paRcOMP} shows the use of the OpenMP library calls in
R. First the number of available processors is queried. Then the
number of threads a parallel application may use is set to 2. With the
last call the current available CPUs to a parallel program is queried.

\begin{Example} OpenMP function calls using \pkg{paRc}
\label{ex:paRcOMP}
\begin{Schunk}
\begin{Sinput}
> library("paRc")
> omp.get.num.procs()
\end{Sinput}
\begin{Soutput}
[1] 4
\end{Soutput}
\begin{Sinput}
> omp.set.num.threads(2)
> omp.get.max.threads()
\end{Sinput}
\begin{Soutput}
[1] 2
\end{Soutput}
\end{Schunk}
\end{Example}

\subsection{High Level OpenMP Functions}

\pkg{paRc} provides the following high level OpenMP function: 

\begin{description}
\item[\texttt{omp.matrix.mult(X, Y, n\_cpu = 1)}] multiplies the matrix
  \code{X} with matrix \code{Y} using \code{n\_cpu} numbers of
  processors.
\end{description}

\subsection{Benchmark Environment}

\pkg{paRc} provides a benchmark environment for measuring the
performance of parallel programs. Two main functions exist in this
context---one for creating a benchmark object and one for running the
benchmark described by the object.

\subsubsection{Class \class{benchmark}}

An S3 object (\cite{chambers91sms}) named \class{benchmark} contains all
the necessary information to run a benchmark. The list elements are

\begin{description}
\item[task] is a character string defining the task of the
  benchmark. Currently, the 
  following tasks are implemented:
  \begin{itemize}
  \item matrix multiplication
  \item Monte Carlo simulation
  \end{itemize}
\item[data] is a list containing the parameters and data to properly
  run the task
\item[type]defines the parallel programming model used to run the
  benchmark. Currently, the following types are implemented:
  \begin{itemize}
  \item OpenMP---C interface calls provided by \pkg{paRc}
  \item MPI---implementation in \pkg{paRc} using \pkg{Rmpi} for
    communication
  \item PVM---implementation in \pkg{paRc} using \pkg{rpvm} for
    communication 
  \item snow-MPI---implementation in \pkg{snow} using \pkg{Rmpi} for
    communication
  \item snow-PVM---implementation in \pkg{snow} using \pkg{rpvm} for
    communication
  \end{itemize}
\item[cpu\_range] contains a vector of integers representing the number
  of CPUs for th corresponding benchmark run 
\item[is\_parallel] is a logical \code{TRUE} or \code{FALSE} wether the
  contains parallel tasks or not
\end{description}

\subsubsection{Main Routines}

These are the main routines for benchmarking parallel applications:

\begin{description}
\item[\code{create.benchmark(task, data, type, cpu\_range, ...)}]
  defines a benchmark object using a specific \code{task} and the
  corresponding \code{data}. The \code{type} refers to the serial or
  parallel paradigm to use. The \code{cpu\_range} specifies the range
  of CPUs to use for this benchmark. 
\item[\code{run.benchmark(x)}] takes a benchmark object as argument
  and carries out the defined benchmark. It returns an object of type
  bm\_results (which inherits from a dataframe) containing the
  results of the benchmark.
\end{description}


\subsubsection{Extractor and Replacement Functions}

The following routines are for handling a benchmark object. They
extract or replace the values in the benchmark \code{x}.

\begin{itemize}
\item \code{bm.task(x)}
\item \code{bm.data(x)}
\item \code{bm.type(x)}
\item \code{bm.cpu.range(x)}
\end{itemize}

The following routines supply extra information about the benchmark object
or the benchmark environment.

\begin{description}
\item[\code{bm.is.parallel}] returns \code{TRUE} if the benchmark
  contains a parallel function. 
\item[\code{bm.tasks}] returns the tasks which are possible to run
  with the benchmark environment.
\item[\code{bm.types}] returns the types of available serial or
  parallel paradigms to run with the benchmark environment.
\end{description}

\subsubsection{Generic Functions}

Generic functions provide methods for different objects. In \pkg{paRc}
a generic function for calculating the speedup is provided:

\begin{description}
\item[\code{speedup(x)}] is a generic function taking an object as
  arguments. Currently there are two methods implemented namely
  \code{speedup.numeric} and \code{speedup.bm\_results}. The methods
  calculate the speedup as it is presented in
  Equation~\ref{eq:speedup}.
\end{description}

\subsubsection{S3 Methods}

The following S3 methods are provided for the benchmark environment:

\begin{description}
\item[\code{print.benchmark}] prints objects of class \class{benchmark}
\item[\code{plot.bm\_results}] supplies a plot method for comparing
  benchmark results.
\item[\code{speedup.default}] returns an error message that there is
  no default method. 
\item[\code{speedup.numeric}] returns the speedups calculated from a
  vector of type \class{numeric}. The reference execution time is  the
  first element in the vector. The return value is a vector of type
  \class{numeric}. 
\item[\code{speedup.bm\_results}] calculates the speedups from an object
  of class \class{bm\_results} and returns them as a vector of type
  \class{numeric}. 
\end{description}

\subsubsection{Example}

\begin{Example} Running a benchmark using OpenMP
\label{ex:benchrun}
\begin{Schunk}
\begin{Sinput}
> n <- 1000
> maxcpu <- omp.get.num.procs()
> dat <- list()
> dat[[1]] <- matrix(runif(n * n, -5, 5), nrow = n)
> dat[[2]] <- matrix(runif(n * n, -5, 5), nrow = n)
> bm <- create.benchmark(task = "matrix multiplication", data = dat, 
+     type = "OpenMP", parallel = TRUE, cpu_range = 1:maxcpu)
> bm
\end{Sinput}
\begin{Soutput}
A parallel benchmark running task: matrix multiplication - OpenMP
\end{Soutput}
\begin{Sinput}
> bmres <- run.benchmark(bm)
> bmres
\end{Sinput}
\begin{Soutput}
                   task   type n_cpu time_usr time_sys time_ela is_parallel
1 matrix multiplication OpenMP     1    7.422    0.051    7.477        TRUE
2 matrix multiplication OpenMP     2    7.218    0.057    3.698        TRUE
3 matrix multiplication OpenMP     3    7.712    0.056    2.712        TRUE
4 matrix multiplication OpenMP     4    7.914    0.052    2.101        TRUE
\end{Soutput}
\begin{Sinput}
> speedup(bmres)
\end{Sinput}
\begin{Soutput}
  OpenMP   OpenMP   OpenMP   OpenMP 
1.000000 2.021904 2.757006 3.558782 
\end{Soutput}
\end{Schunk}
\end{Example}

Running a dedicated benchmark is shown in
Example~\ref{ex:benchrun}. The object \code{bm} contains all the
information to carry out the benchmark. The task to run is matrix
multiplication of two $1000 \times 1000$ matrices. The parallel
paradigm to use is OpenMP using 1 up to the maximum of CPUs available
on the machine. The resulting data.frame is printed and the speedup is
calculated.

\subsection{Environment for Option Pricing}

In package \pkg{paRc} a framework for pricing financial derivatives,
in particular options, is available. The framework is built around the
main function namely Monte Carlo simulation and its parallel
derivative. 

\subsubsection{Class \class{option}}

An S3 object named \class{option} contains all
the necessary information for pricing an option. The list elements are

\begin{description}
\item[underlying] is a numeric vector containing the three elements to
  describe a stock:
  \begin{itemize}
  \item the return $\mu$,
  \item the volatility $\sigma$,
  \item and the present value of the stock.
  \end{itemize}
\item[strikeprice] is a numeric defining the strikeprice of the option.
\item[maturity] is a numeric defining the time until the option expires.
\item[type] is a character representing the type of the option. There
  are two possibilities---either a \code("call") option, or a
  \code("put") option.
\item[kind] contains a character representing the class of the
  option. Only \code{"European"} has been available at the time of
  this writing. 
\item[position] is a character string marking the position of the
  investor. This can either be \code{"long"} or \code{"short"}.
\end{description}


\subsubsection{Main Routines}

These are the main routines for pricing an option:

\begin{description}
\item[\code{define.option(underlying, strikeprice, maturity, type =
    "Call", class = "European", position = "long")}] defines an object
  of class \class{option}. 
\item[\code{blackscholesprice(x)}] takes an object of class
  \class{option} as argument and returns the analytical solution of
  the Black Scholes differential equation (the price of the
  option). This works only for European options (see
  Section~\ref{sec:blackscholes} for details).
\item[\code{monteCarloSimulation(x, r, n, length, n.simulations = 50,
    antithetic = TRUE)}] carries out a Monte Carlo simulation pricing
  an option given by \code{x}. Further arguments are the risk free
  yield \code{r}, the lenght of a wiener path \code{n}, the number of
  simulated paths \code{length}, the number of simulation runs
  \code{n.simulations} and if antithetic variance reduction should be
  used (\code{antithetic}).
\item[\code{mcs.Rmpi(x, r, n, length, n.simulations = 50,
    antithetic = TRUE, n\_cpu = 1, spawnRslaves = FALSE)}] is the same
  like \code{monteCarloSimulation()} but uses parallel generation of
  option prices. Additional arguments \code{n\_cpu} and
  \code{spawnRslaves} have to be provided naming the number of CPUs
  and if R slaves are to be spawned respectively.
\end{description}

\subsubsection{Extractor and Replacement Functions}

The following routines are for handling an object of class
\class{option}. They
extract or replace the values in the option \code{x}.

\begin{itemize}
\item \code{maturity(x)}
\item \code{strikeprice(x)}
\item \code{underlying(x)}
\item \code{optiontype(x)}
\item \code{optionclass(x)}
\item \code{position(x)}
\end{itemize}

\subsubsection{S3 Methods}

The following S3 methods are provided for the option pricing environment:

\begin{description}
\item[\code{print.option}] defines the print method for class \class{option}.
\item[\code{plot.option}] plots the payoff of the given \class{option}.
\end{description}

\subsubsection{Example}

\begin{Example} Handling class \class{option}
\label{ex:option}
\begin{Schunk}
\begin{Sinput}
> european <- define.option(c(0.1, 0.4, 100), 100, 1/12)
> european
\end{Sinput}
\begin{Soutput}
A call option with strike price 100
expiring in 30 days
\end{Soutput}
\begin{Sinput}
> underlying(european)
\end{Sinput}
\begin{Soutput}
 mean    sd value 
  0.1   0.4 100.0 
\end{Soutput}
\begin{Sinput}
> optiontype(european)
\end{Sinput}
\begin{Soutput}
[1] "call"
\end{Soutput}
\begin{Sinput}
> optionclass(european)
\end{Sinput}
\begin{Soutput}
[1] "european"
\end{Soutput}
\begin{Sinput}
> strikeprice(european)
\end{Sinput}
\begin{Soutput}
[1] 100
\end{Soutput}
\begin{Sinput}
> maturity(european)
\end{Sinput}
\begin{Soutput}
[1] 0.08333333
\end{Soutput}
\begin{Sinput}
> underlying(european) <- c(0.2, 0.5, 100)
> underlying(european)
\end{Sinput}
\begin{Soutput}
 mean    sd value 
  0.2   0.5 100.0 
\end{Soutput}
\end{Schunk}
\end{Example}

In Example~\ref{ex:option} the variable \code{european} is assigned an
object of class \class{option}. It is defined as an European option
with strikeprice 100 and maturity 1/12 (30 days). The underlying has a
present value of 100, a return of 0.1 and volatility of 0.4. Invoking the print
method returns a short summary of the object. The rest
of the example shows the use of different extractor functions and an
replacement function. Figure~\ref{fig:payoff} shows the payoff of the
defined option.

\begin{figure}
\centering
\includegraphics{section_paRc-004}
\label{fig:payoff} 
\end{figure}

\subsection{Other Functions}

To complete the set of functions supplied by package \pkg{paRc} the
following function has to be explained:

\begin{description}
\item[\code{serial.matrix.mult(X,Y)}] takes the matrices \code{X} and
  \code{Y} as arguments and performs a serial matrix
  multiplication. This function calls a C routine providing a
  non-optimzed version of the serial matrix multiplication (see
  Chapter~\ref{chap:matrix} for more details).  
\end{description}

\subsection{Conclusion}

