\section{paRc -- PARallel Computations in R}
\label{sec:paRc}

The package paRc contains high-level functions to OpenMP compiled
parallel functions.
Furthermore, it supplies a benchmark environment for performance
evaluation of parallel programs.


\subsection{OpenMP interface functions}

\begin{description}
\item[\texttt{omp.get.num.procs()}] returns the number of threads
  available to the program.
\item[\texttt{omp.set.num.threads()}] sets the number of threads to be
  used in subsequent parallel executions.
\item[\texttt{omp.get.max.threads()}] gets the number of threads to be
  used in subsequent parallel executions.
\end{description}


\subsubsection{OpenMP specific environment variables}

Certain environment variables affect the runtime behaviour of OpenMP
programs. These environment variables are as
follows~(\cite{openMP05}):

\begin{description}
\item[\texttt{OMP_NUM_THREADS}] sets the number of threads to use in
  parallel regions of OpenMP programs. 
\item[\texttt{OMP_SCHEDULE}] sets the runtime schedule type and
  chunk size.
\item[\texttt{OMP_DYNAMIC}] defines wether dynamic adjustments of threads
  should be used in parallel regions.
\item[\texttt{OMP_NESTED}] enables or disables nested parallelism.
\end{description}



\textbf{Example:} OpenMP function calls using paRc \newline
running on cluster@WU using a bignode which provides 4 cores.

\begin{Schunk}
\begin{Sinput}
> library("paRc")
> omp.get.num.procs()
> omp.set.num.threads(2)
> omp.get.max.threads()
\end{Sinput}
\end{Schunk}

\subsection{Built-in high level functions}

Rmpi provides the following high level functions (this is only a selection): 

\begin{description}
\item[\texttt{omp.matrix.mult(X, Y, ncpu = 1)}] multiplies the matrix
  \texttt{X} with matrix \texttt{Y} using \texttt{n_cpu} numbers of
  processors.
\end{description}


\subsection{Benchmark functions}

\subsection{Other important functions}

To complete the set of important functions supplied by the paRc
package the following functions have to be explained:

\subsection{Conclusion}

%The \texttt{PVM.rapply()} example shown in this section followed the Single Program
%Multiple Data (SPMD) paradigm. Data is splitted into different parts
%which are sent to different processes. I/O is handled by a master
%process. When loading rpvm in an R session this session becomes the
%master process. Slaves can easily be spawned provided that there are
%working slave scripts available. A major disadvantage is that the rpvm
%package only has two higher-level function. One of them can be used
%for calculations. That means when using this package for HPC one has
%to deal with low-level message-passing which in turn provides high
%flexibility. New parallel functions can be constructed on the basis of
%the provided interface.

%For further interface functions supplied by the paRc package, a more detailed
%description and further examples please consult the package description
%\cite{yu06Rmpi}.
