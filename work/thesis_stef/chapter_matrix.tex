\chapter{Parallel Matrix Multiplication}
\label{chap:matrix}
\section{Introduction}

If we think of applications in parallel computing matrix
multiplication comes into mind. Because of its nature it is prime
example for data parallelism. There are many algorithms for
parallelizing matrix multiplication available.


In this chapter a short introduction to matrix multiplication is
given. Subsequently implementations of a selection of parallel
algorithms are presented. Eventually results of the comparison are
presented.


\section{Theory}

\subsection{Notation}

$ \mathbf{R} $ denotes the set of real numbers and $ \mathbf{R}^{m
  \times n} $ the vector space of all m-by-n real matrices.

$$ A \in \mathbf{R}^{m \times n} \Longleftrightarrow A = (a_{ij}) = 
\left( \begin{array}{ccc}
a_{11} & \ldots & a_{1n} \\
\vdots &        & \vdots \\
a_{m1} & \ldots & a_{mn}
\end{array} \right)
a_{ij} \in \mathbf{R}
 $$

The lower case letter of the letter which denotes the matrix with
subsripts $ij$ refers to the entry in the matrix. 

matrix-matrix multiplication ($\mathbf{R}^{m \times r} \times
\mathbf{R}^{r \times n} \to \mathbf{R}^{m \times n}$)

$$ C = AB \Longrightarrow c_{ij} = \sum_{k=1}^r a_{ik}b_{kj} $$


\section{Implementation}



\section{Results}

