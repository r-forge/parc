\chapter{High Performance Computing and R}
\label{chap:Rhpc}
\section{Introduction}

This chapter provides an overview of the capabilities of R (\cite{IntrotoR}
and \cite{ihaka96rld}) in the area
of high performance computing. A short description of the
software package R is given at the beginning of this
chapter. Subsequently extensions to the base environment 
(packages) which provide high performance computation functionality to
R are going to be explained. Among these extensions there is the
package called \pkg{paRc}, which has been developed in the course of
this thesis.

\section{The R environment}

R is an integrated suite of software facilities for data manipulation,
calculation and graphical display. Among other things it has (\cite{IntrotoR})
\begin{itemize}
\item an effective data handling and storage facility,
\item a suite of operators for calculations on arrays, in particular matrices,
\item a large, coherent, integrated collection of intermediate tools
  for data analysis,
\item graphical facilities for data analysis and display either
  directly at the computer or on hardcopy, and
\item a well developed, simple and effective programming language
  (called `S') which includes conditionals, loops, user defined
  recursive functions and input and output facilities. (Indeed most of
  the system supplied functions are themselves written in the S
  language.)
\end{itemize}

R is not only an environment for statistical computing and graphics
but also a freely available high-level language for programming. It
can be extended by standardized collections of code called
``packages''. So developers and statisticians around the world can
participate and provide optional code to the base R environment.
Developing and implementing new methods of data analysis can therefore
be rather easy to achieve (for more details on how R can be extended
see \cite{Rcore07Ext}). 

As datasets grow bigger and bigger and algorithms become more and more
complex, R has to join the high performance computing hype. Indeed, R
is already prepared through a few extensions explained in the
subsequent chapters.

%% subsection: The Rmpi package 
\subsection{The Rmpi package}
\label{sec:Rmpi}
The Message Passing Interface (MPI) is a set of library interface
standards for message-passing and there are many implementations using
these standards (see also section \cite{sec:MPI}).
Rmpi is an interface to MPI (\cite{yu06Rmpi}). As of the time of this
writing Rmpi uses
the LAM implementation of MPI. For process spawning the standard
MPI-1.2 is required which is available in the LAM/MPI 
as LAM/MPI (version 7.1.3) supports a large portions of the MPI-2
standard. There are 
a lot of low-level interface functions to the MPI C-library available.
Furthermore, a handful of high-level functions are supplied by the
Rmpi package. 

A windows implementation of this package (which uses MPICH2)
can also be obtained, but the Microsoft operating system is not in the
scope of this thesis.

\subsubsection{Initializing and Status queries}

The LAM/MPI environment has to be booted prior to using any
message-passing library functions. One possibility is to use the
command line, the other is to load the Rmpi package. It automatically
sets up a (small--1 host) LAM/MPI environment (if the executables are
in the search path). 

When using the Sun Grid Engine (SGE) to boot the LAM/MPI parallel
environment the developer is not engaged with
setting up and booting the environment anymore (see
appendix \ref{app:gridengine} on how to do this). On a cluster of
workstations this is the method of choice. 


Rmpi management and query functions:

\begin{description}
\item[\texttt{lamhosts()}] finds the hostname associated with its node
  number.
\item[\texttt{mpi.universe.size()}] returns the total number of CPUs
  available to the MPI environment (ie. in a cluster or in a parallel
  environment started by the grid engine).
\item[\texttt{mpi.finalize()}] cleans all MPI states.
\item[\texttt{mpi.exit()}] terminates the mpi communication
  environment and detaches the Rmpi package which makes reloading of
  the package Rmpi in the same session impossible.  
\item[\texttt{mpi.quit()}] terminates the mpi communication
  environment and quits R.  
\end{description}
  

\textbf{Example:} Simple queries using Rmpi \newline
running on cluster@WU using the node.q -- the parallel environment was
started with the SGE using 8 nodes.

\begin{Schunk}
\begin{Sinput}
> library("Rmpi")
> set.seed(1782)
> lamhosts()
\end{Sinput}
\begin{Soutput}
node059 node059 node055 node055 node064 node064 node062 node062 
      0       1       2       3       4       5       6       7 
\end{Soutput}
\begin{Sinput}
> mpi.universe.size()
\end{Sinput}
\begin{Soutput}
[1] 8
\end{Soutput}
\begin{Sinput}
> mpi.is.master()
\end{Sinput}
\begin{Soutput}
[1] TRUE
\end{Soutput}
\begin{Sinput}
> mpi.get.processor.name()
\end{Sinput}
\begin{Soutput}
[1] "node059"
\end{Soutput}
\end{Schunk}

\subsubsection{Process spawning and communication}

In Rmpi it is easy to spawn R slaves and use them as workhorses.

%% TODO: explain here or in MPI section more about comms ranks etc.
%% mpi.send recv bcast etc. explained here

\begin{description}

\item[\texttt{mpi.spawn.Rslaves(Rscript =
    system.file(nslaves =
    mpi.universe.size(), ...)}] spawns \texttt{nslaves} number of R
  workhorses to those hosts automatically chosen by MPI. For other
  arguments represented by \ldots to this function I refer to
  \cite{yu06Rmpi}.
\item[\texttt{mpi.close.Rslaves(dellog = TRUE, comm = 1)}] closes
  previously spawned R slaves and returns 1 if succesful.
\item[\texttt{mpi.comm.size()}] returns the total number of members in
  a communicator.
\item[\texttt{mpi.comm.rank()}] returns the rank (identifier) of the
  process in a communicator.
\item[\texttt{mpi.remote.exec(cmd, ..., comm = 1, ret = TRUE)}]
  executes a command \texttt{cmd} on R slaves with \ldots arguments to
  \texttt{cmd} and return executed results if \texttt{TRUE}.
\end{description}

\begin{Schunk}
\begin{Sinput}
> mpi.spawn.Rslaves(nslaves = mpi.universe.size() - 1)
\end{Sinput}
\begin{Soutput}
	7 slaves are spawned successfully. 0 failed.
master (rank 0, comm 1) of size 8 is running on: node059 
slave1 (rank 1, comm 1) of size 8 is running on: node059 
slave2 (rank 2, comm 1) of size 8 is running on: node059 
slave3 (rank 3, comm 1) of size 8 is running on: node055 
slave4 (rank 4, comm 1) of size 8 is running on: node055 
slave5 (rank 5, comm 1) of size 8 is running on: node064 
slave6 (rank 6, comm 1) of size 8 is running on: node064 
slave7 (rank 7, comm 1) of size 8 is running on: node062 
\end{Soutput}
\begin{Sinput}
> mpi.comm.size()
\end{Sinput}
\begin{Soutput}
[1] 8
\end{Soutput}
\begin{Sinput}
> mpi.remote.exec(mpi.comm.rank())
\end{Sinput}
\begin{Soutput}
  X1 X2 X3 X4 X5 X6 X7
1  1  2  3  4  5  6  7
\end{Soutput}
\begin{Sinput}
> mpi.comm.rank()
\end{Sinput}
\begin{Soutput}
[1] 0
\end{Soutput}
\begin{Sinput}
> mpi.close.Rslaves()
\end{Sinput}
\begin{Soutput}
[1] 1
\end{Soutput}
\end{Schunk}

\subsubsection{Built-in high level functions}

Rmpi provides the following high level functions (this is only a selection): 

\begin{description}
\item[\texttt{mpi.apply(x, fun, ..., comm = 1)}] applies a function
  \texttt{fun} with additional arguments \ldots to a specific part of
  a vector \texttt{x}. The return value is of type list with the same
  length as of \texttt{x}. The length of
  \texttt{x} must not exceed the 
  number of R slaves spawned as each element of the vector is used
  exactly by one slave. To achieve some sort of load balancing please
  use the corresponding apply functions below.
\item[\texttt{mpi.applyLB(x, fun, ..., comm = 1)}] applies a function
  \texttt{fun} with additional arguments \ldots to a specific part of
  a vector \texttt{x}. There are a few more variants explained in
  \cite{yu06Rmpi}.
\item[\texttt{mpi.bcast.cmd(cmd = NULL, rank = 0, comm = 1)}]
  broadcasts a command \texttt{cmd} from the sender \texttt{rank} to
  all R slaves and evaluates it.
\item[\texttt{mpi.bcast.Robj(obj, rank = 0,comm = 1)}]
  broadcasts an R object \texttt{obj} from process rank \texttt{rank}
  to all other processes (master and slaves).
\item[\texttt{mpi.bcast.Robj2slave(obj, comm = 1)}] broadcasts an R
  object \texttt{obj} to all R slaves from the master process. 
\item[\texttt{mpi.parSim( ... )}] carries out a Monte Carlo simulation
  in parallel. For details on this function see the package manual
  (\cite{yu06Rmpi}) and
  the applications in chapter \ref{chap:options}.
\end{description}


\textbf{Example:} Using mpi.apply\newline
running on cluster@WU using the node.q -- the parallel environment was
started with SGE using 8 nodes

\begin{Schunk}
\begin{Sinput}
> n <- 8
> mpi.spawn.Rslaves(nslaves = n)
\end{Sinput}
\begin{Soutput}
	8 slaves are spawned successfully. 0 failed.
master (rank 0, comm 1) of size 9 is running on: node059 
slave1 (rank 1, comm 1) of size 9 is running on: node059 
slave2 (rank 2, comm 1) of size 9 is running on: node059 
slave3 (rank 3, comm 1) of size 9 is running on: node055 
slave4 (rank 4, comm 1) of size 9 is running on: node055 
slave5 (rank 5, comm 1) of size 9 is running on: node064 
slave6 (rank 6, comm 1) of size 9 is running on: node064 
slave7 (rank 7, comm 1) of size 9 is running on: node062 
slave8 (rank 8, comm 1) of size 9 is running on: node062 
\end{Soutput}
\begin{Sinput}
> x <- rep(n, n)
> rows <- mpi.apply(x, runif)
> X <- matrix(unlist(rows), ncol = n, byrow = TRUE)
> X
\end{Sinput}
\begin{Soutput}
          [,1]      [,2]      [,3]       [,4]      [,5]     [,6]      [,7]
[1,] 0.6775106 0.2118004 0.6072327 0.03405821 0.9282862 0.969057 0.6612636
[2,] 0.6775106 0.2118004 0.6072327 0.03405821 0.9282862 0.969057 0.6612636
[3,] 0.6775106 0.2118004 0.6072327 0.03405821 0.9282862 0.969057 0.6612636
[4,] 0.6775106 0.2118004 0.6072327 0.03405821 0.9282862 0.969057 0.6612636
[5,] 0.6775106 0.2118004 0.6072327 0.03405821 0.9282862 0.969057 0.6612636
[6,] 0.6775106 0.2118004 0.6072327 0.03405821 0.9282862 0.969057 0.6612636
[7,] 0.6775106 0.2118004 0.6072327 0.03405821 0.9282862 0.969057 0.6612636
[8,] 0.6775106 0.2118004 0.6072327 0.03405821 0.9282862 0.969057 0.6612636
          [,8]
[1,] 0.2767014
[2,] 0.2767014
[3,] 0.2767014
[4,] 0.2767014
[5,] 0.2767014
[6,] 0.2767014
[7,] 0.2767014
[8,] 0.2767014
\end{Soutput}
\begin{Sinput}
> mpi.close.Rslaves()
\end{Sinput}
\begin{Soutput}
[1] 1
\end{Soutput}
\end{Schunk}


\subsubsection{Other important functions}

To complete the set of important functions supplied by the Rmpi
package the following functions have to be explained:

\begin{description}
\item[\texttt{.PVM.gather(x, count = length(x), msgtag, group,
    rootginst = 0}] gathers data distributed on the nodes (x) to a
  specific process (mostly the root) into a single array. It performs
  a send of messages from each member of a group of processes. A
  specific process (the root) accumulates this messages into a single vector.
\item[\texttt{.PVM.scatter(x, count, msgtag, group, rootqinst = 0}]
  sends to each member of a group a partition of  a vector x from a
  specified member of the group (mostly the root) where \texttt{count}
  is an integer specifying the number of elements to be sent to each
  member. 
\end{description}

\subsubsection{conclusion}

%The \texttt{PVM.rapply()} example shown in this section followed the Single Program
%Multiple Data (SPMD) paradigm. Data is splitted into different parts
%which are sent to different processes. I/O is handled by a master
%process. When loading rpvm in an R session this session becomes the
%master process. Slaves can easily be spawned provided that there are
%working slave scripts available. A major disadvantage is that the rpvm
%package only has two higher-level function. One of them can be used
%for calculations. That means when using this package for HPC one has
%to deal with low-level message-passing which in turn provides high
%flexibility. New parallel functions can be constructed on the basis of
%the provided interface.

For further interface functions supplied by the Rmpi package, a more detailed
description and further examples please consult the package description
\cite{yu06Rmpi}.


%% subsection: The rpvm package 
\section{The rpvm Package}
\label{sec:rpvm}
Parallel virtual machine uses the message passing model and makes a
collection of computers appear as a single virtual machine (see Section
\ref{sec:PVM} for details).
The package \pkg{rpvm} (\cite{nali07rpvm}) provides an interface to
low level PVM functions and
a few high level parallel functions to R. It uses most of the
facilities provided by the PVM system which makes \pkg{rpvm} ideal for
prototyping parallel statistical applications in R. Generally, parallel
applications can be either written in compiled languages like C or
FORTRAN or can be called as R processes. The latter method is used in
this thesis and therefore a good selection of \pkg{rpvm}
functions are explained in this section to show how PVM and R can be
used together.

Provided functions are categorized as follows:

\begin{itemize}
\item Initialization and Status Queries
\item Process Spawning and Communication
\item Built-in High Level Functions
\item Other Important Functions
\end{itemize}  

\subsection{Initialization and Status Queries}

At first for using the PVM system \textit{pvmd3} has to be
booted. This can be done via the command line using the \code{pvm}
command (see pages 22 and 23 in \cite{geist94pvm}) or directly within
R after loading the \pkg{rpvm} package using \code{.PVM.start.pvmd()}
explained in this section.

\pkg{rpvm} functions for managing the virtual machine:

\begin{description}
\item[\code{.PVM.start.pvmd()}] boots the \textit{pvmd3} daemon. The
  currently running R session becomes the master process. 
\item[\code{.PVM.add.hosts(hosts)}] takes a vector of hostnames to
  be added to the current virtual machine. The syntax of the
  hostnames is similiar to the lines of a pvmd hostfile (for details
  see the man page of \textit{pvmd3}). 
\item[\code{.PVM.del.hosts()}] simply deletes the given hosts from
  the virtual machine configuration.
\item[\code{.PVM.config()}] returns information about the present
  virtual machine.
\item[\code{.PVM.exit()}] tells the PVM daemon that this process
  leaves the parallel environment.
\item[\code{.PVM.halt()}] shuts down the entire PVM system and exits
  the current R session.
\end{description}
  
When using a job queueing system like he Sun Grid Engine (SGE) to boot
the PVM parallel environment the developer is not engaged with
setting up and booting the environment anymore (see
appendix \ref{app:gridengine} on how to do this).

Example~\ref{ex:rpvm-init} shows how the configuration of the parallel
environment can be obtained. First it returns the hosts connected to
the parallel virtual machine. Finally the
parallel environment is stopped. 

\begin{Example} Query status of PVM \newline
running on cluster@WU using the node.q---the parallel environment was
started with SGE using 8 nodes

\begin{Schunk}
\begin{Sinput}
> library("rpvm")
> set.seed(1782)
> .PVM.config()
\end{Sinput}
\begin{Soutput}
  host.id    name    arch speed
1  262144 node015 LINUX64  1000
2  524288 node003 LINUX64  1000
3  786432 node053 LINUX64  1000
4 1048576 node001 LINUX64  1000
\end{Soutput}
\begin{Sinput}
> .PVM.exit()
\end{Sinput}
\end{Schunk}
\caption{Query status of PVM}
\label{ex:rpvminit}
\end{Example}

\subsection{Process Spawning and Communication}

The package \pkg{rpvm} uses the master-slave paradigm where one
process is the master task and the others are slave tasks. \pkg{rpvm}
provides a routine to spawn R slaves but these slaves cannot be used
interactively like the slaves in \pkg{Rmpi}. The spawned R slaves
source an R script which contains all the necessary function calls to
set up communication and carry out the computation and after
processing terminate.
PVM uses task IDs (tid---a positive integer for identifying a task)
and tags for communication (see
also the fundamentals of message passing in
Section~\ref{sec:messagepassing}).

\begin{description}
\item[\code{.PVM.spawnR(slave, ntask = 1, ...}] spawns \code{ntask}
  copies 
  of an executable or \code{slave} R processes. There are more
  parameters indicated by the \ldots (we refer to
  \cite{nali07rpvm}). The \code{tids} of the successfully spawned R
  slaves are returned.
\item[\code{.PVM.mytid()}] returns the \code{tid} of the calling
  process.
\item[\code{.PVM.parent()}] returns the \code{tid} of the parent
  process that spawned the calling process.
\item[\code{.PVM.siblings()}] returns the \code{tid} of the processes
  that were spawned in a single spawn call.
\end{description}

\subsection{Built-in high level functions}

\pkg{rpvm} provides two high level functions: 

\begin{description}
\item[\code{PVM.rapply(X, FUN = mean, NTASK = 1))}] Apply a function
  \texttt{FUN} to the rows of a matrix \texttt{X} in
  parallel using \texttt{NTASK} tasks.
\item[\code{PVM.options(option, value)}] Get or set values of libpvm
  options (for details see \cite{nali07rpvm} and \cite{geist94pvm}).
\end{description}

\begin{Example} Using PVM.rapply\newline
running on cluster@WU using the node.q -- the parallel environment was
started with SGE using 8 nodes

\begin{Schunk}
\begin{Sinput}
> n <- 8
> X <- matrix(rnorm(n * n), nrow = n)
> round(X, 3)
\end{Sinput}
\begin{Soutput}
       [,1]   [,2]   [,3]   [,4]   [,5]   [,6]   [,7]   [,8]
[1,] -0.200 -0.183  0.560  1.286  0.468  0.502  0.874 -0.778
[2,] -1.371  0.484 -0.498  1.788  0.534 -0.566  0.152 -1.307
[3,]  1.041  0.484  0.399  0.580  0.586 -0.660  1.833 -1.405
[4,] -1.117 -0.893  0.408 -1.612  0.486  0.644  0.422 -1.639
[5,]  1.397 -0.237 -1.287 -0.122 -1.076  0.225 -0.047  0.020
[6,] -0.046  0.537 -1.287 -0.089  0.564  2.671 -0.715 -0.901
[7,]  1.085  0.706 -0.034  0.929  0.057 -2.402 -1.233  1.135
[8,]  0.605 -0.076 -0.554  1.385 -0.436  0.249  0.338  1.369
\end{Soutput}
\begin{Sinput}
> PVM.rapply(X, sum, n)

%% subsection: The snow package 
\section{The snow Package}
\label{sec:snow}
The aim of simple network of workstations
(\pkg{snow}---\cite{rossini03snow}, \cite{tierney07snow}) is to
provide a simple parallel computing environment in R. To make a
collection of computers to appear as a virtual cluster in R 
three different message passing environments can be used:

\begin{itemize}
\item PVM via R package \pkg{rpvm} (see section \ref{sec:rpvm})
\item MPI via R package \pkg{Rmpi} (see section \ref{sec:Rmpi})
\item SOCK via TCP sockets
\end{itemize}

The details of the mechanism used and how it implemented are hidden
from the high level user. As the name suggests it should be simple to
use.

After setting up a virtual cluster developing parallel R functions
can be achieved via a standardized interface to the computation
nodes.

Moreover, when using \pkg{snow} one can rely on a good handful of
high level functions. This makes it rather easy to use the underlying
parallel computational engine.
Indeed \pkg{snow} uses existing interfaces to R namely \pkg{Rmpi} when
using MPI (see Section~\ref{sec:Rmpi}), \pkg{rpvm} when using PVM (see
Section~\ref{sec:rpvm}) and a new possibility of message passing namely
TCP sockets, which is a rather simple way of achieving communication
between nodes (in most applications this is not the optimal way).
What follows is a description of high level functions supplied by the
\pkg{snow} package. %They are assigned to one of the topics:

%\begin{itemize}
%\item Initialization
%\item Built-in High Level Functions
%\item Fault Tolerance
%\end{itemize}

\subsection{Initialization}

Initializing a \pkg{snow} cluster is rather easy if the system is
prepared accordingly. When using MPI (achieved through \pkg{Rmpi}) a
LAM/MPI environment has to be booted prior starting the virtual
cluster (see Section~\ref{sec:Rmpi}). Is PVM the method of choice the
\pkg{rpvm} package must be available and an appropriate PVM has to be
started (see Section~\ref{sec:rpvm}). For both MPI and PVM the
parallel environment 
can be configured through a grid engine (see Appendix
\ref{app:gridengine}). TCP sockets can be set up directly using the
package. MPI or PVM offer the possibility to query the status of the
parallel environment. This can be done using the functions supplied
from the corresponding package.

\subsubsection{Management Functions}
\begin{description}
\item[\code{makecluster(spec, type = getClusterOption("type"))}]
  starts a cluster of type \code{type} with \code{spec} numbers of
  slaves. If the cluster is of connection type SOCK then \code{spec}
  must be a character vector containing the hostnames of the
  slavenodes to join the cluster. The return value is a list
  containing the cluster specifications. This object is necessary in
  further function calls.
\item[\code{stopCluster(cl)}] stops a cluster specified in \code{cl}.
\end{description}

Example~\ref{ex:snowstartstop} shows how a virtual cluster can be
created using MPI. 

\begin{Example} Start/stop cluster in \pkg{snow}
\begin{Schunk}
\begin{Sinput}
> library("snow")
> set.seed(1782)
> n <- 8
> cl <- makeCluster(n, type = "MPI")
\end{Sinput}
\begin{Soutput}
	8 slaves are spawned successfully. 0 failed.
\end{Soutput}
\begin{Sinput}
> stopCluster(cl)
\end{Sinput}
\begin{Soutput}
[1] 1
\end{Soutput}
\end{Schunk}
\label{ex:snowstartstop}
\end{Example}

\subsection{Built-in High Level Functions}

\pkg{snow} provides a good handful of high-level functions. They can
be used as building blocks for further high level routines.

\subsubsection{High Level Functions}
\begin{description}
\item[\code{clusterEvalQ(cl, expr)}] evaluates an R expression
  \code{expr} on
  each cluster node provided by \code{cl}. 
\item[\code{clusterCall(cl, fun, ...)}] calls a function
  \code{fun} with arguments \ldots{} on each node found in \code{cl}
  and returns a list of the results.
\item[\code{clusterApply(cl, x, fun, ...)}] applies a function
  \code{fun} with additional arguments \ldots{} to a specific part of
  a vector \code{x}. The return value is of type list with the same
  length as of \code{x}. The length of
  \code{x} must not exceed the 
  number of R slaves spawned as each element of the vector is used
  exactly by one slave. To achieve some sort of load balancing please
  use the corresponding apply functions below.
\item[\code{clusterApplyLB(cl, x, fun, ...)}] is a load balancing
  version of \\ \code{clusterApply()} which applies a function
  \code{fun} with additional arguments \ldots{} to a specific part of
  a vector \code{x} with the difference that the length of
  \code{x} can exceed the number of cluster nodes. If a node
  finished with the computation the next job is placed on the
  available node. This is repeated until all jobs have completed.
\item[\code{clusterExport(cl, list)}] broadcasts a list of global
  variables on the master (\code{list}) to all slaves.
\item[\code{parApply(cl, x, fun, ...)}] is one of the parallel
  versions of the \code{apply} functions available in R. We refer to
  the package documentation (\cite{tierney07snow}) for further details.
\item[\code{parMM(cl, A,B)}] is a simple parallel implementation of
  matrix multiplication. 
\end{description}

Example~\ref{ex:snowapply} shows the use of \code{clusterApply()} on a
virtual cluster of 8 nodes using MPI as communication layer. Like in
Example~\ref{ex:Rmpi3} a vector of $n$ random numbers is generated on
each of the $n$ slaves and are returned to the master as a
list (each list element representing one row). Finally an $n  \times n$
matrix is formed and printed. The output of the matrix shows again for
each row the same random numbers. This is because of the fact, that
each slave has the same seed. This problem is treated more specifically in
Chapter~\ref{chap:options}. For more information about parallel random
number generators see the descriptions of the packages \pkg{rsprng}
and \pkg{rlecuyer} in Section~\ref{sec:otherpackages}. 

\begin{Example} Using high level functions of snow
\begin{Schunk}
\begin{Sinput}
> n <- 8
> cl <- makeCluster(n, type = "MPI")
\end{Sinput}
\begin{Soutput}
	8 slaves are spawned successfully. 0 failed.
\end{Soutput}
\begin{Sinput}
> x <- rep(n, n)
> rows <- clusterApply(cl, x, runif)
> X <- matrix(unlist(rows), ncol = n, byrow = TRUE)
> round(X, 3)
\end{Sinput}
\begin{Soutput}
      [,1]  [,2]  [,3]  [,4]  [,5]  [,6]  [,7]  [,8]
[1,] 0.691 0.128 0.188 0.206 0.119 0.171 0.873 0.238
[2,] 0.691 0.128 0.188 0.206 0.119 0.171 0.873 0.238
[3,] 0.691 0.128 0.188 0.206 0.119 0.171 0.873 0.238
[4,] 0.691 0.128 0.188 0.206 0.119 0.171 0.873 0.238
[5,] 0.691 0.128 0.188 0.206 0.119 0.171 0.873 0.238
[6,] 0.691 0.128 0.188 0.206 0.119 0.171 0.873 0.238
[7,] 0.691 0.128 0.188 0.206 0.119 0.171 0.873 0.238
[8,] 0.691 0.128 0.188 0.206 0.119 0.171 0.873 0.238
\end{Soutput}
\end{Schunk}
\label{ex:snowapply}
\end{Example}

\subsection{Fault Tolerance}

Providing fault tolerance, computational reproducibility and dynamic
adjustment of the cluster configuration is required for practical
real-world parallel applications according to
\cite{sevcikova04pragmatic}. Failure detection and recovery, dynamic
cluster resizing and the possibility to obtain intermediate results of
a parallel computation is implemented in the package \pkg{snowFT}
(\cite{sevcikova04snowFT}). A detailed introduction to fault tolerance
in statistical simulations can be found in
\cite{sevcikova04simulations}. As we pointed out in
Section~\ref{sec:MPI} and Section~\ref{sec:PVM} MPI does not provide
tools for implementing fault tolerance and therefore package
\pkg{rpvm} is required for \pkg{snowFT}. For more details on the fault
tolerant version of \pkg{snow} we refer to the package documentation
(\cite{sevcikova04snowFT}).

\subsection{Conclusion}

The routines available in package \pkg{snow} are easy to understand
and use, provided that there is a corresponding communication
environment set up. Generally, the user need not know the underlying
parallel infrastructure, she just ports her sequential code so that it
uses the functions supplied by \pkg{snow}. All in all as the title
suggests simple network of workstations is simple to get started with
and is simple with respect to the possibilities of parallel
computations. 

For further interface functions supplied by the \pkg{snow} package, a
more detailed 
description and further examples please consult the package description
\cite{tierney07snow}.


%% subsection: The paRc package 
\section{paRc---PARallel Computations in R}
\label{sec:paRc}

In the course of this thesis a package called \pkg{paRc}
(\cite{theussl07paRc}) has been developed with the aim
to evaluate performance of parallel applications and to show how
interfacing high performance applications written in C can be done
using OpenMP~(see Section~\ref{sec:OpenMP}).

The package \pkg{paRc} contains interface functions to the OpenMP library
and provides high level interface functions to a few C implementations
of parallel applications using OpenMP (e.g., matrix multiplication---see
Chapter~\ref{chap:matrix}).
Furthermore, it supplies a benchmark environment for performance
evaluation of parallel programs and a framework for pricing options
with parallel Monte Carlo simulation (see
Chapter~\ref{chap:options}).

\pkg{paRc} can be obtained from \url{R-Forge.R-project.org}---the
R-project community service. To install this package directly within R
call \\ \code{install.packages("paRc", repos = "R-Forge.R-project.org")}.

To properly install this package you need either the Intel compiler
with version 9.1 or newer (the Linux compiler is free for
non-commercial use) or the GNU~C compiler with version 4.2 or
newer. They are known to support OpenMP.

Examples in this section are produced on a bignode of
cluster@WU. Big-nodes provide a shared memory platform with up to 4
CPUs. Shared memory platforms are necessary for running parallel
OpenMP applications.

\subsection{OpenMP Interface Functions}

The user is provided with a few interface functions to the OpenMP
library. They are used to query the internal variables of the compiled
parallel environment or to change them.

\subsubsection{OpenMP Routines}
\begin{description}
\item[\texttt{omp\_get\_num\_procs()}] returns the number of threads
  available to the program.
\item[\texttt{omp\_set\_num\_threads()}] sets the number of threads to be
  used in subsequent parallel executions.
\item[\texttt{omp\_get\_max\_threads()}] gets the number of threads to be
  used in subsequent parallel executions.
\end{description}

\subsubsection{OpenMP Specific Environment Variables}

Moreover, environment variables can affect the runtime behavior of
OpenMP programs. These environment variables are~(\cite{openMP05}):

\begin{description}
\item[\texttt{OMP\_NUM\_THREADS}] sets the number of threads to use in
  parallel regions of OpenMP programs. 
\item[\texttt{OMP\_SCHEDULE}] sets the runtime schedule type and
  chunk size.
\item[\texttt{OMP\_DYNAMIC}] defines whether dynamic adjustments of threads
  should be used in parallel regions.
\item[\texttt{OMP\_NESTED}] enables or disables nested parallelism.
\end{description}

Example~\ref{ex:paRcOMP} shows the use of the OpenMP library calls in
R. First the number of available processors is queried. Then the
number of threads a parallel application may use is set to 2. With the
last call the current available CPUs to a parallel program is queried.
\begin{Example} OpenMP function calls using \pkg{paRc}
\label{ex:paRcOMP}
\begin{Schunk}
\begin{Sinput}
> library("paRc")
> omp_get_num_procs()
\end{Sinput}
\begin{Soutput}
[1] 4
\end{Soutput}
\begin{Sinput}
> omp_set_num_threads(2)
> omp_get_max_threads()
\end{Sinput}
\begin{Soutput}
[1] 2
\end{Soutput}
\end{Schunk}
\end{Example}

\subsection{High Level OpenMP Functions}

\pkg{paRc} provides the following high level OpenMP function: 

\begin{description}
\item[\texttt{omp\_matrix\_multiplication(X, Y, n\_cpu = 1)}] multiplies the matrix
  \code{X} \\with matrix \code{Y} using \code{n\_cpu} numbers of
  processors.
\end{description}

\subsection{Benchmark Environment}

\pkg{paRc} provides a benchmark environment for measuring the
performance of parallel programs. Two main functions exist in this
context---one for creating a benchmark object and one for running the
benchmark described by the object.

\subsubsection{Class \class{benchmark}}

An S3 object (\cite{chambers91sms}) of class \class{benchmark} contains all
the necessary information to run a benchmark. The elements in the
object are

\begin{description}
\item[task] is a character string defining the task of the
  benchmark. Currently, the 
  following tasks are implemented:
  \begin{itemize}
  \item matrix multiplication
  \item Monte Carlo simulation
  \end{itemize}
\item[data] is a list containing the parameters and data to properly
  run the task.
\item[type]defines the parallel programming model used to run the
  benchmark. Currently, the following types are implemented (not
  necessarily all of them are available for each task):
  \begin{itemize}
  \item OpenMP---C interface calls provided by \pkg{paRc},
  \item MPI---implementation in \pkg{paRc} using \pkg{Rmpi} for
    communication,
  \item PVM---implementation in \pkg{paRc} using \pkg{rpvm} for
    communication,
  \item snow-MPI---implementation in \pkg{snow} using \pkg{Rmpi} for
    communication,
  \item and snow-PVM---implementation in \pkg{snow} using \pkg{rpvm} for
    communication.
  \end{itemize}
\item[cpu\_range] contains a vector of integers representing the number
  of CPUs for th corresponding benchmark run.
\item[is\_parallel] is a logical \code{TRUE} or \code{FALSE} whether
  the benchmark
  contains parallel tasks or not.
\item[runs] is an integer defining the number of repetitions of the
  benchmark.
\end{description}

\subsubsection{Main Routines}

These are the main routines for benchmarking parallel applications:

\begin{description}
\item[\code{create\_benchmark(task, data, type, cpu\_range, ...)}]
  defines a benchmark object using a specific \code{task} and the
  corresponding \code{data}. The \code{type} refers to the serial or
  parallel paradigm to use. The \code{cpu\_range} specifies the range
  of CPUs to use for this benchmark. 
\item[\code{run\_benchmark(x)}] takes a benchmark object as argument
  and carries out the defined benchmark. An object of type
  \class{benchmark\_results} is returned containing the
  results of the benchmark.
\end{description}

\subsubsection{Results}

Results of the benchmark are stored in a data frame. It is an object
of class \class{benchmark\_results} and inherits from class
\class{data.frame}. Each row represents a run of the dedicated
benchmark and for each run the following data is saved in addition to
the data defined in the \class{benchmark} object:
\begin{description}
%\item[\code{task}] is a character representing the task to benchmark.
%\item[\code{type}] is a character representing the type of the parallel
%  programming model to be used. 
\item[\code{time\_usr}, \code{time\_sys}, \code{time\_ela}] contain the
  measured runtimes 
  (measured with the R function \code{system.time()}). 
%\item[\code{n\_cpu}] is an integer representing the number of CPUs used for
%  the corresponding run.
%\item[\code{is.parallel}] is a logical \code{TRUE} or \code{FALSE} whether the
%  contains parallel tasks or not.
\item[\code{run}] is an integer representing the number of the
  benchmark run when using a specific task with a given number of CPUs
  and a given programming model.
\end{description} 

Example~\ref{ex:benchrun} runs an OpenMP benchmark which returns an
object of class \class{benchmark\_results}.

% The task to run with 1 and 2 processors is ``matrix
%multiplication''. The programming models to use are ``OpenMP'' and
%``MPI''. For every CPU count and programming model the benchmark is to
%be run 3 times. 

%\begin{Example} A sample data frame
%\label{ex:sampleframe}
%<<echo=FALSE>>=
% expand.grid(task="matrix multiplication",type=c("OpenMP","MPI"),time_usr=NA, time_sys=NA, time_ela=NA, n_cpu=1:2, runs=1:3, is.parallel=TRUE)
%@

%\end{Example}

\subsubsection{Extractor and Replacement Functions}

The following routines are for handling a benchmark object. They
extract or replace the values in the benchmark \code{x}.

\begin{itemize}
\item \code{benchmark\_task(x)}
\item \code{benchmark\_data(x)}
\item \code{benchmark\_type(x)}
\item \code{benchmark\_cpu\_range(x)}
\end{itemize}

The following routines supply extra information about the benchmark object
or the benchmark environment.

\begin{description}
\item[\code{benchmark\_is\_parallel}] returns \code{TRUE} if the benchmark
  contains a parallel function to apply. 
\item[\code{benchmark\_tasks}] returns the tasks which are possible to run
  with the benchmark environment.
\item[\code{benchmark\_types}] returns the types of available serial or
  parallel paradigms to run with the benchmark environment.
\end{description}

\subsubsection{Generic Functions}

Generic functions provide methods for different objects. In \pkg{paRc}
a generic function for calculating the speedup is provided:

\begin{description}
\item[\code{speedup(x)}] is a generic function taking an object as
  an argument. Currently there are two methods implemented namely
  \code{speedup.numeric} and \code{speedup.benchmark\_results}. The methods
  calculate the speedup as it is presented in
  Equation~\ref{eq:speedup}.
\end{description}

\subsubsection{S3 Methods}

The following S3 methods are provided for the benchmark environment:

\begin{description}
\item[\code{print.benchmark}] prints objects of class \class{benchmark}
\item[\code{plot.benchmark\_results}] supplies a plot method for comparing
  benchmark results.
\item[\code{speedup.default}] returns an error message that there is
  no default method. 
\item[\code{speedup.numeric}] returns the speedups calculated from a
  vector of type \class{numeric}. The reference execution time is  the
  first element in the vector. The return value is a vector of type
  \class{numeric}. 
\item[\code{speedup.benchmark\_results}] calculates the speedups from a given
  object of \\class~\class{benchmark\_results} and the results are
  returned as a vector of 
  type~\class{numeric}. 
\end{description}

\begin{Example} Running a benchmark using OpenMP
\label{ex:benchrun}
\begin{Schunk}
\begin{Sinput}
> n <- 1000
> max_cpu <- omp_get_num_procs()
> dat <- list()
> dat[[1]] <- dat[[2]] <- n
> dat[[3]] <- function(x) {
+     runif(x, -5, 5)
+ }
> bm <- create_benchmark(task = "matrix multiplication", 
+     data = dat, type = "OpenMP", parallel = TRUE, 
+     cpu_range = 1:max_cpu)
> bm
\end{Sinput}
\begin{Soutput}
A parallel benchmark running task: matrix multiplication - OpenMP
\end{Soutput}
\begin{Sinput}
> bmres <- run_benchmark(bm)
> bmres
\end{Sinput}
\begin{Soutput}
                   task   type n_cpu time_usr time_sys
2 matrix multiplication OpenMP     1    7.421    0.056
3 matrix multiplication OpenMP     2    7.268    0.060
4 matrix multiplication OpenMP     3    7.666    0.056
5 matrix multiplication OpenMP     4    7.955    0.059
  time_ela is_parallel run
2    7.494        TRUE   1
3    3.697        TRUE   1
4    2.697        TRUE   1
5    2.110        TRUE   1
\end{Soutput}
\begin{Sinput}
> speedup(bmres)
\end{Sinput}
\begin{Soutput}
  OpenMP   OpenMP   OpenMP   OpenMP 
1.000000 2.027049 2.778643 3.551659 
\end{Soutput}
\end{Schunk}
\end{Example}

Running a dedicated benchmark is shown in
Example~\ref{ex:benchrun}. The object \code{bm} contains all the
information to carry out the benchmark. The task to run is a matrix
multiplication of two $1000 \times 1000$ matrices. The parallel
paradigm to use is OpenMP using 1 up to the maximum of CPUs available
on the machine. The resulting data frame is printed and the speedup is
calculated.

\subsection{Environment for Option Pricing}

In package \pkg{paRc} a framework for pricing financial derivatives,
in particular options, is available. The framework is built around the
main function namely Monte Carlo simulation and its parallel
derivative. 

\subsubsection{Class \class{option}}

An S3 object of class \class{option} contains all
the necessary information for pricing an option. The list elements are

\begin{description}
\item[underlying] is a numeric vector containing three elements to
  describe a stock:
  \begin{itemize}
  \item the return $\mu$,
  \item the volatility $\sigma$,
  \item and the present value of the stock.
  \end{itemize}
\item[strike\_price] is a numeric defining the strike price of the option.
\item[maturity] is a numeric defining the time until the option expires.
\item[type] is a character representing the type of the option. There
  are two possibilities---either a \code{"Call"} option, or a
  \code{"Put"} option.
\item[kind] contains a character representing the class of the
  option. Only options of type \code{"European"} can be priced at
  the time of this writing. 
\item[position] is a character string marking the position of the
  investor. This can either be \code{"long"} or \code{"short"}.
\end{description}


\subsubsection{Main Routines}

These are the main routines for pricing an option:

\begin{description}
\item[\code{define\_option(underlying, strike\_price, maturity, ...)}]
  defines an \\object of class \class{option}. Further arguments
  represented by the \ldots{} are \code{type} (with default
  \code{"Call"}), \code{class} (default \code{"European"}) and
  \code{position} (default \code{"long"}).
\item[\code{Black\_Scholes\_price(x)}] takes an object of class
  \class{option} as argument and returns the analytical solution of
  the Black Scholes differential equation (the price of the
  option). This works only for European options (see
  Section~\ref{sec:blackscholes} for details).
\item[\code{Monte\_Carlo\_simulation(x, r, path\_length, n\_paths,
    ...)}] carries out a Monte Carlo simulation pricing 
  an option given by \code{x}. Further arguments are the risk free
  yield \code{r}, the length of a Wiener path \code{path\_length}, the
  number of 
  simulated paths \code{n\_paths}, the number of simulation runs
  \code{n\_simulations} (with default \code{50}) and if antithetic
  variance reduction should be used (\code{antithetic = TRUE}).
\item[\code{mcs\_Rmpi(n\_cpu = 1, spawnRslaves =
    FALSE, ...)}] is the same
  like the serial version above (\code{Monte\_Carlo\_simulation()})
  but uses parallel generation of
  option prices. Additional arguments have to be provided naming the
  number of CPUs (\code{n\_cpu})
  and if R slaves are to be spawned respectively (\code{spawnRslaves}).
\end{description}

\subsubsection{Extractor and Replacement Functions}

The following routines are for handling an object of class
\class{option}. They
extract or replace the values in the option \code{x}.

\begin{itemize}
\item \code{maturity(x)}
\item \code{strike\_price(x)}
\item \code{underlying(x)}
\item \code{option\_type(x)}
\item \code{option\_class(x)}
\item \code{position(x)}
\item \code{price\_of(x)}
\end{itemize}

\subsubsection{S3 Methods}

The following S3 methods are provided for the option pricing environment:

\begin{description}
\item[\code{print.option}] defines the print method for class \class{option}.
\item[\code{plot.option}] plots the payoff of the given \class{option}.
\end{description}

\begin{Example} Handling class \class{option}
\label{ex:option}
\begin{Schunk}
\begin{Sinput}
> european <- define_option(c(0.1, 0.4, 100), 100, 
+     1/12)
> underlying(european)
\end{Sinput}
\begin{Soutput}
 mean    sd value 
  0.1   0.4 100.0 
\end{Soutput}
\begin{Sinput}
> option_type(european)
\end{Sinput}
\begin{Soutput}
[1] "Call"
\end{Soutput}
\begin{Sinput}
> option_class(european)
\end{Sinput}
\begin{Soutput}
[1] "European"
\end{Soutput}
\begin{Sinput}
> strike_price(european)
\end{Sinput}
\begin{Soutput}
[1] 100
\end{Soutput}
\begin{Sinput}
> maturity(european)
\end{Sinput}
\begin{Soutput}
[1] 0.08333333
\end{Soutput}
\begin{Sinput}
> underlying(european) <- c(0.2, 0.5, 100)
> underlying(european)
\end{Sinput}
\begin{Soutput}
 mean    sd value 
  0.2   0.5 100.0 
\end{Soutput}
\begin{Sinput}
> price_of(european) <- Black_Scholes_price(european, 
+     r = 0.045)
> european
\end{Sinput}
\begin{Soutput}
A Call option with strike price 100
expiring in 30 days
Call price:  5.93155761968948
\end{Soutput}
\end{Schunk}
\end{Example}

In Example~\ref{ex:option} the variable \code{european} is assigned an
object of class \class{option}. It is defined as an European option
with strike price 100 and maturity 1/12 (30 days). The underlying has a
present value of 100, a return of 0.1 and volatility of 0.4. Invoking the print
method returns a short summary of the object. The rest
of the example shows the use of different extractor functions and an
replacement function. Figure~\ref{fig:payoff} shows the payoff of the
defined option.

\begin{figure}[t]
\centering
\includegraphics{section_paRc-005}
\label{fig:payoff}
\caption{Payoff of a European call option with price 5.93}
\end{figure}

\subsection{Other Functions}

To complete the set of functions supplied by package \pkg{paRc} the
following function has to be explained:

\begin{description}
\item[\code{serial\_matrix\_multiplication(X,Y)}] takes the matrices \code{X} and
  \code{Y} as arguments and performs a serial matrix
  multiplication. This function calls a C routine providing a
  non-optimized version of the serial matrix multiplication (see
  Chapter~\ref{chap:matrix} for more details).  
\end{description}


\section{Other Packages Providing HPC Functionality}

There are a few other packages which supply the user with parallel
computing functions or extend other high performance computing
packages. In this section a short description of each of these
packages is given.

\begin{description}
\item[rsprng] (\cite{li07rsprng}) is an R interface to SPRNG (Scalable Parallel Random
  Number Generators---\cite{mascagni00ssl}). SPRNG is a package for parallel
  pseudo random number generation with the aim to be easy to use on a
  variety of architectures, especially in large-scale parallel Monte
  Carlo applications.
\item[RScaLAPACK] (\cite{samatova05RSca} and \cite{yoginath05rhp})
  uses the high performance ScaLAPACK library
  (\cite{dongarra97sus})for linear algebra computations. ScaLAPACK is
  a library of high performance linear algebra routines which makes
  use of message passing to run on distributed memory machines. Among
  other routines it provides functionality to solve systems of linear equations, linear
  least squares problems, eigenvalue problems and singular value
  problems. ScaLAPACK is an acronym for Scalable Linear Algebra
  PACKage or Scalable LAPACK. \cite{samatova06hps} have shown how to
  use \pkg{RScaLAPACK} in biology and climate modelling and analized
  its performance.
\item[papply] (\cite{currie05papply}) implements a similar interface
  to lapply and apply but distributes the processing evenly among the
  nodes of a cluster. \pkg{papply} uses the package \pkg{Rmpi} (see
  Section~\ref{sec:Rmpi}) for
  communication but supports more error messages for debugging.
\item[biopara] (\cite{lazar06biopara}) allows users to distribute
  execution of large problems over multiple machines. It uses R socket
  connections (TCP) for communication. \pkg{biopara} supplies two high
  level functions, one for distributing work (represented by function
  and its arguments) among ``workers'' and the other for doing
  parallel bootstrapping like the native R function boot(). 
\item[taskPR] (\cite{samatova04taskPR}) provides a parallel execution environment on
  the basis of TCP or MPI-2.
\end{description}

\section{Conclusion}

