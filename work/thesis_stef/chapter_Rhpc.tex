\chapter{High Performance Computing and R}
\label{chap:Rhpc}
\section{Introduction}

This chapter provides an overview of the capabilities of R
(\cite{Rcore07R}) in the area 
of high performance computing. A short description of the
software package R is given at the beginning of this
chapter. Subsequently extensions to the base environment 
(so called packages) which provide high performance computation
functionality to 
R are going to be explained. Among these extensions there is the
package called \pkg{paRc}, which was developed in the course of
this thesis.

Examples shown in this chapter have been produced on cluster@WU~(see
Section~\ref{sec:hardwaresoftware} for details).  

\section{The R environment}

R is an integrated suite of software facilities for data manipulation,
calculation and graphical display. R is open source originally
developed by Ross Ihaka and Robert Gentleman
(\cite{ihaka96rld}). Since 1997 a group of scientists (the
``R Core Team'') is responsible for the development of the R-project
and has write access to the source code. R has a homepage
which can be found on \url{http://www.R-project.org}. Sources, binaries
and documentation can be obtained from CRAN, the Comprehensive R
Archive Network (\url{http://cran.R-project.org}). Among other things
R has (\cite{Rcore07R}) 
\begin{itemize}
\item an effective data handling and storage facility,
\item a suite of operators for calculations on arrays, in particular matrices,
\item a large, coherent, integrated collection of intermediate tools
  for data analysis,
\item graphical facilities for data analysis and display either
  directly at the computer or on hardcopy, and
\item a well developed, simple and effective programming language
  (called `R') which includes conditionals, loops, user defined
  recursive functions and input and output facilities. (Indeed most of
  the system supplied functions are themselves written in the R
  language.)
\end{itemize}

R is not only an environment for statistical computing and graphics
but also a freely available high-level language for programming. It
can be extended by standardized collections of code called
``packages''. So developers and statisticians around the world can
participate and provide optional code to the base R environment.
Developing and implementing new methods of data analysis can therefore
be rather easy to achieve (for more information about R and on how R can
be extended see \cite{hornik07Rfaq} and \cite{Rcore07Ext}). 


Now, as datasets grow bigger and bigger and algorithms become more and
more complex, R has to be made ready for high performance
computing. Indeed, R is already prepared through a few extensions
explained in the 
subsequent chapters. The packages mentioned in this chapter can be
obtained from CRAN except \pkg{paRc} which can be obtained from
R-Forge (\url{http://R-Forge.R-project.org}), a platform for
collaborative software development for the R community
(\cite{theussl07R-Forge}).

%% subsection: The Rmpi package 
\section{The Rmpi Package}
\label{sec:Rmpi}
The Message Passing Interface (MPI) is a set of library interface
standards for message passing and there are many implementations using
these standards (see also Section~\ref{sec:MPI}).
\pkg{Rmpi} is an interface to MPI (\cite{yu02Rmpi} and
\cite{yu06Rmpi}). As 
of the time of this writing \pkg{Rmpi} uses
the LAM implementation of MPI. For process spawning the standard
MPI-1.2 is required which is available in the LAM/MPI implementation 
as LAM/MPI (version 7.1.3) supports a large portions of the MPI-2
standard. This is necessary if one likes to use interactive spawning
of R processes. With MPI versions prior to MPI-1.2 separate R
processes have to be created by hand using \code{mpirun} (part of many
MPI implementations) for example.

\pkg{Rmpi} contains a lot of low level interface functions to the MPI
C-library. 
Furthermore, a handful of high level functions are supplied. A
selection of routines is going to be presented in this section arranged
into the following topics:

\begin{itemize}
\item Initialization and Status Queries
\item Process Spawning and Communication
\item Built-in High Level Functions
\item Other Important Functions
\end{itemize}  

A windows implementation of this package (which uses MPICH2)
is available from~\url{http://www.stats.uwo.ca/faculty/yu/Rmpi}.

\subsection{Initialization and Status Queries}

The LAM/MPI environment has to be booted prior to using any
message passing library functions. One possibility is to use the
command line, the other is to load the \pkg{Rmpi} package. It automatically
sets up a (small---1 host) LAM/MPI environment (if the executables are
in the search path). 

When using the Sun Grid Engine (SGE) or other queueing systems to boot
the LAM/MPI parallel environment the developer is not engaged with
setting up and booting the environment anymore (see
appendix \ref{app:gridengine} on how to do this). On a cluster of
workstations this is the method of choice. 

\subsubsection{Management and Query Functions}

\begin{description}
\item[\code{lamhosts()}] finds the hostname associated with its node
  number.
\item[\code{mpi.universe.size()}] returns the total number of CPUs
  available to the MPI environment (ie. in a cluster or in a parallel
  environment started by the grid engine).
\item[\code{mpi.is.master()}] returns TRUE if the process is the
  master process or FALSE otherwise. 
\item[\code{mpi.get.processor.name()}] returns the hostname where the
  process is executed.
\item[\code{mpi.finalize()}] cleans all MPI states (this is done when
  calling \code{mpi.exit} or \code{mpi.quit}.
\item[\code{mpi.exit()}] terminates the mpi communication
  environment and detaches the \pkg{Rmpi} package which makes reloading of
  the package \pkg{Rmpi} in the same session impossible.  
\item[\code{mpi.quit()}] terminates the mpi communication
  environment and quits R.  
\end{description}

Example~\ref{ex:Rmpi-init} shows how the configuration of the
parallel environment can be obtained. First it returns the hosts
connected to the parallel environment and then prints the number of
CPUs available in it. After a query if this process is the master
process, the hostname the current process runs on is
returned. 

\begin{Example} Queries to the MPI communication environment
\label{ex:Rmpi-init}
\begin{Schunk}
\begin{Sinput}
> library("Rmpi")
> lamhosts()
\end{Sinput}
\begin{Soutput}
node065 node065 node045 node045 node025 node025 node047 node047 
      0       1       2       3       4       5       6       7 
\end{Soutput}
\begin{Sinput}
> mpi.universe.size()
\end{Sinput}
\begin{Soutput}
[1] 8
\end{Soutput}
\begin{Sinput}
> mpi.is.master()
\end{Sinput}
\begin{Soutput}
[1] TRUE
\end{Soutput}
\begin{Sinput}
> mpi.get.processor.name()
\end{Sinput}
\begin{Soutput}
[1] "node065"
\end{Soutput}
\end{Schunk}
\end{Example}

\subsection{Process Spawning and Communication}

In \pkg{Rmpi} it is easy to spawn R slaves and use them as
workhorses. The 
communication between all the involved processes is carried out in a
so called communicator (comm). All processes within the same
communicator are able to send or receive messages from other
processes. The processes are identified through their commrank (see
also the fundamentals of message passing in
Section~\ref{sec:messagepassing}). The big advantage of \pkg{Rmpi}
slaves is, that they can be used interactively when using the default
R slave script.

\subsubsection{Process Management  Functions}
\begin{description}
\item[\code{mpi.spawn.Rslaves(Rscript =
    system.file(nslaves =
    mpi.universe.size(), ...)}] spawns \code{nslaves} number of R
  workhorses to those hosts automatically chosen by MPI. For other
  arguments represented by \ldots to this function we refer to
  \cite{yu06Rmpi}.
\item[\code{mpi.close.Rslaves(dellog = TRUE, comm = 1)}] closes
  previously spawned R slaves and returns 1 if succesful.
\item[\code{mpi.comm.size()}] returns the total number of members in
  a communicator.
\item[\code{mpi.comm.rank()}] returns the rank (identifier) of the
  process in a communicator.
\item[\code{mpi.remote.exec(cmd, ..., comm = 1, ret = TRUE)}]
  executes a command \code{cmd} on R slaves with \ldots arguments to
  \code{cmd} and returns executed results if \code{ret} is
  \code{TRUE}.
\end{description}

In Example~\ref{ex:Rmpi2} as many slaves are spawned as are available
in the parallel environment. The size of the communicator is returned
(1 master plus the spawned slaves) and a remote query of the commrank
is carried out. Before the slaves are closed the commrank of the
master is printed.

\begin{Example} Process management and communication 

\begin{Schunk}
\begin{Sinput}
> mpi.spawn.Rslaves(nslaves = mpi.universe.size())
\end{Sinput}
\begin{Soutput}
	8 slaves are spawned successfully. 0 failed.
master (rank 0, comm 1) of size 9 is running on: node065 
slave1 (rank 1, comm 1) of size 9 is running on: node065 
slave2 (rank 2, comm 1) of size 9 is running on: node065 
slave3 (rank 3, comm 1) of size 9 is running on: node045 
slave4 (rank 4, comm 1) of size 9 is running on: node045 
slave5 (rank 5, comm 1) of size 9 is running on: node025 
slave6 (rank 6, comm 1) of size 9 is running on: node025 
slave7 (rank 7, comm 1) of size 9 is running on: node047 
slave8 (rank 8, comm 1) of size 9 is running on: node047 
\end{Soutput}
\begin{Sinput}
> mpi.comm.size()
\end{Sinput}
\begin{Soutput}
[1] 9
\end{Soutput}
\begin{Sinput}
> mpi.remote.exec(mpi.comm.rank())
\end{Sinput}
\begin{Soutput}
  X1 X2 X3 X4 X5 X6 X7 X8
1  1  2  3  4  5  6  7  8
\end{Soutput}
\begin{Sinput}
> mpi.comm.rank()
\end{Sinput}
\begin{Soutput}
[1] 0
\end{Soutput}
\begin{Sinput}
> mpi.close.Rslaves()
\end{Sinput}
\begin{Soutput}
[1] 1
\end{Soutput}
\end{Schunk}
\label{ex:Rmpi2}
\end{Example}
 
\subsection{Built-in High Level Functions}

\pkg{Rmpi} provides many high level functions. We selected a few
of them which we think are commonly used. Most of them have been
utilized to build parallel programs presented in the subsequent
chapters.

\subsubsection{High Level Functions}
\begin{description}
\item[\code{mpi.apply(x, fun, ..., comm = 1)}] applies a function
  \code{fun} with additional arguments \ldots to a specific part of
  a vector \code{x}. The return value is of type list with the same
  length as of \code{x}. The length of
  \code{x} must not exceed the 
  number of R slaves spawned as each element of the vector is used
  exactly by one slave. To achieve some sort of load balancing please
  use the corresponding apply functions below.
\item[\code{mpi.applyLB(x, fun, ..., comm = 1)}] applies a function
  \code{fun} with additional arguments \ldots to a specific part of
  a vector \code{x}. There are a few more variants explained in
  \cite{yu06Rmpi}.
\item[\code{mpi.bcast.cmd(cmd = NULL, rank = 0, comm = 1)}]
  broadcasts a command \code{cmd} from the sender \code{rank} to
  all R slaves and evaluates it.
\item[\code{mpi.bcast.Robj(obj, rank = 0,comm = 1)}]
  broadcasts an R object \code{obj} from process rank \code{rank}
  to all other processes (master and slaves).
\item[\code{mpi.bcast.Robj2slave(obj, comm = 1)}] broadcasts an R
  object \code{obj} to all R slaves from the master process. 
\item[\code{mpi.parSim( ... )}] carries out a Monte Carlo simulation
  in parallel. For details on this function see the package manual
  (\cite{yu06Rmpi}) and on Monte Carlo simulation
  the applications in chapter \ref{chap:options}.
\end{description}

How to use the high level function \code{mpi.apply()} is shown in
Example~\ref{ex:Rmpi3}. A vector of $n$ random numbers is generated on
each of the $n$ slaves and are returned to the master as a
list (each list element representing one row). Finally a $n  \times n$
matrix is formed and printed. The output of the matrix shows for each
row the same random numbers. This is because of the fact, that each
slave has the same seed. This problem is more specific treated in
Chapter~\ref{chap:options}. For more information about parallel random
number generators see the descriptions of the packages \pkg{rsprng}
and \pkg{rlecuyer} in Section~\ref{sec:otherpackages}.

\begin{Example} Using mpi.apply
\label{ex:Rmpi3}
\begin{Schunk}
\begin{Sinput}
> n <- 8
> mpi.spawn.Rslaves(nslaves = n)
\end{Sinput}
\begin{Soutput}
	8 slaves are spawned successfully. 0 failed.
master (rank 0, comm 1) of size 9 is running on: node065 
slave1 (rank 1, comm 1) of size 9 is running on: node065 
slave2 (rank 2, comm 1) of size 9 is running on: node065 
slave3 (rank 3, comm 1) of size 9 is running on: node045 
slave4 (rank 4, comm 1) of size 9 is running on: node045 
slave5 (rank 5, comm 1) of size 9 is running on: node025 
slave6 (rank 6, comm 1) of size 9 is running on: node025 
slave7 (rank 7, comm 1) of size 9 is running on: node047 
slave8 (rank 8, comm 1) of size 9 is running on: node047 
\end{Soutput}
\begin{Sinput}
> x <- rep(n, n)
> rows <- mpi.apply(x, runif)
> X <- matrix(unlist(rows), ncol = n, byrow = TRUE)
> round(X, 3)
\end{Sinput}
\begin{Soutput}
     [,1]  [,2]  [,3]  [,4]  [,5]  [,6]  [,7]  [,8]
[1,]  0.5 0.976 0.347 0.266 0.471 0.444 0.069 0.955
[2,]  0.5 0.976 0.347 0.266 0.471 0.444 0.069 0.955
[3,]  0.5 0.976 0.347 0.266 0.471 0.444 0.069 0.955
[4,]  0.5 0.976 0.347 0.266 0.471 0.444 0.069 0.955
[5,]  0.5 0.976 0.347 0.266 0.471 0.444 0.069 0.955
[6,]  0.5 0.976 0.347 0.266 0.471 0.444 0.069 0.955
[7,]  0.5 0.976 0.347 0.266 0.471 0.444 0.069 0.955
[8,]  0.5 0.976 0.347 0.266 0.471 0.444 0.069 0.955
\end{Soutput}
\begin{Sinput}
> mpi.close.Rslaves()
\end{Sinput}
\begin{Soutput}
[1] 1
\end{Soutput}
\end{Schunk}
\end{Example}

\subsection{Other Important Functions}

To complete the set of important functions supplied by the \pkg{Rmpi}
package the following functions have to be explained.

\subsubsection{Collective Communication Routines}
\begin{description}
\item[\code{.mpi.gather(x, type, rdata, root = 0, comm = 1)}] gathers
  data distributed on the nodes (\code{x}) to a 
  specific process (mostly the master) into a single array or list
  (depending on the length of the data) of type \code{type} which can
  be integer, double or character. It
  performs  
  a send of messages from each member in a comm. A
  specific process (\code{root}) accumulates this messages into a 
  single array or list prepared with the \code{rdata} command.
\item[\code{.mpi.scatter(x, type, rdata, root = 0, comm = 1)}]
  sends to each member of a comm a partition of  a vector \code{x},
  type \code{type} which can be integer, double or character,
  from a specified member of the group (mostly the master). Each
  member of the comm receive its part of \code{x} after preparing the
  receive buffer with the argument \code{rdata}.
\end{description}

\subsection{Conclusion}

The package \pkg{Rmpi} implements many of the routines available in
MPI-2. But there are some that have to be ommitted or are not included
because they are not needed for use in R (e.g., data management
routines are not necessary as R has its own tools for data handling).
A really interesting aspect of the \pkg{Rmpi} package is the
possibilty to spawn interactive R slaves. That enables the user to
interactively 
define functions which can be executed remotely on the slaves in
parallel. An example how one can do this is shown in
Chapter~\ref{chap:matrix}, where the implementation of matrix
multiplication is shown. A major disadvantage is that MPI in its
current implementations lack in fault tolerance. This results in a
rather instable execution of MPI applications. Moreover, debugging
is really difficult as there is no support for it in \pkg{Rmpi}.

All in all this package is a good start in creating parallel programs
as this can easy be achieved entirely in R.

For further interface functions supplied by the \pkg{Rmpi} package, a
more detailed description and further examples please consult the
package description \cite{yu06Rmpi}.


%% subsection: The rpvm package 
\section{The rpvm Package}
\label{sec:rpvm}
Parallel virtual machine uses the message passing model and makes a
collection of computers appear as a single virtual machine (see Section
\ref{sec:PVM} for details).
The package \pkg{rpvm} (\cite{nali07rpvm}) provides an interface to
low level PVM functions and
a few high level parallel functions to R. It uses most of the
facilities provided by the PVM system which makes \pkg{rpvm} ideal for
prototyping parallel statistical applications in R.

Generally, parallel applications can be either written in compiled
languages like C or 
FORTRAN or can be called as R processes. The latter method is used in
this thesis and therefore a good selection of \pkg{rpvm}
functions are explained in this section to show how PVM and R can be
used together. 
%Provided functions are categorized as follows:

%\begin{itemize}
%\item Initialization and Status Queries
%\item Process Spawning and Communication
%\item Built-in High Level Functions
%\item Other Important Functions
%\end{itemize}  

\subsection{Initialization and Status Queries}

At first for using the PVM system \textit{pvmd3} has to be
booted. This can be done via the command line using the \code{pvm}
command (see pages 22 and 23 in \cite{geist94pvm}) or directly within
R after loading the \pkg{rpvm} package using \code{.PVM.start.pvmd()}
explained in this section.

\subsubsection{Functions for Managing the Virtual Machine}
\begin{description}
\item[\code{.PVM.start.pvmd()}] boots the \textit{pvmd3} daemon. The
  currently running R session becomes the master process. 
\item[\code{.PVM.add.hosts(hosts)}] takes a vector of hostnames to
  be added to the current virtual machine. The syntax of the
  hostnames is similar to the lines of a pvmd hostfile (for details
  see the man page of \textit{pvmd3}). 
\item[\code{.PVM.del.hosts()}] simply deletes the given hosts from
  the virtual machine configuration.
\item[\code{.PVM.config()}] returns information about the present
  virtual machine.
\item[\code{.PVM.exit()}] tells the PVM daemon that this process
  leaves the parallel environment.
\item[\code{.PVM.halt()}] shuts down the entire PVM system and exits
  the current R session.
\end{description}
  
When using a job queueing system like the Sun Grid Engine (SGE) to boot
the PVM parallel environment the developer is not engaged with
setting up and booting the environment anymore (see
appendix \ref{app:gridengine} on how to do this).

Example~\ref{ex:rpvm-init} shows how the configuration of the parallel
environment can be obtained. \code{.PVM.config()} returns the hosts
connected to 
the parallel virtual machine. After that the
parallel environment is stopped. 
\begin{Example} Query status of PVM 
\label{ex:rpvm-init}
\begin{Schunk}
\begin{Sinput}
> library("rpvm")
> set.seed(1782)
> .PVM.config()
\end{Sinput}
\begin{Soutput}
  host.id    name    arch speed
1  262144 node066 LINUX64  1000
2  524288 node020 LINUX64  1000
3  786432 node036 LINUX64  1000
4 1048576 node016 LINUX64  1000
\end{Soutput}
\begin{Sinput}
> .PVM.exit()
\end{Sinput}
\end{Schunk}
\end{Example}

\subsection{Process Spawning and Communication}

The package \pkg{rpvm} uses the master-slave paradigm meaning that one
process is the master task and the others are slave tasks. \pkg{rpvm}
provides a routine to spawn R slaves but these slaves cannot be used
interactively like the slaves in \pkg{Rmpi}. The spawned R slaves
source an R script which contains all the necessary function calls to
set up communication and carry out the computation and after
processing terminate.
PVM uses task IDs (\code{tid}---a positive integer for identifying a task)
and tags for communication (see
also the fundamentals of message passing in
Section~\ref{sec:messagepassing}).

\subsubsection{Process Management  Functions}
\begin{description}
\item[\code{.PVM.spawnR(slave, ntask = 1, ...)}] spawns \code{ntask}
  copies of an \code{slave} R process. \code{slave} is a character
  specifying the source file for the R slaves located in the package's
  demo directory (the default). There are more
  parameters indicated by the \ldots (we refer to
  \cite{nali07rpvm}). The \code{tids} of the successfully spawned R
  slaves are returned.
\item[\code{.PVM.mytid()}] returns the \code{tid} of the calling
  process.
\item[\code{.PVM.parent()}] returns the \code{tid} of the parent
  process that spawned the calling process.
\item[\code{.PVM.siblings()}] returns the \code{tid} of the processes
  that were spawned in a single spawn call.
\item[\code{.PVM.pstats(tids)}] returns the status of the PVM
  process(es) with task ID(s) \code{tids}.
\end{description}

\subsection{Built-in High Level Functions}

\pkg{rpvm} provides only two high level functions. One of them is a
function to get or set values in the virtual machine settings. This is
certainly because each new high level functions needs separate source
files for the slaves and this is not what developers do intuitively. 

\subsubsection{High Level Functions}
\begin{description}
\item[\code{PVM.rapply(X, FUN = mean, NTASK = 1))}] Apply a function
  \code{FUN} to the rows of a matrix \code{X} in
  parallel using \code{NTASK} tasks.
\item[\code{PVM.options(option, value)}] Get or set values of libpvm
  options (for details see \cite{nali07rpvm} and \cite{geist94pvm}).
\end{description}

Example~\ref{ex:rpvm-rapply} shows how the rows of a matrix \code{X}
can be summed up in parallel via \code{PVM.rapply()}.

\begin{Example} Using PVM.rapply
\begin{Schunk}
\begin{Sinput}
> n <- 8
> X <- matrix(rnorm(n * n), nrow = n)
> round(X, 3)
\end{Sinput}
\begin{Soutput}
       [,1]   [,2]   [,3]   [,4]   [,5]   [,6]   [,7]   [,8]
[1,] -0.200 -0.183  0.560  1.286  0.468  0.502  0.874 -0.778
[2,] -1.371  0.484 -0.498  1.788  0.534 -0.566  0.152 -1.307
[3,]  1.041  0.484  0.399  0.580  0.586 -0.660  1.833 -1.405
[4,] -1.117 -0.893  0.408 -1.612  0.486  0.644  0.422 -1.639
[5,]  1.397 -0.237 -1.287 -0.122 -1.076  0.225 -0.047  0.020
[6,] -0.046  0.537 -1.287 -0.089  0.564  2.671 -0.715 -0.901
[7,]  1.085  0.706 -0.034  0.929  0.057 -2.402 -1.233  1.135
[8,]  0.605 -0.076 -0.554  1.385 -0.436  0.249  0.338  1.369
\end{Soutput}
\begin{Sinput}
> PVM.rapply(X, sum, 3)
\end{Sinput}
\begin{Soutput}
Try to spawn tasks...
Work sent to  524289 
Work sent to  786433 
Work sent to  1048577 
[1]  2.5291297 -0.7840539  2.8572559 -3.3002539 -1.1275728
[6]  0.7329710  0.2431164  2.8806456
\end{Soutput}
\end{Schunk}
\label{ex:rpvm-rapply}
\end{Example}

Before we explain how \code{PVM.rapply} works the following \pkg{rpvm}
functions have to be explained:

\subsubsection{Other Important functions}
\begin{description}
\item[\code{.PVM.initsend()}] clears the default send buffer and
  prepares it for packing a new message.
\item[\code{.PVM.pkstr(data = "")}] and \code{.PVM.pkint(data =
    0, stride = 1)} are low level correspondents of the PVM packing
    routines (see \cite{geist94pvm} for more information on packing
    data). 
\item[\code{.PVM.pkdblmat(data)}] packs a double matrix including
  the dimension information. There are more packing routines
  available. They are explained in \cite{nali07rpvm}.
\item[\code{.PVM.send(tid, msgtag)}] sends the message stored in the
  active buffer to the PVM process identified by \texttt{tid}. The
  content is labeled by the identifier \texttt{msgtag}.
\item[\code{.PVM.recv(tid = -1, msgtag = -1)}] blocks the process
  until a message with label \texttt{msgtag} has arrived from
  \texttt{tid}. $-1$ means any. The receive buffer is cleared
  and the received message is placed there instead.
\item[\code{.PVM.upkstr(), .PVM.upkint(), .PVM.upkdblvec()}] and
  others are the corresponding unpack functions to the
  pack functions explained before.
\item[\texttt{.PVM.gather(x, count = length(x), msgtag, group,
    rootginst = 0)}] gathers data distributed on the nodes (\code{x})
  to a 
  specific process (mostly the root) into a single array. It performs
  a send of messages from each member of a group of processes. A
  specific process (the root) accumulates this messages into a single
  vector. 
\item[\texttt{.PVM.scatter(x, count, msgtag, group, rootqinst = 0)}]
  sends to each member of a group a partition of  a vector \code{x}
  from a 
  specified member of the group (mostly the root) where \texttt{count}
  is an integer specifying the number of elements to be sent to each
  member. 
\end{description}

Let us now examine the function to see how parallel programs using
\pkg{rpvm} can be written.

\begin{Example} PVM.rapply master routine
\begin{Scode}

PVM.rapply <- function (X, FUN = mean, NTASK = 1)
{
    WORKTAG <- 22
    RESULTAG <- 33
    if (!is.matrix(X)) {
        stop("X must be a matrix!")
    }
    if (NTASK == 0) {
        return(apply(X, 1, FUN))
    }
    end <- nrow(X)
    chunk <- end%/%NTASK + 1
    start <- 1
    mytid <- .PVM.mytid()
    children <- .PVM.spawnR(ntask = NTASK, slave = "slapply")
    if (all(children < 0)) {
        cat("Failed to spawn any task: ", children, "\n")
        .PVM.exit()
    }
    else if (any(children < 0)) {
        cat("Failed to spawn some tasks.  Successfully spawned ",
            sum(children > 0), "tasks\n")
        children <- children[children > 0]
    }
    for (id in 1:length(children)) {
        .PVM.initsend()
        range <- c(start, ifelse((start + chunk - 1) > end, end,
            start + chunk - 1))
        work <- X[(range[1]):(range[2]), , drop = FALSE]
        start <- start + chunk
        .PVM.pkstr(deparse(substitute(FUN)))
        .PVM.pkint(id)
        .PVM.pkdblmat(work)
        .PVM.send(children[id], WORKTAG)
        cat("Work sent to ", children[id], "\n")
    }
    partial.results <- list()
    for (child in children) {
        .PVM.recv(-1, RESULTAG)
        order <- .PVM.upkint()
        partial.results[[order]] <- .PVM.upkdblvec()
    }
    .PVM.exit()
    return(unlist(partial.results))
}
\end{Scode} 
\label{ex:rpvm-rapplymaster}
\end{Example}
The corresponding slave R script (slapply.R) looks as follows:

\begin{Example} PVM.rapply slave routine
\begin{Scode}

library (rpvm)
WORKTAG <- 22
RESULTAG <- 33
mytid  <- .PVM.mytid ()
myparent  <- .PVM.parent ()
## Receive work from parent (a matrix)
buf <- .PVM.recv (myparent, WORKTAG)
## Function to apply
func  <- .PVM.upkstr ()
cat ("Function to apply: ", func, "\n")
## Order
order <- .PVM.upkint ()
partial.work <- .PVM.upkdblmat ()
print (partial.work)
## actually work, take the mean of the rows
partial.result <- apply (partial.work, 1, func)
print (partial.result)
## Send result back
.PVM.initsend ()
.PVM.pkint (order)
.PVM.pkdblvec (partial.result)
.PVM.send (myparent, RESULTAG)
## Exit PVM
.PVM.exit ()
## Exit R
q (save="no")
\end{Scode}
\label{ex:rpvm-rapplyslave}
\end{Example}


Example~\ref{ex:rpvm-rapplymaster} shows the
master routine of \code{PVM.rapply()}. This function takes a matrix,
the function which is going to be applied and the number of processors
to use as arguments. At first 
the message tags are specified. These tags are necessary to uniquely
identify messages sent in a message passing environment. After
input validation \code{NTASK} child processes are spawned using the
\code{.PVM.spawnR()} command. After initializing the send buffer the
partitioned data (packed in the buffer using the \code{.PVM.pk*}
commands) is send to the corresponding child processes represented by
their task IDs using
\texttt{.PVM.send()}. PVM uses these task identifiers (tid) to
address pvmds, tasks, and groups of tasks within a virtual
machine.

Meanwhile the spawned slave processes (see
Example~\ref{ex:rpvm-rapplyslave}) have been idle because they wait
for input (\code{.PVM.receive()} is a blocking command). After
receiving data from the parent the data gets unpacked. Now the slaves
apply the given function to their part of the matrix. Finally another
send is initialized to provide the results to the parent process and
the slaves are detached from the virtual machine by calling a
\texttt{.PVM.exit()}. 

\subsection{Conclusion}

The \code{PVM.rapply()} example shown in this section followed the
Single Program Multiple Data (SPMD) paradigm. Data is split into
different parts which are sent to different processes. I/O is handled
solely by a master
process. When loading \pkg{rpvm} in an R session this session becomes the
master process. Slaves can easily be spawned provided that there are
working slave scripts available.

We encountered no problems when using the routines in \pkg{rpvm}. This
package seem to be rather stable in contrast to \pkg{Rmpi}, where for
unknown reasons the MPI environment sometimes crashed.

A major disadvantage is that the \pkg{rpvm}
package only has two higher level function. One of them can be used
for calculations. That means when using this package for parallel
computing one has 
to deal with low level message passing but which in turn may provide
higher flexibility. New parallel functions can be constructed on the
basis of the provided interface. The \code{PVM.rapply} code
can be taken as a template for further routines. 

Another disadvantage is the missing support for interactive R
slaves. Parallel tasks have to be created on the basis of separate
slave source files which are sourced on the creation of the slaves.

For further interface functions supplied by the \pkg{rpvm} package, a more
detailed 
description and further examples please consult the package description
\cite{nali07rpvm}.


%% subsection: The snow package 
\section{The snow Package}
\label{sec:snow}
The aim of simple network of workstations
(\pkg{snow}---\cite{rossini03snow}, \cite{tierney07snow}) is to
provide a simple parallel computing environment in R. To make a
collection of computers to appear as a virtual cluster in R \pkg{snow}
three different message passing environments can be used:

\begin{itemize}
\item PVM via R package rpvm (see section \ref{sec:rpvm})
\item MPI via R package Rmpi (see section \ref{sec:Rmpi})
\item SOCK via TCP sockets
\end{itemize}

The details of which mechanism is used and how it is used are hidden
from the high level user.
After setting up this virtual cluster developing parallel R functions
can be achieved via an standardized interface to the computation
nodes.
Moreover, when using \pkg{snow} one can rely on a good handful of
high level functions. This makes it rather easy to use the underlying
parallel computational engine.
Indeed \pkg{snow} uses existing interfaces to R namely \pkg{Rmpi} when
using MPI (see Section~\ref{sec:Rmpi}), \pkg{rpvm} when using PVM (see
Section~\ref{sec:rpvm}) and a new possibilty of message passing namely
TCP sockets, which is a rather simple way of achieving communication
between nodes (in most application this is not the optimal way).
What follows is a description of high level functions supplied by the
\pkg{snow} package. They are assigned to one of the topics:

\begin{itemize}
\item Initialization
\item Built-in High Level Functions
\item Fault Tolerance
\end{itemize}

\subsection{Initialization}

Initializing a \pkg{snow} cluster is rather easy if the system is
prepared accordingly. When using MPI (achieved through \pkg{Rmpi}) a
LAM/MPI environment has to be booted prior starting the virtual
cluster (see section \ref{sec:Rmpi}). Is PVM the method of choice the
\pkg{rpvm} package must be available and an appropriate PVM has to be
started (see section \ref{sec:rpvm}). For both MPI and PVM the
parallel environment 
can be configured through a grid engine (see appendix
\ref{app:gridengine}). TCP sockets can be set up directly using the
package. MPI or PVM have the possibility to query the status of the
parallel environment. This can be done using the functions supplied
from the corresponding package.


\pkg{snow} management functions:

\begin{description}
\item[\code{makecluster(spec, type = getClusterOption(``type''))}]
  starts a cluster of type \code{type} with \code{spec} numbers of
  slaves. If the cluster is of connection type SOCK then \code{spec}
  must be a charactor vector containing the hostnames of the
  slavenodes to join the cluster. The return value is a list
  containing the cluster specifications. This object is necessary in
  further function calls.
\item[\code{stopCluster(cl)}] stops a cluster specified in \code{cl}.
\end{description}


\begin{Example} Start/stop cluster in \pkg{snow} \newline
running on cluster@WU using the node.q -- the parallel environment was
started with the SGE using 8 nodes.

\begin{Schunk}
\begin{Sinput}
> library("snow")
> set.seed(1782)
> n <- 8
> cl <- makeCluster(n, type = "MPI")
\end{Sinput}
\begin{Soutput}
	8 slaves are spawned successfully. 0 failed.
\end{Soutput}
\begin{Sinput}
> stopCluster(cl)
\end{Sinput}
\begin{Soutput}
[1] 1
\end{Soutput}
\end{Schunk}
\label{ex:snowstartstop}
\end{Example}

\subsection{Built-in High Level Functions}

\pkg{snow} provides a good handful of high-level functions: 


%% TODO
\begin{description}
\item[\code{clusterEvalQ(cl, expr)}] evaluates an R expression
  \code{expr} on
  each cluster node provided by \code{cl}. 
\item[\code{clusterCall(cl, fun, ...)}] calls a function
  \code{fun} with arguments \ldots on each node found in \code{cl}
  and returns a list of the results.
\item[\code{clusterApply(cl, x, fun, ...)}] applies a function
  \code{fun} with additional arguments \ldots to a specific part of
  a vector \code{x}. The return value is of type list with the same
  length as of \code{x}. The length of
  \code{x} must not exceed the 
  number of R slaves spawned as each element of the vector is used
  exactly by one slave. To achieve some sort of load balancing please
  use the corresponding apply functions below.
\item[\code{clusterApplyLB(cl, x, fun, ...)}] is a load balancing
  version of \code{clusterApply()} which applies a function
  \code{fun} with additional arguments \ldots to a specific part of
  a vector \code{x} with the difference that the length of
  \code{x} can exceed the number of cluster nodes. If a node
  finished with the computation the next job is placed on the
  available node. This is repeated until all jobs have completed.
\item[\code{clusterExport(cl, list)}] broadcasts a list of global
  variables on the master (\code{list}) to all slaves.
\item[\code{parApply(cl, x, fun, ...)}] is one of the parallel
  versions of the \code{apply} functions available in R. We refer to
  the package documentation (\cite{tierney07snow}) for further details.
\item[\code{parMM(cl, A,B)}] is a simple parallel implementation of
  matrix multiplication. 
\end{description}


\begin{Example} Using high-level functions of snow\newline
running on cluster@WU using the node.q -- the parallel environment was
started with SGE using 8 nodes

\begin{Schunk}
\begin{Sinput}
> n <- 8
> cl <- makeCluster(n, type = "MPI")
\end{Sinput}
\begin{Soutput}
	8 slaves are spawned successfully. 0 failed.
\end{Soutput}
\begin{Sinput}
> x <- rep(n, n)
> rows <- clusterApply(cl, x, runif)
> X <- matrix(unlist(rows), ncol = n, byrow = TRUE)
> X
\end{Sinput}
\begin{Soutput}
           [,1]      [,2]     [,3]      [,4]      [,5]      [,6]       [,7]
[1,] 0.03582783 0.3065854 0.228593 0.2512860 0.1055006 0.1938628 0.05882335
[2,] 0.03582783 0.3065854 0.228593 0.2512860 0.1055006 0.1938628 0.05882335
[3,] 0.03582783 0.3065854 0.228593 0.2512860 0.1055006 0.1938628 0.05882335
[4,] 0.03582783 0.3065854 0.228593 0.2512860 0.1055006 0.1938628 0.05882335
[5,] 0.03582783 0.3065854 0.228593 0.2512860 0.1055006 0.1938628 0.05882335
[6,] 0.03582783 0.3065854 0.228593 0.2512860 0.1055006 0.1938628 0.05882335
[7,] 0.03582783 0.3065854 0.228593 0.2512860 0.1055006 0.1938628 0.05882335
[8,] 0.03582783 0.3065854 0.228593 0.2512860 0.1055006 0.1938628 0.05882335
          [,8]
[1,] 0.3884308
[2,] 0.3884308
[3,] 0.3884308
[4,] 0.3884308
[5,] 0.3884308
[6,] 0.3884308
[7,] 0.3884308
[8,] 0.3884308
\end{Soutput}
\end{Schunk}
\label{ex:snowapply}
\end{Example}

\subsection{Fault Tolerance}



\subsection{Conclusion}

The routines available in package \pkg{snow} are easy to understand
and use, provided that there is a corresponding communication
environment set up. Generally, the user need not know the underlying
parallel infrastructure, she just ports her sequential code so that it
uses the functions supplied by \pkg{snow}. All in all as the title
suggests simple network of workstations is simple to get started with
and is simple with respect to the possibilities of parallel
computations. 

For further interface functions supplied by the \pkg{snow} package, a
more detailed 
description and further examples please consult the package description
\cite{tierney07snow}.


%% subsection: The paRc package 
\section{paRc---PARallel Computations in R}
\label{sec:paRc}

In the course of this thesis a package called \pkg{paRc}
(\cite{theussl07paRc}) has been developed with the aim
to evaluate performance of parallel applications and to show how
interfacing high performance applications written in C can be done
using OpenMP~(see Section~\ref{sec:OpenMP}).

The package \pkg{paRc} contains interface functions to the OpenMP library
and provides high level function to a few C implementations of
parallel applications using OpenMP (e.g., matrix multiplication---see
Chapter~\ref{chap:matrix}).
Furthermore, it supplies a benchmark environment for performance
evaluation of parallel programs and a framework for pricing options
with parallel Monte Carlo simulation (see
Chapter~\ref{chap:options}).

\pkg{paRc} can be obtained from \url{R-Forge.R-project.org}---the
R-project community service. To install this package directly within R
call \code{install.packages("paRc", repos = "R-Forge.r-project.org")}.
To properly install this package you need either the Intel compiler
with version 9.1 or newer (the Linux compiler is free for
non-commercial use) or the GNU~C compiler with version 4.2 or
newer. They are known to support OpenMP.

Examples in this section are produced on a bignode of
cluster@WU. Bignodes provide a shared memory platform with up to 4
CPUs. Shared memory platforms are necessary for running parallel
OpenMP applications.

\subsection{OpenMP Interface Functions}

The user is provided with a few interface functions to the OpenMP
library. They are used to query the internal variables of the compiled
parallel environment or to change them.

\subsubsection{OpenMP Routines}
\begin{description}
\item[\texttt{omp.get.num.procs()}] returns the number of threads
  available to the program.
\item[\texttt{omp.set.num.threads()}] sets the number of threads to be
  used in subsequent parallel executions.
\item[\texttt{omp.get.max.threads()}] gets the number of threads to be
  used in subsequent parallel executions.
\end{description}

\subsubsection{OpenMP Specific Environment Variables}

Moreover, environment variables can affect the runtime behaviour of
OpenMP programs. These environment variables are~(\cite{openMP05}):

\begin{description}
\item[\texttt{OMP\_NUM\_THREADS}] sets the number of threads to use in
  parallel regions of OpenMP programs. 
\item[\texttt{OMP\_SCHEDULE}] sets the runtime schedule type and
  chunk size.
\item[\texttt{OMP\_DYNAMIC}] defines wether dynamic adjustments of threads
  should be used in parallel regions.
\item[\texttt{OMP\_NESTED}] enables or disables nested parallelism.
\end{description}

Example~\ref{ex:paRcOMP} shows the use of the OpenMP library calls in
R. First the number of available processors is queried. Then the
number of threads a parallel application may use is set to 2. With the
last call the current available CPUs to a parallel program is queried.

\begin{Example} OpenMP function calls using \pkg{paRc}
\label{ex:paRcOMP}
\begin{Schunk}
\begin{Sinput}
> library("paRc")
> omp.get.num.procs()
\end{Sinput}
\begin{Soutput}
[1] 4
\end{Soutput}
\begin{Sinput}
> omp.set.num.threads(2)
> omp.get.max.threads()
\end{Sinput}
\begin{Soutput}
[1] 2
\end{Soutput}
\end{Schunk}
\end{Example}

\subsection{High Level OpenMP Functions}

\pkg{paRc} provides the following high level OpenMP function: 

\begin{description}
\item[\texttt{omp.matrix.mult(X, Y, n\_cpu = 1)}] multiplies the matrix
  \code{X} with matrix \code{Y} using \code{n\_cpu} numbers of
  processors.
\end{description}

\subsection{Benchmark Environment}

\pkg{paRc} provides a benchmark environment for measuring the
performance of parallel programs. Two main functions exist in this
context---one for creating a benchmark object and one for running the
benchmark described by the object.

\subsubsection{Class \class{benchmark}}

An S3 object (\cite{chambers91sms}) named \class{benchmark} contains all
the necessary information to run a benchmark. The list elements are

\begin{description}
\item[task] is a character string defining the task of the
  benchmark. Currently, the 
  following tasks are implemented:
  \begin{itemize}
  \item matrix multiplication
  \item Monte Carlo simulation
  \end{itemize}
\item[data] is a list containing the parameters and data to properly
  run the task
\item[type]defines the parallel programming model used to run the
  benchmark. Currently, the following types are implemented:
  \begin{itemize}
  \item OpenMP---C interface calls provided by \pkg{paRc}
  \item MPI---implementation in \pkg{paRc} using \pkg{Rmpi} for
    communication
  \item PVM---implementation in \pkg{paRc} using \pkg{rpvm} for
    communication 
  \item snow-MPI---implementation in \pkg{snow} using \pkg{Rmpi} for
    communication
  \item snow-PVM---implementation in \pkg{snow} using \pkg{rpvm} for
    communication
  \end{itemize}
\item[cpu\_range] contains a vector of integers representing the number
  of CPUs for th corresponding benchmark run 
\item[is\_parallel] is a logical \code{TRUE} or \code{FALSE} wether the
  contains parallel tasks or not
\end{description}

\subsubsection{Main Routines}

These are the main routines for benchmarking parallel applications:

\begin{description}
\item[\code{create.benchmark(task, data, type, cpu\_range, ...)}]
  defines a benchmark object using a specific \code{task} and the
  corresponding \code{data}. The \code{type} refers to the serial or
  parallel paradigm to use. The \code{cpu\_range} specifies the range
  of CPUs to use for this benchmark. 
\item[\code{run.benchmark(x)}] takes a benchmark object as argument
  and carries out the defined benchmark. It returns an object of type
  bm\_results (which inherits from a dataframe) containing the
  results of the benchmark.
\end{description}


\subsubsection{Extractor and Replacement Functions}

The following routines are for handling a benchmark object. They
extract or replace the values in the benchmark \code{x}.

\begin{itemize}
\item \code{bm.task(x)}
\item \code{bm.data(x)}
\item \code{bm.type(x)}
\item \code{bm.cpu.range(x)}
\end{itemize}

The following routines supply extra information about the benchmark object
or the benchmark environment.

\begin{description}
\item[\code{bm.is.parallel}] returns \code{TRUE} if the benchmark
  contains a parallel function. 
\item[\code{bm.tasks}] returns the tasks which are possible to run
  with the benchmark environment.
\item[\code{bm.types}] returns the types of available serial or
  parallel paradigms to run with the benchmark environment.
\end{description}

\subsubsection{Generic Functions}

Generic functions provide methods for different objects. In \pkg{paRc}
a generic function for calculating the speedup is provided:

\begin{description}
\item[\code{speedup(x)}] is a generic function taking an object as
  arguments. Currently there are two methods implemented namely
  \code{speedup.numeric} and \code{speedup.bm\_results}. The methods
  calculate the speedup as it is presented in
  Equation~\ref{eq:speedup}.
\end{description}

\subsubsection{S3 Methods}

The following S3 methods are provided for the benchmark environment:

\begin{description}
\item[\code{print.benchmark}] prints objects of class \class{benchmark}
\item[\code{plot.bm\_results}] supplies a plot method for comparing
  benchmark results.
\item[\code{speedup.default}] returns an error message that there is
  no default method. 
\item[\code{speedup.numeric}] returns the speedups calculated from a
  vector of type \class{numeric}. The reference execution time is  the
  first element in the vector. The return value is a vector of type
  \class{numeric}. 
\item[\code{speedup.bm\_results}] calculates the speedups from an object
  of class \class{bm\_results} and returns them as a vector of type
  \class{numeric}. 
\end{description}

\subsubsection{Example}

\begin{Example} Running a benchmark using OpenMP
\label{ex:benchrun}


\section{Other Packages Providing HPC Functionality}
\label{sec:otherpackages}
There are a few other packages which supply the user with parallel
computing functionality or extend other high performance computing
packages. In this section a short description of each of these
packages is given.

\begin{description}
\item[rsprng] (\cite{li07rsprng}) is an R interface to SPRNG (Scalable
  Parallel Random 
  Number Generators---\cite{mascagni00ssl}). SPRNG is a package for parallel
  pseudo random number generation with the aim to be easy to use on a
  variety of architectures, especially in large-scale parallel Monte
  Carlo applications.
\item[rlecuyer] (\cite{sevcikova05rlecuyer}) like \pkg{rsprng}
  provides an interface to a parallel pseudo random number
  generator. \pkg{rlecuyer} is the C implementation of the  
  random number generator with multiple independent streams developed
  by \cite{l'ecuyer02RNG}.
\item[RScaLAPACK] (\cite{samatova05RSca} and \cite{yoginath05rhp})
  uses the high performance ScaLAPACK library
  (\cite{dongarra97sus})for linear algebra computations. ScaLAPACK is
  a library of high performance linear algebra routines which makes
  use of message passing to run on distributed memory machines. Among
  other routines it provides functionality to solve systems of linear equations, linear
  least squares problems, eigenvalue problems and singular value
  problems. ScaLAPACK is an acronym for Scalable Linear Algebra
  PACKage or Scalable LAPACK. \cite{samatova06hps} have shown how to
  use \pkg{RScaLAPACK} in biology and climate modelling and analyzed
  its performance.
\item[papply] (\cite{currie05papply}) implements a similar interface
  to lapply and apply but distributes the processing evenly among the
  nodes of a cluster. \pkg{papply} uses the package \pkg{Rmpi} (see
  Section~\ref{sec:Rmpi}) for
  communication but supports in addition error messages for debugging.
\item[biopara] (\cite{lazar06biopara}) allows users to distribute
  execution of large problems over multiple machines. It uses R socket
  connections (TCP) for communication. \pkg{biopara} supplies two high
  level functions, one for distributing work (represented by function
  and its arguments) among ``workers'' and the other for doing
  parallel bootstrapping like the native R function boot(). 
\item[taskPR] (\cite{samatova04taskPR}) provides a parallel execution
  environment on the basis of TCP or MPI-2.
\end{description}

\section{Conclusion}

In this chapter we have shown that R offers a lot of possibilities in
high performance computing. Extensions which interface the main
message passing environments, MPI and PVM, offer the highest
flexibility as parallel applications can be based upon low level
message passing routines. Especially package \pkg{Rmpi} has to be
noted as it offers interactive testing of developed parallel
functions. 

For easy parallelization of existing C code on shared memory machine a
new approach has been shown using package \pkg{paRc}. Implicit
parallel programming can be achieved using the compiler directives
offered by OpenMP. A major disadvantage is that parallel functionality
has to be implemented low level in C or FORTRAN.