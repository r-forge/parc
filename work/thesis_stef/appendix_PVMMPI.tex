
\section{Installation of Message Passing Environments}
\label{app:mpi_imp}

In this appendix it is shown how to set up an appropriate message
passing environment and how this environment can be handled.

\subsection{LAM/MPI}

For installing LAM/MPI on a workstation or a cluster
either the source code or pre-compiled binaries can be downloaded from
the project website (\url{http://www.lam-mpi.org/}).
The source code can be compiled as follows (jobscript for SGE):

\begin{verbatim}
#$ -N compile-lam-gcc

## change to the source directory of downloaded LAM/MPI
cd /home/stheussl/src/lam-7.1.3/
echo "#### clean ####"
make clean
echo "#### configure ####"
## configure - enable shared library support
./configure CC=gcc CXX=g++  FC=gfortran --prefix=/home/stheussl/lib/lam-gcc --enable-shared
echo "#### make ####"
make all
echo "#### install ####"
make install
echo "#### finished ####"
\end{verbatim}

If Debian Etch is used LAM/MPI can be installed using \code{apt-get
  install} followed by the packages

\begin{description}
\item[lam-runtime] LAM runtime environment for executing parallel programs,
\item[lam4-dev] Development of parallel programs using LAM,
\item[lam4c2] Shared libraries used by LAM parallel programs.
\end{description}

lamboot lamwipe etc.




\subsection{PVM}



If Debian Etch is used PVM can be installed using \code{apt-get
  install} followed by the packages

\begin{description}
\item[pvm] Parallel Virtual Machine - binaries 
\item[pvm-dev] Parallel Virtual Machine - development files
\item[libpvm3] Parallel Virtual Machine - shared libraries
\end{description}


\subsection{Intel Compiler}

on AMD64 Debian systems install ia32-libs package to use the intel
compiler.

an R Installation can be build via

./configure --prefix=/home/stefan/lib/R-i --enable-R-shlib CC=icc
CXX=icpc F77=ifort FC=ifort

