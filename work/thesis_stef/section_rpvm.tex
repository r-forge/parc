\section{The rpvm Package}
\label{sec:rpvm}
Parallel virtual machine uses the message passing model and makes a
collection of computers appear as a single virtual machine (see Section
\ref{sec:PVM} for details).
The package \pkg{rpvm} (\cite{nali07rpvm}) provides an interface to
low level PVM functions and
a few high level parallel functions to R. It uses most of the
facilities provided by the PVM system which makes \pkg{rpvm} ideal for
prototyping parallel statistical applications in R.

Generally, parallel applications can be either written in compiled
languages like C or 
FORTRAN or can be called as R processes. The latter method is used in
this thesis and therefore a good selection of \pkg{rpvm}
functions are explained in this section to show how PVM and R can be
used together. 
%Provided functions are categorized as follows:

%\begin{itemize}
%\item Initialization and Status Queries
%\item Process Spawning and Communication
%\item Built-in High Level Functions
%\item Other Important Functions
%\end{itemize}  

\subsection{Initialization and Status Queries}

At first for using the PVM system \textit{pvmd3} has to be
booted. This can be done via the command line using the \code{pvm}
command (see pages 22 and 23 in \cite{geist94pvm}) or directly within
R after loading the \pkg{rpvm} package using \code{.PVM.start.pvmd()}
explained in this section.

\subsubsection{Functions for Managing the Virtual Machine}
\begin{description}
\item[\code{.PVM.start.pvmd()}] boots the \textit{pvmd3} daemon. The
  currently running R session becomes the master process. 
\item[\code{.PVM.add.hosts(hosts)}] takes a vector of hostnames to
  be added to the current virtual machine. The syntax of the
  hostnames is similar to the lines of a pvmd hostfile (for details
  see the man page of \textit{pvmd3}). 
\item[\code{.PVM.del.hosts()}] simply deletes the given hosts from
  the virtual machine configuration.
\item[\code{.PVM.config()}] returns information about the present
  virtual machine.
\item[\code{.PVM.exit()}] tells the PVM daemon that this process
  leaves the parallel environment.
\item[\code{.PVM.halt()}] shuts down the entire PVM system and exits
  the current R session.
\end{description}
  
When using a job queueing system like the Sun Grid Engine (SGE) to boot
the PVM parallel environment the developer is not engaged with
setting up and booting the environment anymore (see
appendix \ref{app:gridengine} on how to do this).

Example~\ref{ex:rpvm-init} shows how the configuration of the parallel
environment can be obtained. \code{.PVM.config()} returns the hosts
connected to 
the parallel virtual machine. After that the
parallel environment is stopped. 
\begin{Example} Query status of PVM 
\label{ex:rpvm-init}
\begin{Schunk}
\begin{Sinput}
> library("rpvm")
> set.seed(1782)
> .PVM.config()
\end{Sinput}
\begin{Soutput}
  host.id    name    arch speed
1  262144 node066 LINUX64  1000
2  524288 node020 LINUX64  1000
3  786432 node036 LINUX64  1000
4 1048576 node016 LINUX64  1000
\end{Soutput}
\begin{Sinput}
> .PVM.exit()
\end{Sinput}
\end{Schunk}
\end{Example}

\subsection{Process Spawning and Communication}

The package \pkg{rpvm} uses the master-slave paradigm meaning that one
process is the master task and the others are slave tasks. \pkg{rpvm}
provides a routine to spawn R slaves but these slaves cannot be used
interactively like the slaves in \pkg{Rmpi}. The spawned R slaves
source an R script which contains all the necessary function calls to
set up communication and carry out the computation and after
processing terminate.
PVM uses task IDs (\code{tid}---a positive integer for identifying a task)
and tags for communication (see
also the fundamentals of message passing in
Section~\ref{sec:messagepassing}).

\subsubsection{Process Management  Functions}
\begin{description}
\item[\code{.PVM.spawnR(slave, ntask = 1, ...)}] spawns \code{ntask}
  copies of an \code{slave} R process. \code{slave} is a character
  specifying the source file for the R slaves located in the package's
  demo directory (the default). There are more
  parameters indicated by the \ldots (we refer to
  \cite{nali07rpvm}). The \code{tids} of the successfully spawned R
  slaves are returned.
\item[\code{.PVM.mytid()}] returns the \code{tid} of the calling
  process.
\item[\code{.PVM.parent()}] returns the \code{tid} of the parent
  process that spawned the calling process.
\item[\code{.PVM.siblings()}] returns the \code{tid} of the processes
  that were spawned in a single spawn call.
\item[\code{.PVM.pstats(tids)}] returns the status of the PVM
  process(es) with task ID(s) \code{tids}.
\end{description}

\subsection{Built-in High Level Functions}

\pkg{rpvm} provides only two high level functions. One of them is a
function to get or set values in the virtual machine settings. This is
certainly because each new high level functions needs separate source
files for the slaves and this is not what developers do intuitively. 

\subsubsection{High Level Functions}
\begin{description}
\item[\code{PVM.rapply(X, FUN = mean, NTASK = 1))}] Apply a function
  \code{FUN} to the rows of a matrix \code{X} in
  parallel using \code{NTASK} tasks.
\item[\code{PVM.options(option, value)}] Get or set values of libpvm
  options (for details see \cite{nali07rpvm} and \cite{geist94pvm}).
\end{description}

Example~\ref{ex:rpvm-rapply} shows how the rows of a matrix \code{X}
can be summed up in parallel via \code{PVM.rapply()}.

\begin{Example} Using PVM.rapply
\begin{Schunk}
\begin{Sinput}
> n <- 8
> X <- matrix(rnorm(n * n), nrow = n)
> round(X, 3)
\end{Sinput}
\begin{Soutput}
       [,1]   [,2]   [,3]   [,4]   [,5]   [,6]   [,7]   [,8]
[1,] -0.200 -0.183  0.560  1.286  0.468  0.502  0.874 -0.778
[2,] -1.371  0.484 -0.498  1.788  0.534 -0.566  0.152 -1.307
[3,]  1.041  0.484  0.399  0.580  0.586 -0.660  1.833 -1.405
[4,] -1.117 -0.893  0.408 -1.612  0.486  0.644  0.422 -1.639
[5,]  1.397 -0.237 -1.287 -0.122 -1.076  0.225 -0.047  0.020
[6,] -0.046  0.537 -1.287 -0.089  0.564  2.671 -0.715 -0.901
[7,]  1.085  0.706 -0.034  0.929  0.057 -2.402 -1.233  1.135
[8,]  0.605 -0.076 -0.554  1.385 -0.436  0.249  0.338  1.369
\end{Soutput}
\begin{Sinput}
> PVM.rapply(X, sum, 3)
\end{Sinput}
\begin{Soutput}
Try to spawn tasks...
Work sent to  524289 
Work sent to  786433 
Work sent to  1048577 
[1]  2.5291297 -0.7840539  2.8572559 -3.3002539 -1.1275728
[6]  0.7329710  0.2431164  2.8806456
\end{Soutput}
\end{Schunk}
\label{ex:rpvm-rapply}
\end{Example}

Before we explain how \code{PVM.rapply} works the following \pkg{rpvm}
functions have to be explained:

\subsubsection{Other Important functions}
\begin{description}
\item[\code{.PVM.initsend()}] clears the default send buffer and
  prepares it for packing a new message.
\item[\code{.PVM.pkstr(data = "")}] and \code{.PVM.pkint(data =
    0, stride = 1)} are low level correspondents of the PVM packing
    routines (see \cite{geist94pvm} for more information on packing
    data). 
\item[\code{.PVM.pkdblmat(data)}] packs a double matrix including
  the dimension information. There are more packing routines
  available. They are explained in \cite{nali07rpvm}.
\item[\code{.PVM.send(tid, msgtag)}] sends the message stored in the
  active buffer to the PVM process identified by \texttt{tid}. The
  content is labeled by the identifier \texttt{msgtag}.
\item[\code{.PVM.recv(tid = -1, msgtag = -1)}] blocks the process
  until a message with label \texttt{msgtag} has arrived from
  \texttt{tid}. $-1$ means any. The receive buffer is cleared
  and the received message is placed there instead.
\item[\code{.PVM.upkstr(), .PVM.upkint(), .PVM.upkdblvec()}] and
  others are the corresponding unpack functions to the
  pack functions explained before.
\item[\texttt{.PVM.gather(x, count = length(x), msgtag, group,
    rootginst = 0)}] gathers data distributed on the nodes (\code{x})
  to a 
  specific process (mostly the root) into a single array. It performs
  a send of messages from each member of a group of processes. A
  specific process (the root) accumulates this messages into a single
  vector. 
\item[\texttt{.PVM.scatter(x, count, msgtag, group, rootqinst = 0)}]
  sends to each member of a group a partition of  a vector \code{x}
  from a 
  specified member of the group (mostly the root) where \texttt{count}
  is an integer specifying the number of elements to be sent to each
  member. 
\end{description}

Let us now examine the function to see how parallel programs using
\pkg{rpvm} can be written.

\begin{Example} PVM.rapply master routine
\begin{Scode}

PVM.rapply <- function (X, FUN = mean, NTASK = 1)
{
    WORKTAG <- 22
    RESULTAG <- 33
    if (!is.matrix(X)) {
        stop("X must be a matrix!")
    }
    if (NTASK == 0) {
        return(apply(X, 1, FUN))
    }
    end <- nrow(X)
    chunk <- end%/%NTASK + 1
    start <- 1
    mytid <- .PVM.mytid()
    children <- .PVM.spawnR(ntask = NTASK, slave = "slapply")
    if (all(children < 0)) {
        cat("Failed to spawn any task: ", children, "\n")
        .PVM.exit()
    }
    else if (any(children < 0)) {
        cat("Failed to spawn some tasks.  Successfully spawned ",
            sum(children > 0), "tasks\n")
        children <- children[children > 0]
    }
    for (id in 1:length(children)) {
        .PVM.initsend()
        range <- c(start, ifelse((start + chunk - 1) > end, end,
            start + chunk - 1))
        work <- X[(range[1]):(range[2]), , drop = FALSE]
        start <- start + chunk
        .PVM.pkstr(deparse(substitute(FUN)))
        .PVM.pkint(id)
        .PVM.pkdblmat(work)
        .PVM.send(children[id], WORKTAG)
        cat("Work sent to ", children[id], "\n")
    }
    partial.results <- list()
    for (child in children) {
        .PVM.recv(-1, RESULTAG)
        order <- .PVM.upkint()
        partial.results[[order]] <- .PVM.upkdblvec()
    }
    .PVM.exit()
    return(unlist(partial.results))
}
\end{Scode} 
\label{ex:rpvm-rapplymaster}
\end{Example}
The corresponding slave R script (slapply.R) looks as follows:

\begin{Example} PVM.rapply slave routine
\begin{Scode}

library (rpvm)
WORKTAG <- 22
RESULTAG <- 33
mytid  <- .PVM.mytid ()
myparent  <- .PVM.parent ()
## Receive work from parent (a matrix)
buf <- .PVM.recv (myparent, WORKTAG)
## Function to apply
func  <- .PVM.upkstr ()
cat ("Function to apply: ", func, "\n")
## Order
order <- .PVM.upkint ()
partial.work <- .PVM.upkdblmat ()
print (partial.work)
## actually work, take the mean of the rows
partial.result <- apply (partial.work, 1, func)
print (partial.result)
## Send result back
.PVM.initsend ()
.PVM.pkint (order)
.PVM.pkdblvec (partial.result)
.PVM.send (myparent, RESULTAG)
## Exit PVM
.PVM.exit ()
## Exit R
q (save="no")
\end{Scode}
\label{ex:rpvm-rapplyslave}
\end{Example}


Example~\ref{ex:rpvm-rapplymaster} shows the
master routine of \code{PVM.rapply()}. This function takes a matrix,
the function which is going to be applied and the number of processors
to use as arguments. At first 
the message tags are specified. These tags are necessary to uniquely
identify messages sent in a message passing environment. After
input validation \code{NTASK} child processes are spawned using the
\code{.PVM.spawnR()} command. After initializing the send buffer the
partitioned data (packed in the buffer using the \code{.PVM.pk*}
commands) is send to the corresponding child processes represented by
their task IDs using
\texttt{.PVM.send()}. PVM uses these task identifiers (tid) to
address pvmds, tasks, and groups of tasks within a virtual
machine.

Meanwhile the spawned slave processes (see
Example~\ref{ex:rpvm-rapplyslave}) have been idle because they wait
for input (\code{.PVM.receive()} is a blocking command). After
receiving data from the parent the data gets unpacked. Now the slaves
apply the given function to their part of the matrix. Finally another
send is initialized to provide the results to the parent process and
the slaves are detached from the virtual machine by calling a
\texttt{.PVM.exit()}. 

\subsection{Conclusion}

The \code{PVM.rapply()} example shown in this section followed the
Single Program Multiple Data (SPMD) paradigm. Data is split into
different parts which are sent to different processes. I/O is handled
solely by a master
process. When loading \pkg{rpvm} in an R session this session becomes the
master process. Slaves can easily be spawned provided that there are
working slave scripts available.

We encountered no problems when using the routines in \pkg{rpvm}. This
package seem to be rather stable in contrast to \pkg{Rmpi}, where for
unknown reasons the MPI environment sometimes crashed.

A major disadvantage is that the \pkg{rpvm}
package only has two higher level function. One of them can be used
for calculations. That means when using this package for parallel
computing one has 
to deal with low level message passing but which in turn may provide
higher flexibility. New parallel functions can be constructed on the
basis of the provided interface. The \code{PVM.rapply} code
can be taken as a template for further routines. 

Another disadvantage is the missing support for interactive R
slaves. Parallel tasks have to be created on the basis of separate
slave source files which are sourced on the creation of the slaves.

For further interface functions supplied by the \pkg{rpvm} package, a more
detailed 
description and further examples please consult the package description
\cite{nali07rpvm}.
