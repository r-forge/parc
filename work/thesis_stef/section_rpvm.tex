\section{The rpvm package}
\label{sec:rpvm}
Parallel virtual machine uses the message-passing model and makes a
collection of computers appear as a single virtual machine (see section
\ref{sec:PVM} for details).
The package rpvm (\cite{nali07rpvm}) provides an interface to
low-level PVM-functions and
a few high-level parallel functions to R. It uses most of the
facilities provided by the PVM system. A good selection of these
functions are explained in this section to show how PVM and R can be
used together.

\subsection{Initializing and Status queries}

At first for using the PVM system \textit{pvmd3} has to be
booted. This can done via the command
line or directly within R after loading the rpvm package.


rpvm functions for managing the virtual machine:

\begin{description}
\item[\texttt{.PVM.start.pvmd()}] boots the \textit{pvmd3} daemon. The
  currently running R session becomes the master process. 
\item[\texttt{.PVM.add.hosts(hosts)}] takes a vector of hostnames to
  be added to the current virtual machine. The syntax of the
  hostnames is similiar to the lines of a pvmd hostfile (for details
  see the man page of \textit{pvmd3}). 
\item[\texttt{.PVM.del.hosts()}] simply deletes the given hosts from
  the virtual machine configuration.
\item[\texttt{.PVM.halt()}] shuts down the entire PVM system and exits
  the current R session.
\end{description}
  
It is better to use the Sun Grid Engine (SGE) to boot the PVM parallel
environment as the developer is not engaged with
setting up and booting the environment anymore (see
appendix \ref{app:gridengine} on how to do this).


\textbf{Example:} Query status of PVM \newline
running on cluster@WU using the node.q -- the parallel environment was
started with SGE using 8 nodes

\begin{Schunk}
\begin{Sinput}
> library("rpvm")
> set.seed(1782)
> .PVM.config()
\end{Sinput}
\begin{Soutput}
  host.id    name    arch speed
1  262144 node057 LINUX64  1000
2  524288 node044 LINUX64  1000
3  786432 node002 LINUX64  1000
4 1048576 node053 LINUX64  1000
\end{Soutput}
\begin{Sinput}
> mytid <- .PVM.mytid()
> .PVM.tasks()
\end{Sinput}
\begin{Soutput}
$tid
[1] 262146

$parent
[1] 0

$host
[1] 262144

$status
[1] "Unknown"

$name
[1] ""
\end{Soutput}
\begin{Sinput}
> .PVM.pstats(mytid)
\end{Sinput}
\begin{Soutput}
262146 
  "OK" 
\end{Soutput}
\begin{Sinput}
> .PVM.pstats(mytid + 9)
\end{Sinput}
\begin{Soutput}
       262155 
"Not Running" 
\end{Soutput}
\end{Schunk}


\subsection{Built-in high level functions}

rpvm provides two high level functions: 

\begin{description}
\item[\texttt{PVM.rapply(X, FUN = mean, NTASK = 1))}] Apply a function
  \texttt{FUN} to the rows of a matrix \texttt{X} in
  parallel using \texttt{NTASK} tasks.
\item[PVM.options(option, value)] Get or set values of libpvm options
  (for details see \cite{nali07rpvm} and \cite{geist94pvm}).
\end{description}


\textbf{Example:} Using PVM.rapply\newline
running on cluster@WU using the node.q -- the parallel environment was
started with SGE using 8 nodes

\begin{Schunk}
\begin{Sinput}
> n <- 8
> X <- matrix(rnorm(n * n), nrow = n)
> round(X, 3)
\end{Sinput}
\begin{Soutput}
       [,1]   [,2]   [,3]   [,4]   [,5]   [,6]   [,7]   [,8]
[1,] -0.200 -0.183  0.560  1.286  0.468  0.502  0.874 -0.778
[2,] -1.371  0.484 -0.498  1.788  0.534 -0.566  0.152 -1.307
[3,]  1.041  0.484  0.399  0.580  0.586 -0.660  1.833 -1.405
[4,] -1.117 -0.893  0.408 -1.612  0.486  0.644  0.422 -1.639
[5,]  1.397 -0.237 -1.287 -0.122 -1.076  0.225 -0.047  0.020
[6,] -0.046  0.537 -1.287 -0.089  0.564  2.671 -0.715 -0.901
[7,]  1.085  0.706 -0.034  0.929  0.057 -2.402 -1.233  1.135
[8,]  0.605 -0.076 -0.554  1.385 -0.436  0.249  0.338  1.369
\end{Soutput}
\end{Schunk}

Let us examine the function to see how parallel execution of rpvm
functions should be done. 

\begin{Schunk}
\begin{Sinput}
> PVM.rapply
\end{Sinput}
\begin{Soutput}
function (X, FUN = mean, NTASK = 1) 
{
    WORKTAG <- 22
    RESULTAG <- 33
    if (!is.matrix(X)) {
        stop("X must be a matrix!")
    }
    if (NTASK == 0) {
        return(apply(X, 1, FUN))
    }
    end <- nrow(X)
    chunk <- end%/%NTASK + 1
    start <- 1
    mytid <- .PVM.mytid()
    children <- .PVM.spawnR(ntask = NTASK, slave = "slapply")
    if (all(children < 0)) {
        cat("Failed to spawn any task: ", children, "\n")
        .PVM.exit()
    }
    else if (any(children < 0)) {
        cat("Failed to spawn some tasks.  Successfully spawned ", 
            sum(children > 0), "tasks\n")
        children <- children[children > 0]
    }
    for (id in 1:length(children)) {
        .PVM.initsend()
        range <- c(start, ifelse((start + chunk - 1) > end, end, 
            start + chunk - 1))
        work <- X[(range[1]):(range[2]), , drop = F]
        start <- start + chunk
        .PVM.pkstr(deparse(substitute(FUN)))
        .PVM.pkint(id)
        .PVM.pkdblmat(work)
        .PVM.send(children[id], WORKTAG)
        cat("Work sent to ", children[id], "\n")
    }
    partial.results <- list()
    for (child in children) {
        .PVM.recv(-1, RESULTAG)
        order <- .PVM.upkint()
        partial.results[[order]] <- .PVM.upkdblvec()
    }
    .PVM.exit()
    return(unlist(partial.results))
}
\end{Soutput}
\end{Schunk}

The corresponding slave R script looks as follows.

\begin{Schunk}
\begin{Sinput}
> library(rpvm)
> WORKTAG <- 22
> RESULTAG <- 33
> mytid <- .PVM.mytid()
> myparent <- .PVM.parent()
> buf <- .PVM.recv(myparent, WORKTAG)
> func <- .PVM.upkstr()
> cat("Function to apply: ", func, "\n")
> order <- .PVM.upkint()
> partial.work <- .PVM.upkdblmat()
> print(partial.work)
> partial.result <- apply(partial.work, 1, func)
> print(partial.result)
> .PVM.initsend()
> .PVM.pkint(order)
> .PVM.pkdblvec(partial.result)
> .PVM.send(myparent, RESULTAG)
> .PVM.exit()
> q(save = "no")
\end{Sinput}
\end{Schunk}


New rpvm functions used in \texttt{PVM.rapply()}:

\begin{description}
\item[\texttt{.PVM.spawnR(slave, ntask = 1, ...}] spawns \texttt{ntask} copies
  of an executable or \texttt{slave} R processes. There are more
  parameters indicated by the \ldots (again I refer to \cite{nali07rpvm}). In
  PVM.rapply \texttt{NTASK} slaves sourcing ``slapply.R'' are
  spawned. The \texttt{tids} of the successfully spawned R slaves are
  returned.
\item[\texttt{.PVM.exit()}] tells the local PVM daemon that this process leaves
  the virtual machine.
\item[\texttt{.PVM.initsend()}] clears the default send buffer and
  prepares it for packing a new message.
\item[\texttt{.PVM.pkstr(data = ``'')} and \texttt{.PVM.pkint(data =
    0, stride = 1)}] are low-level correspondents of the PVM packing
    routines (see \cite{geist94pvm} for more information on packing
    data). 
\item[\texttt{.PVM.pkdblmat(data)}] packs a double matrix including
  the dimension information. There are more packing routines
  available. They are explained in \cite{nali07rpvm}.
\item[\texttt{.PVM.parent()}] returns the tid of the parent process
  that spawned the calling process.
\item[\texttt{.PVM.send(tid, msgtag)}] sends the message stored in the
  active buffer to the PVM process identified by \texttt{tid}. The
  content is labeled by the identifier \texttt{msgtag}.
\item[\texttt{.PVM.recv(tid = -1, msgtag = -1)}] blocks the process
  until a message with label \texttt{msgtag} has arrived from
  \texttt{tid}. -1 means any. The receive buffer is cleared
  and the received message is placed there instead.
\item[\texttt{.PVM.upkstr(), .PVM.upkint(), .PVM.upkdblvec()}] and other unpack data functions are
  the correspondig unpack functions to the pack functions explained above. 
\end{description}

\texttt{PVM.rapply()} takes a matrix, the function which is going to
be applied and the number of processors to use as arguments. At first
the message tags are specified. These tags are necessary to uniquely
identify messages sent in a message-passing environment. After
input validation \texttt{NTASK} child processes are spawned using the
\texttt{.PVM.spawnR()} command. After initializing the send buffer the
partitioned data (packed in the buffer using the \texttt{.PVM.pk*}
commands) is send to the corresponding child processes represented by
their task IDs using
\texttt{.PVM.send()}. PVM uses these task identifiers (TID) to
address pvmds, tasks, and groups of tasks within a virtual machine.
Meanwhile the spawned slave processes have been idle because they wait
for input (\texttt{.PVM.receive()} is a blocking command). After
receiving data from the parent the data gets unpacked. Now the slaves
apply the given function to their part of the matrix. At last another
send is initialized to provide the results to the parent process and
the slaves are detached from the virtual machine by calling a
\texttt{.PVM.exit()}. 

\subsection{Other important functions}

To complete the set of important functions supplied by the rpvm
package two more functions have to be explained:

\begin{description}
\item[\texttt{.PVM.gather(x, count = length(x), msgtag, group,
    rootginst = 0}] gathers data distributed on the nodes (x) to a
  specific process (mostly the root) into a single array. It performs
  a send of messages from each member of a group of processes. A
  specific process (the root) accumulates this messages into a single vector.
\item[\texttt{.PVM.scatter(x, count, msgtag, group, rootqinst = 0}]
  sends to each member of a group a partition of  a vector x from a
  specified member of the group (mostly the root) where \texttt{count}
  is an integer specifying the number of elements to be sent to each
  member. 
\end{description}

\subsection{conclusion}

The \texttt{PVM.rapply()} example shown in this section followed the Single Program
Multiple Data (SPMD) paradigm. Data is splitted into different parts
which are sent to different processes. I/O is handled by a master
process. When loading rpvm in an R session this session becomes the
master process. Slaves can easily be spawned provided that there are
working slave scripts available. A major disadvantage is that the rpvm
package only has two higher-level function. One of them can be used
for calculations. That means when using this package for HPC one has
to deal with low-level message-passing which in turn provides high
flexibility. New parallel functions can be constructed on the basis of
the provided interface.

For further interface functions supplied by the rpvm package, a more detailed
description and further examples please consult the package description
\cite{nali07rpvm}.
