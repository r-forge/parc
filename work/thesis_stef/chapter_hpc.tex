\chapter{Parallel Computing}
\section{Introduction}

Massive parallel machines (MPPs) or network of workstations (NOW)


Furthermore greater speed implicates greater memory, otherwise you have
the situation that a computer capable of performing trillions of
operations in each second only has access to a small memory which is
rather useless for solving data driven calculations.


critical problems in the development of parallel computing (see
pacheco p5-6)
\begin{itemize}
\item decide on and implement an interconnection network for the
  processors
\item design and implement system software for the hardware
\item devise algorithms and data structures for solving th problem
\item divide the algorithms and data structures up into subproblems
\item identify the communications that will be needed among the
  subproblems, and
\item assign subproblems to processors and memory modules
\end{itemize}

\section{Computer Architecture}

traditionally computers can be  categorized as follows:
\begin{itemize}
\item personal computers
\item workstations
\item mini computers
\item mainframe computers
\item High performance computing servers (super computers)
\end{itemize}

This has changed rapidly in the last years, as the requirements of
people went through an interesting process. Not only schientists
needed more and more computing power but also endusers. 

In this diploma thesis we will have a closer look on the last category
namely high performance computing servers. I start with a definition
of what is a HPC server and continue with a taxonomy of such
machines. I'm closing this section with a short description of the
Servers I used to produce all the output in this thesis.

\subsection{Beyond Flynn's Taxonomy}

Providing a high-level standard for programming HPC applications is a
major challenge nowadays. There are a variety of architectures for
large-scale computing. They all have specific features and 
therefore there should be a taxonomy in which such architectures can
be classified. \cite{flynn72sco} classified architectures on the
presence of single or multiple either instructions or data streams
known as Flynn's taxonomy:
\begin{description}
\item[Single Instruction Single Data (SISD)]
  For a long time this type of architecture of CPUs, developed by the
  mathematician John von Neumann, was the standard. This computers are
  known as serial computers. 
\item[Multiple Instruction Single Data (MISD)] the theoritical possibility
  of applying multiple instructions on a single datum is generally
  impractical.
\item[Single Instruction Multiple Data (SIMD)] 
  a single instruction is applied by multiple processors to different
  data.
\item[Multiple Instruction Multiple Data (MIMD)] processors apply
  different instructions on different data. 
\end{description}

But these distinctions are insufficient for classifying modern
computers according to \cite{duncan90survey}. For example there are
pipelined vector processors capable of concurrent arithmetic execution and
manipulating hundreds of vector elements in parallel.

Therefore \cite{duncan90survey} defines that a parallel architecture
provides an explicit, high-level framework for 
the development of parallel programming solutions by providing
multiple processors that cooperate to solve problems through
concurrent execution:

\begin{description}
\item[Synchronos architectures] coordinate concurrent operations in
  lockstep through global clocks, central control unit, or vector unit
  controllers. These architectures involve pipelined vector processors
  (characterized by multiple, pipelined functional units, which
  implement arithmetic and Boolean operations), SIMD architectures
  (typically a control unit broadcasting a single instruction to
  all processors executing the instruction on local data) and Systolic
  architectures (pipelined multiprocessors in which data flows from
  memory through a network of processors back to memory synchronized
  by a global clock. 
\item[MIMD architectures] consist of multiple processors applying
  different instructions on local (different) data. The MIMD models
  are asynchronous computers although they may be synchronized by
  messages passing through an interconnection network (or by accessing
  data in a shared memory). The advantage is the possibility of
  executing largely independent subcalculations. These architectures
  can further be classified in distributed or shared memory systems
  whereas the first achieves interprocess communication via an
  interconnection network and the last via a global memory each
  processor can address.
\item[MIMD-based architectural paradigms] involve MIMD/SIMD hybrids,
  dataflow architectures, reduction machines, and wavefront arrays. A
  MIMD architecture can be called a MIMD/SIMD hybrid if parts of the
  MIMD architecture are controlled in SIMD fashion (some sort of
  master/slave relation). In dataflow architectures instructions are
  enabled for execution as soon as all of their operands become
  available whereas in reduction architectures an instruction is
  enabled for execution when its results are required as operands for
  another instruction already enabled for execution. Wavefront array
  processors are a mixture of systolic data piplining and asynchronous
  dataflow execution.
\end{description}

\subsection{High performance computing server}

These computers have the highest computation power. They are mainly used
for scientific or technical applications. The number of processors
such a computer can have reaches from 100 to several thousand (mainly
64 bit processors).
You can divide this category of computers in ``vector computers'' or ``parallel
computers''. The amount of memory they have varies according to the type of
these servers or application, which is running on them. The maximum
amount of memory can absolutely be beyond several tera bytes.  

This development started in the 1990's where parallel programming has
become more and more important. With the increasing need of powerful
hardware computer vendors started to build supercomputers capable of
executing more and more instructions in parallel. In the beginning
vector supercomputing system using a single shared memory were widely
used to run large-scale applications. Due to the fact that vector
systems were rather expensive in both purchasing and operation and
bus-based shared-memory parallel systems were technologically bounded
to smaller CPU counts the first distributed-memory parallel platforms (DMPs)
came up. The big advantage was that they were inexpensive and could be
built in many different sizes. DMPs could be rather small like a
connection of workstations connected via a network to form a cluster of
workstations (COW) or arbitrarily large. In fact individual
organizations could buy clusters of a size that suited their
budgets. The big disadvantage DMPs have is the higher transportation
time of messages passing through a network in comparison to a shared
memory system. Although the technology has improved (Infiniband) there
is still space for improvements. 

Current HPC systems are still formed of vector supercomputers
(providing highest level of performance, but only for certain
applications), SMPs with two to over 100 hundred CPUs and DMPs. As
more than one CPU has become commonplace in desktop machines clusters
can nowadays provide both small SMPs and big DMPs. When it comes to a
decition which architecture to buy companies, laboratories or
universities often choose to use a cluster of workstations because the
cost of networking has decreased as I mentioned before and performance
has increased since systems of this kind already dominate the Top500
list. Furthermore, COWs are so successful because they can be
configured from commodity chips and network and built the open-source
Linux operating system and therefore offer high performance for a
really good price.

New developments show that there are interesting alternatives. Some
HPC platforms offer a global addressing scheme and therefore enables
CPUs to directly access the entire system's memory. This can be done
with ccNUMA (cache coherent nonuniform memory access) systems (cite hp
sgi paper and sgi.pdf-luthien) or
global arrays (see section \ref{globalArrays})
They have become an important improvement to distributed memory
paltforms because amongst others it enables compiler driven
parallelizing techniques like OpenMP (see section \ref{OpenMP})to be
used on these systems. 
 
\subsection{Hardware used}

R-Forge specs

2 x Dual Core AMD Opteron 2.4 GHz
12 GB RAM
1.2 TB RAID 5

Cluster specs (node.q, bignode, bignode.q)

4 Nodes (bignodes):
2 x Dual Core Intel XEON CPU 5140 @ 2.33 GHz
16 GB RAM

68 Nodes:
1 x Intel Core 2 Duo CPU 6600 @ 2.4 GHz
4 GB RAM




\section{Parallel Programming Models}
\label{sec:programming_models}

Programmers who like to create an application which should run in
parallel have to face a challenging task. They have to distribute all
the computation involved to a large number of processors. It is
important that this is done in a way so that each of these computation
nodes performs roughly the same work. Furthermore, developers have to
ensure that the data required for the computation is available to the
processor witch as little delay as possible. Therefore some sort of
coordination is required for the locality of data in the memory.
One might think that this could be hard work, and indeed it is. But
there are programming models for HPC already available which make life
easier.

When parallel computing came up there have been a lot of programming
languages developed. Only few of them experienced any real use. There
have also been many implementations of computer vendors for their own
machines. For a long time there has been no standard in sight as no
agreement between hardware vendors and developers emerged. It was
common that application developers have to write seperate HPC
applications for each architecture.

In the 1990's broad commonity efforts produced the first defacto
standards for parallel computing. Commonalities were identified
between all the implementations available and the field of parallel
computing had been understood better. Since then libraries and certain
implementations as well as compiler driven parallelizer have been
developed.

The programming models presented in this section are common and widely
used. Starting with the first generation the Message Passing Interface
(MPI - see \cite{forum94:MPI}) and Parallel Virtual Machine (PVM - see
\cite{geist94pvm})
are going to be described. The second generation of HPC programming
models involve OpenMP (see \cite{openMP05}) and Global Arrays (GA - see
\cite{nieplocha96gan}). These programming models are not the only ones
but are commonly used. 

\subsection{The Message Passing Interface (MPI)}
\label{MPI}

In 1994 the Message Passing Interface Forum (MPIF) has defined a set
of library interface standards for message passing. Over 40
organizations participated in the discussion which started in
1993. The aim was to develop a widely standard for writing
message-passing programs. The standard was called the Message Passing
Interface (see \cite{forum94:MPI}).

As mentioned in section \ref{sec:programming_models} before there
were a lot of different programming models and languages for high
performance computing available. With MPI a standard was established
that should be practical, portable, efficient and flexible. After
releasing Version 1.0 in 1994 the MPIF continued to correct errors and
made clarifications in the MPI document so that in June 1995 Version
1.1 of MPI had been released.

Further corrections and clarifications have been made Version 1.2 of
MPI and with Version 2 completely new types of functionality has been
added (see \cite{forum94:MPI-2} for details).

MPI is now a library which can be called from various programming
languages like C or Fortran. It contains all the infrastructure for
inter process communication.



\subsubsection{Why using message-passing?}
\label{sec:why_m-p}

If one uses a distributed memory machine (DMP) as platform for
carrying out computations in parallel, data hs to be transmitted from
one computation node to the other. With MPI there are higher level
routines and abstractions available which build upon lower level
message-passing routines.
  
A message-passing function is a function which explicitly transmits
data from one process to another. With these
functions, creating parallel programs can be extremely
efficient. Developers don't have to care about low level
message-passing anymore.

\subsubsection{Implementations of the MPI standard}


\begin{description}
\item[LAM/MPI] is an open-source implementation of MPI. It runs on a
  network of UNIX machines connected via a local area network or via
  the Internet. On each node runs one UNIX daemon which is a
  micro-kernel plus a dozen system processes responsible for network
  communication among other things. The micro-kernel is
  responsible for the communication between local processes (see
  \cite{burns94lam}). The LAM/MPI implementation includes all of the
  MPI-1.2 but not everything of the MPI-2 standard. LAM/MPI is
  available at \url{http://www.lam-mpi.org/}.
\item[MPICH] is a freely available complete implementation of the MPI
  specification (MPICH1 includes all MPI-1.2 and MPICH2 all MPI-2
  specifications). The main goal was to deliver high performance with
  respect to portability. ``CH'' in MPICH stands for ``Chameleon''
  which means adaptability to one's environment and high performance
  as chameleons are fast (see \cite{gropp96mpich}). MPICH had been
  developed during the discussion process of the MPIF and had immediately been
  available to the community after the standard had been
  confirmed. There are many other implementaions of MPI based on MPICH
  available because of its high and easy portability.
\item[OpenMPI] (see \cite{gabriel04:_open_mpi}) 
\end{description}

for setting up these implementations in Debian GNU Linux and cluster
jobscripts see appendix \ref{app:mpi_imp}. 

\subsection{Parallel Virtual Machine (PVM)}
\label{PVM}



\subsection{OpenMP}
\label{OpenMP}



ie. High Performance Fortran~(HPF)

Dividing the data among the processors and having each processor apply
the same program to its portion of the data.

MPI

\subsection{Global Arrays}
\label{globalArrays}


\section{Implementation of parallel applications in R}
\label{sec:parallel_R}

\subsection{Example: Matrix multiplication}
\subsection{The Rmpi package}
\subsection{The Rpvm package}
\subsection{SNOW package}
\subsection{Building an OpenMP shared library}
\subsection{Comparison}

\section{Conclusion}

TODO: performance in a closer sense

      benchmarks

      LINPACK benchmark (implement LINPACK Java in R
      http://www.top500.org/about/linpack

      performance in a wider sense
