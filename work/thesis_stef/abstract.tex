%\thispagestyle{empty}
\begin{center}
  {\Large Abstract\\}
  {\Large{\bf \Title}\\}
  %\medskip
\end{center}
\bigskip

In the 1990s the Beowulf project smoothed to way for massively parallel
computing as access to parallel computing power became affordable for
research institutions and the industry. But the massive breakthrough
of parallel computing has still not occurred. This is because two
things were missing: low cost parallel computers and simple to use
parallel programming models. However, with the introduction of
multicore processors for mainstream computers and implicit parallel
programming models like OpenMP a fundamental change of the way
developers design and build their software applications is taken
place---a change towards parallel computing.

This thesis gives an overview of the field of high performance
computing with a special focus on parallel computing in connection
with the R environment for statistical computing and
graphics. Furthermore, an introduction to parallel computing using
various extensions to R is given.

The major contribution of this thesis is the package called
\pkg{paRc}, which contains an interface to OpenMP and provides a
benchmark environment to compare various parallel programming models like
MPI or PVM with each other as well as with highly optimized (BLAS)
libraries. The dot product matrix multiplication was chosen as the
benchmark task as it is a prime example in parallel computing. 

Eventually a case study in computational finance is presented in this
thesis. It deals with the pricing of derivatives (European call
options) using parallel Monte Carlo simulation. 

%\thispagestyle{empty}
\begin{center}
  {\Large Kurzfassung\\}
  {\Large{\bf \Titel}\\}
  %\medskip
\end{center}
\bigskip

In den 1990er Jahren ebnete das Beowulf Projekt den Weg f\"ur massives
Parallelrechnen, weil dadurch der Zugang zu parallelen Rechenresourcen f\"ur
viele 
Forschungsinstitute und der Industrie erst leistbar wurde. Trotzdem
war der gro\ss{}e Durchbruch lange nicht erkennbar. Ausschlaggebend
daf\"ur waren zwei Aspekte:
Erstens gab es bisher keine g\"unstigen Parallelrechner und zweitens
waren auch keine einfachen Parallelprogrammiermodelle vorhanden. Die
Einf\"uhrung von Multicoreprozessoren f\"ur Desktopsysteme und
impliziter Programmiermodelle wie OpenMP f\"uhrte zu einem Umbruch,
der auch Auswirkungen auf die Softwareentwicklung hat. Der Trend geht
dabei immer mehr in Richtung Parallelrechnen.

Diese Diplomarbeitet bietet einen \"Uberblick \"uber das Feld des
Hochleistungsrechnens und im Speziellen des Parallelrechnens unter
Verwendung von R, einer Softwareumgebung f\"ur statistisches Rechnen
und Grafiken. Dar\"uber- hinaus beinhaltet diese Arbeit eine Einf\"uhrung %%SELBSTGEMACHTERUMBRUCH
in paralleles Rechnen unter Verwendung mehrerer Erweiterung zu R.

Der Kern dieser Arbeit ist die Entwicklung eines
Paketes namens  \pkg{paRc}. Diese Erweiterung beinhaltet eine
Schnittstelle zu OpenMP und stellt eine Benchmark Umgebung zur
Verf\"ugung, mit der verschiedene Paralleleprogrammiermodelle wie MPI oder
PVM sowie hochoptimierte Bibliotheken (BLAS) miteinander verglichen
werden k\"onnen. Als Aufgabe f\"ur diese Benchmarks wurde das
klassische Beispiel aus dem Bereich des Parallelrechnens, n\"amlich
die Matrix Multiplikation, gew\"ahlt.

Schlussendlich wird in dieser Arbeit eine Fallstudie aus der
computationalen Finanzwirtschaft pr\"asentiert. Im Mittelpunkt dieser
Fallstudie steht die Bewertung von Derivaten (Europ\"aische Call
Optionen) unter Verwendung paralleler Monte Carlo Simulation.
