
\chapter{Sun Grid Engine}
\label{app:gridengine}

A detailed documentation is available from the Sun Microsystems
website (\cite{sge07}). In this appendix the fundamental steps for
running parallel routines with R are explained.

Important for running applications with the grid engine are job
scripts. They work in the same way like shell scripts but additionally
contain control sequences for the grid engine. These control sequences
always start with \code{\#\$} 
%$  
followed by a command flag and an argument.

Often used flags are:
\begin{description}
\item[\code{-N}] defines that the following argument is the name of
  the job.  
\item[\code{-q}] defines the queue to use.
\item[\code{-pe}] defines which parallel environment is to be started
  and how many nodes have to be reserved.
\end{description}

\section{LAM/MPI Parallel Environment}

To use the LAM/MPI parallel environment (pe) with the grid engine the
following has to be inserted into the user's .bashrc:

\begin{verbatim}

export LD_LIBRARY_PATH=/home/stheussl/lib/lam-gcc/lib:$LD_LIBRARY_PATH
export LAMHOME=/home/stheussl/lib/lam-gcc

\end{verbatim}

where the paths point to the corresponding LAM/MPI installation.

\subsubsection{Cluster Job Scripts}

A job script for starting a LAM/MPI application looks like this:

\begin{verbatim}

#$ -N Rmpi-example
# use parallel environment lam with 12 nodes
#$ -pe lam 12
# run job on queue node.q to have a homogeneous processing environment 
#$ -q node.q

R --vanilla < /path/to/my/Rmpi-example.R
\end{verbatim}

This job of the name ``Rmpi-example'' starts a parallel LAM/MPI environment
with 12 nodes on \code{node.q} of cluster@WU. Eventually R is
called sourcing a specific file containing parallel R code.

\section{PVM Parallel Environment}

To use the PVM pe with the grid engine the following has to be
inserted into the user's .bashrc:

\begin{verbatim}

# PVM>
# you may wish to use this for your own programs (edit the last
# part to point to a different directory f.e. ~/bin/_\$PVM_ARCH.
#
if [ -z $PVM_ROOT ]; then
    if [ -d /home/stheussl/lib/pvm3 ]; then
        export PVM_ROOT=/home/stheussl/lib/pvm3
    else
        echo "Warning - PVM_ROOT not defined"
        echo "To use PVM, define PVM_ROOT and rerun your .bashrc"
    fi
fi

if [ -n $PVM_ROOT ]; then
    export PVM_ARCH=`$PVM_ROOT/lib/pvmgetarch`

# uncomment one of the following lines if you want the PVM commands
# directory to be added to your shell path.
#
       export PATH=$PATH:$PVM_ROOT/lib            # generic
       export PATH=$PATH:$PVM_ROOT/lib/$PVM_ARCH  # arch-specific
#
# uncomment the following line if you want the PVM executable directory
# to be added to your shell path.
#
       export PATH=$PATH:$PVM_ROOT/bin/$PVM_ARCH
fi

\end{verbatim}


\subsubsection{Cluster Job Scripts}

A job script for starting a PVM application looks like this:

\begin{verbatim}

#$ -N rpvm-test
# use parallel environment pvm with 8 nodes
#$ -pe pvm 8 
# run job on queue node.q to have a homogeneous processing environment 
#$ -q node.q

R --vanilla < /path/to/my/rpvm-example.R

\end{verbatim}

This job of the name ``rpvm-test'' starts a parallel PVM environment 
with 12 nodes on \code{node.q} of cluster@WU. Eventually R is
called sourcing a specific file containing parallel R code.
