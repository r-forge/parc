
\documentclass[notes=show,beamer]{beamer}

\usepackage{graphicx}

%gute: Goettingen, Copenhagen, Malmoe, Montpellier, Warsaw, Singapore (minimalversion: Pittsburgh)
\usetheme{Malmoe}

\title{Parallel Computations in R}
\author{Stefan Theussl}
\institute{Department of Statistics and Mathematics\newline
Vienna University of Economics and Business Administration}
\date{\today}

\begin{document}

\frame{
\titlepage
}
\section*{outline}
\frame{
\frametitle{Outline}
\tableofcontents}


\section{Computer Architecture}
\frame{\frametitle{Computer Architecture}}

\subsection{Beyond Flynn's Taxonomy}

\frame{
\frametitle{Flynns Taxonomy}

\begin{description}
\item[Single Instruction Single Data (SISD)]
  For a long time this type of architecture of CPUs, developed by the
  mathematician John von Neumann, was the standard. This computers are
  known as serial computers. 
\item[Multiple Instruction Single Data (MISD)] the theoritical possibility
  of applying multiple instructions on a single datum is generally
  impractical.
\item[Single Instruction Multiple Data (SIMD)] 
  a single instruction is applied by multiple processors to different
  data.
\item[Multiple Instruction Multiple Data (MIMD)] processors apply
  different instructions on different data. 
\end{description}

see \cite{flynn72sco}
}

\frame{
\frametitle{Beyond Flynns Taxonomy}

\cite{duncan90survey} tries to define a taxonomy explicit for parallel
frameworks:

\begin{description}
\item[Synchronos architectures] coordinate concurrent operations in
  lockstep through global clocks, central control unit, or vector unit
  controllers.  
\item[MIMD architectures] consist of multiple processors applying
  different instructions on local (different) data. The MIMD models
  are asynchronous computers although they may be synchronized by
  messages passing through an interconnection network (or by accessing
  data in a shared memory). 
\item[MIMD-based architectural paradigms] involve MIMD/SIMD hybrids,
  dataflow architectures, reduction machines, and wavefront arrays.
\end{description}

}

\subsection{High Performance Computing Systems}
\frame{
\frametitle{Current HPC systems}

\begin{itemize}
\item vector supercomputers
\item shared memory computing servers
\item distributed memory computing servers (NOW or COW)
\end{itemize}

}

\frame{
\frametitle{platforms@WU}

\begin{itemize}
\item AMD Opteron Server of Stat-Math\newline
2 x Dual Core AMD Opteron 2.4 GHz\newline
12 GB RAM\newline
1.2 TB RAID 5
\item Cluster of FIRM\newline
4 Nodes (bignodes):\newline
2 x Dual Core Intel XEON CPU 5140 @ 2.33 GHz\newline
16 GB RAM\newline
68 Nodes:\newline
1 x Intel Core 2 Duo CPU 6600 @ 2.4 GHz\newline
4 GB RAM
\end{itemize}

}


\section{Parallel Programming Models}

\frame{
\frametitle{Parallel Programming Models}

\begin{itemize}

\item The Message Passing Interface (MPI)

\item Parallel Virtual Machine (PVM)

\item OpenMP

\end{itemize}
}

\subsection{MPI}

\frame{
\frametitle{The Message Passing Interface Standard}

\begin{itemize}
\item distributed memory systems
\item a set of library interface standards
\item callable from various programming languages
\item difficult to implement
\end{itemize}

}


\frame{
\frametitle{MPI Implementations}

\begin{itemize}
\item LAM/MPI
\item MPICH
\item Open MPI
\end{itemize}

}


\subsection{OpenMP}
\frame{
\frametitle{OpenMP}

\begin{itemize}
\item shared memory systems
\item simple scalable programming
\item set of compiler directives (GNU Compiler since 4.2, Intel
  Compiler)
\item fork join execution
\end{itemize}

}

\section{Package paRc}
\subsection{Parallel Computations in R}
\frame{
\frametitle{Parallel Computations in R - paRc}

The R package paRc provides
\begin{itemize}
\item high level functions for parallel applications
\item a framework for performance evaluation
\end{itemize}

}

\frame{
\frametitle{Parallel Computations in R - paRc}

Todo:
\begin{itemize}
\item Parallel Monte Carlo Simulation (Intel Paper)
\item Parallel GRASP
\item key figures like speedup
\end{itemize}
}

\subsection{Literature}

\frame{
\frametitle{Literature so far}

\begin{thebibliography}{9}


\bibitem{hpc} Laurence T. Yang, Minyi Guo:
  \emph{High Performance Computing - Paradigm and Infrastructure},
  Wiley, 2006

\bibitem{lam} John L. Hennessy and David A. Patterson:
  \emph{Computer Architecture - A quantitative Approach},
  Morgan Kaufmann, 2006

\bibitem{mpi} Peter S. Pacheco:
  \emph{Parallel Programming with MPI},
  The MIT Press, 1994

\bibitem{pvm} Geist, Beguelin, Dongarra:
  \emph{PVM Parallel Virtual Machine},
  The MIT Press, 1994

\end{thebibliography}


}

\begin{frame}

\textbf{Thank you for your attention}

\end{frame}


%\bibliographystyle{plainnat}
%\bibliography{hpcreferences}
\end{document}
