\chapter[Parallel Monte Carlo Simulation]{Option Pricing using
  Parallel Monte Carlo Simulation}
\label{chap:options}
\section{Introduction}
\label{sec:optionintro}

The search for competitive advantages and especially in economics and in
financial markets is in man's nature. The biggest advantage a trader
in financial markets can have is to know something before all
other market participants. On the stock exchange quickness and an advantage
in time are the key factors to be successful. On today's financial
markets more and more trades are made by software and their
algorithms (this is called algorithmic trading) or even artificial
intelligence (\cite{bloomberg:hal9000}). A human trader could
process five to ten complex 
transactions per minute software in contrast could process several
thousand transactions per second and parallelism preconditioned, look
after several market places simultaneously.


The performance of the electronic exchange system is of great
importance nowadays. Every millisecond counts and therefore financial
firms keep their trading software as close to the trading system as
possible (\cite{wstonline:cuttingedge}). This is because investors
want to keep the time
between the input of a trade until its confirmation (this is called
round tripping) at a minimum. Recently the Gruppe Deutsche B\"orse
upgraded their trading 
system Xetra to offer more bandwidth with lower latency (\cite{gdb:latency},
\cite{gdb:bandwidth}).
These developments show the importance of high performance computing
in this sector.


In this chapter we investigate how high performance computing can be
applied in finance. This is done as a case study in which a
European call option is priced using parallel Monte Carlo
simulation and the outcome is compared to the analytical solution.

This chapter is organized as follows: First, a short overview of the
theory of option pricing is given, starting with a definition of
derivatives and continuing with the explanation of the Black-Scholes
model. Then the theory of Monte Carlo simulation of option prices is
summarized. Before showing the results of the parallel Monte Carlo
simulation the implementation is described.


\input{section_options.tex}

\section{Conclusion}

In this chapter we presented a possible application of parallel
computing in computational finance--a field with an increasing demand
for high performance computing.

Parallel Monte Carlo simulation makes extensive use of parallel random
number generators and therefore they have a major impact on the
overall performance. Further research is promising in this area
as only few alternatives exist. Furthermore, Monte Carlo simulation is
not only an issue in finance but also for other problems which can
only be solved numerically.

All in all in this case study we have shown that high scalability can
be achieved for this type of application as work can nearly be
perfectly divided among the slaves and only few 
data has to be sent via the network. Results showed good speedups
with increasing CPU counts.



