\chapter{Option Pricing using Parallel Monte Carlo Simulation}
\section{Introduction}
\label{sec:optionintro}
Derivatives are important in todays financial markets. Futures and
options have been increasingly traded since the last 20 years. Now,
there are many different types of derivatives.
\begin{description}
\item[A derivative] is an financial instrument whose value depends on
  the value of an other variable (or the values of other
  variables). This variable is called \textit{Underlying}.
\end{description}
Derivatives are traded on exchange-traded markets (people trade
standardized contracts defined by the exchange) or over-the-counter
markets (trades are done in a computer linked network or over the
phone - no physical contact).

There are two main types of derivatives, namely forwards (or futures)
and options.

\begin{description}
\item[A Forward] is a contract in which one party buys (long position)
  or sells (short postion) an
  asset at a certain time in the future for a certain price. Unlike
  forward contracts, \textit{futures} are standardized and therefore
  are normally traded on an exchange.
\item[A call Option] is a contract, which gives the holder the right
  to buy an asset at a certain time for a certain price. 
\item[A put Option] is a contract, which gives the holder the right
  to sell an asset at a certain time for a certain price. 
\end{description}

With this information given we can write down their payoff functions:\newline
for the call option
\begin{equation}\label{eq:call}
C_T = \left\{ \begin{array}{lcl} 0 & \textrm{when}& S_T \leq K \\
                         S_T - K & \textrm{when} & S_T > K, \end{array}\right.
\end{equation}
and
\begin{equation}\label{eq:put}
P_T = \left\{ \begin{array}{lcl} 0 & \textrm{when}& K \leq S_T \\
                         K - S_T & \textrm{when} & K > S_T \end{array}\right.
\end{equation}
for the put option.

Figure \ref{fig:payoffs} shows the payoff functions of a call and
a put option and the position of the investor (left:~long position,
right:~short position).


%% TODO: Figure fig:payoffs produced from .rnw

Forward Prices (or Future Prices) can be determined in a simple way
and therefore it is not computational interesting for this chapter.

\subsection{Random Walk}\label{sec:randomwalk}
Given a probability space ($\Omega,{\cal F},P$) we can differentiate
between stochastic processes in discrete time (sequences of random
variables $(X_n)_{n=0}^N$, $N \in\mathbf{N}$) and stochastic processes
in continous time $(X_t)_{0 \leq t \leq T}$.

A process $(X_n)_{n=0}^N$ with independent and identical distributed
increments~($Z_i ~ Z_1$ for $n=1,\ldots,N$)is called random walk.
$$ X_n = X_0 + Z_1 + Z_2 + \ldots + Z_n $$
Figure~\ref{fig:randomwalk} shows a random walk starting from $X_0 =
10$ and $Z_i ~ Z_1 ~ N(0,1)$.

%% TODO: Figure produced from RNW


\subsection{Binomial process}

A random walk, which can only have 1 of 2 values $u$ and $d$ as increments, is
called a binomial process.

There is a $0 \leq p \leq 1$ where the probability $P(Z_n = u)$ equals
$p$ and the probability $P(Z_n = d)$ equals $1 - p$.

Figure~\ref{fig:binomial} shows a binomial tree (relevant for the
pricing of the option) and a path of a binomial process. 


%% TODO: Figure produced from RNW

\subsection{Wiener process}

A process $(W_t)_{0 \leq < \infty}$ is a Wiener process if
\begin{enumerate}
\item $W_0 = 0$,
\item the paths are continous,
\item all increments $W_{t_1}, W_{t_2} - W_{t_1}, \ldots W_{t_n} - W_{t_{n-1}}$
  are independent and normal distributed for all $0 < t_1 < t_2 < \ldots < t_n$ 
\end{enumerate}


%% TODO: Figure produced from RNW

In figure \ref{fig:wienerpath} one can see an example of a Wiender
path.

\subsection{Black-Scholes model}

There is a market which consists of
\begin{itemize}
\item a bank account process $B_t = e^{rt}$
\item and a share $S_t = S_0 e^{(\mu - \frac{\sigma^2}{2}) t + \sigma W_t}$. 
\end{itemize}

Equation~\ref{eq:blackscholes} gives the price of a European call
option of time $t$ with maturity $T$ and payoff function $h(S_T)$.
\begin{equation}\label{eq:blackscholes}
 F(t,x) = \int e^{-r(T-t)}h(xe^{(r - \frac{\sigma^2}{2})(T-t)+\sigma \sqrt{T-tz}})\phi(z)dz 
\end{equation}

\section{Implementation}
\section{Results}

