\chapter{Option Pricing using Parallel Monte Carlo Simulation}
\label{chap:options}
\section{Introduction}
\label{sec:optionintro}

The search for competitive advantages and especially in economics and in
financial markets is in human's nature. The biggest advantage a trader
in financial markets can have is to know something before all of the
other market participants. On the stock exchange quickness and an advantage
in time are the key factors to be successful. On todays financial
markets more and more trades are made by software and their
algorithms (this is called algorithmic-trading) or even artificial
intelligence (\cite{bloomberg:hal9000}). A human trader could
process five to ten complex 
transactions per minute software in contrast could process several
thousand transactions per second and parallelism preconditioned, look
after several market places simultanously.


The performance of the electronic exchange system is of great
importance nowadays. Every millisecond counts and therefore financial
firms keep their trading software as close to the trading system as
possible (\cite{wstonline:cuttingedge}). This is because investors
want to keep the time
between the input of a trade until its confirmation (this is called
round-tripping) at a minimum. Recently the Gruppe Deutsche B\"orse
upgraded their trading 
system Xetra to offer more bandwidth with lower latency (\cite{gdb:latency},
\cite{gdb:bandwidth}).
These developments show the importance of high performance computing
in this sector.


In this chapter a first attempt is made to contribute to the high performance
computing segment in finance. This is done as a case study in which a
European call option is priced using parallel Monte Carlo
simulation and the outcome is compared to the analytical solution.

This chapter is organized as follows: First, a short overview of the
theory of option pricing is given, starting with a definition of
derivatives and continuing with the explaination of the Black-Scholes
model. Then the theory of Monte Carlo simulation of option prices is
summarized. Before showing the results of the parallel Monte Carlo
simulation the implementation is described.


\input{section_options.tex}

\section{Conclusion}

