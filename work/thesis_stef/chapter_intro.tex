\chapter{Introduction}
\section{Motivation}
During the last decades the demand for computing power has steadily
increased, since problems like climate prediction, data mining, difficult
computations in physics, or optimizations in various fields
make use of more accurate and therefore time consuming models. It is
obvious that minimizing time needed for these calculations has become
an important task. Therefore a new field in computer science has come
up---high performance computing.

Being able to write code which can be executed in parallel is a key
ability in this field because high performance computing servers
provide in general more than just one processor. But one has to
consider that
parallelization of sequential code is a difficult task and is indeed
another access in writing programs. To help scientists and software
developers standards and parallelizing techniques like MPI or PVM have
been introduced 
to make their work easier. Nevertheless, sequential programs have to
be rewritten, or completely new applications have to be designed to
make use of this enormous parallel computing power. Another approach
has become popular again recently---compiler driven
parallelization. Though writing parallel programs can be achieved
more easily the performance cannot be compared to code which has been
parallelized manually. All in all,
writing efficient code remains a challenging, time consuming task and
needs years of experience. 

Therefore, the major aim of this thesis is not only to 
provide a good overview of the work which has been done in this field
but more importantly to become familiar with the state of the art of
parallel computing.

Moreover, a lot of  time consuming models exist in a scientific field
called computational statistics; a field which has many connections to other
scientific areas (e.g., economics, physics, biology or
engineering). As a computational environment a software called
R (\cite{Rcore07R}) is available. This project is on its way to provide 
high performance computing to statistical computing  through various
extensions. To become familiar with the capabilities of R in
this area is of major interest as it would be a key for solving many
problems not only in statistics but also in all other areas where
computational statistics plays a major role. In this thesis parallel
Monte Carlo simulation used in finance to price options has been
chosen as a case study.

\section{Contributions}
In the course of this thesis an extension to the R
environment for statistical computing has been developed---the
\pkg{paRc} package (\cite{theussl07paRc}. First, it provides a
benchmark environment for 
identifying performance gains of parallel algorithms. Results of
parallel matrix multiplication were produced using this benchmark
environment. Second, a first
attempt to use OpenMP (\cite{openMP05} in combination with R has been
made and a 
C~interface to OpenMP library calls has been implemented. Finally,
a framework for pricing options with parallel Monte Carlo simulation
has been developed.
 
\section{Organization of this Thesis}

This thesis is organized as follows. First I briefly summarize the
fundamentals of high performance computing and parallel processing in
Chapter~\ref{chap:parallelcomputing}. In the course of this chapter
computer architectures and programming models are described as well as the
principles of performance analysis.

Chapter~\ref{chap:Rhpc} includes explanations of how to use these
paradigms in R. This part of
this thesis provides an overview of the
available extensions supplying parallel computing functionality to R.

In the remaining chapters case studies for high performance
computing are presented. Each of these chapters has the following
structure: First,
a description of the topic is given in a comprehensive way. Second
details of the implementations are explained. Finally, results of the
comparison between each implementation are presented. The selected
topics are
\begin{itemize}
\item Parallel matrix multiplication in Chapter~\ref{chap:matrix},
%\item Parallel Greedy Randomized Adaptive Search procedure in
%  Chapter~\ref{chap:pgrasp},
\item Option pricing using Parallel Monte Carlo simulation in
  Chapter~\ref{chap:options}. 
\end{itemize}
Eventually Chapter~\ref{chap:conclusion} summarizes
the findings and give economical interpretations as well as an outlook
to future work.
